% Options for packages loaded elsewhere
\PassOptionsToPackage{unicode}{hyperref}
\PassOptionsToPackage{hyphens}{url}
%
\documentclass[
]{book}
\usepackage{amsmath,amssymb}
\usepackage{lmodern}
\usepackage{ifxetex,ifluatex}
\ifnum 0\ifxetex 1\fi\ifluatex 1\fi=0 % if pdftex
  \usepackage[T1]{fontenc}
  \usepackage[utf8]{inputenc}
  \usepackage{textcomp} % provide euro and other symbols
\else % if luatex or xetex
  \usepackage{unicode-math}
  \defaultfontfeatures{Scale=MatchLowercase}
  \defaultfontfeatures[\rmfamily]{Ligatures=TeX,Scale=1}
\fi
% Use upquote if available, for straight quotes in verbatim environments
\IfFileExists{upquote.sty}{\usepackage{upquote}}{}
\IfFileExists{microtype.sty}{% use microtype if available
  \usepackage[]{microtype}
  \UseMicrotypeSet[protrusion]{basicmath} % disable protrusion for tt fonts
}{}
\makeatletter
\@ifundefined{KOMAClassName}{% if non-KOMA class
  \IfFileExists{parskip.sty}{%
    \usepackage{parskip}
  }{% else
    \setlength{\parindent}{0pt}
    \setlength{\parskip}{6pt plus 2pt minus 1pt}}
}{% if KOMA class
  \KOMAoptions{parskip=half}}
\makeatother
\usepackage{xcolor}
\IfFileExists{xurl.sty}{\usepackage{xurl}}{} % add URL line breaks if available
\IfFileExists{bookmark.sty}{\usepackage{bookmark}}{\usepackage{hyperref}}
\hypersetup{
  pdftitle={Malawi Open NAP},
  pdfauthor={Open NAP Initiative},
  hidelinks,
  pdfcreator={LaTeX via pandoc}}
\urlstyle{same} % disable monospaced font for URLs
\usepackage{color}
\usepackage{fancyvrb}
\newcommand{\VerbBar}{|}
\newcommand{\VERB}{\Verb[commandchars=\\\{\}]}
\DefineVerbatimEnvironment{Highlighting}{Verbatim}{commandchars=\\\{\}}
% Add ',fontsize=\small' for more characters per line
\usepackage{framed}
\definecolor{shadecolor}{RGB}{248,248,248}
\newenvironment{Shaded}{\begin{snugshade}}{\end{snugshade}}
\newcommand{\AlertTok}[1]{\textcolor[rgb]{0.94,0.16,0.16}{#1}}
\newcommand{\AnnotationTok}[1]{\textcolor[rgb]{0.56,0.35,0.01}{\textbf{\textit{#1}}}}
\newcommand{\AttributeTok}[1]{\textcolor[rgb]{0.77,0.63,0.00}{#1}}
\newcommand{\BaseNTok}[1]{\textcolor[rgb]{0.00,0.00,0.81}{#1}}
\newcommand{\BuiltInTok}[1]{#1}
\newcommand{\CharTok}[1]{\textcolor[rgb]{0.31,0.60,0.02}{#1}}
\newcommand{\CommentTok}[1]{\textcolor[rgb]{0.56,0.35,0.01}{\textit{#1}}}
\newcommand{\CommentVarTok}[1]{\textcolor[rgb]{0.56,0.35,0.01}{\textbf{\textit{#1}}}}
\newcommand{\ConstantTok}[1]{\textcolor[rgb]{0.00,0.00,0.00}{#1}}
\newcommand{\ControlFlowTok}[1]{\textcolor[rgb]{0.13,0.29,0.53}{\textbf{#1}}}
\newcommand{\DataTypeTok}[1]{\textcolor[rgb]{0.13,0.29,0.53}{#1}}
\newcommand{\DecValTok}[1]{\textcolor[rgb]{0.00,0.00,0.81}{#1}}
\newcommand{\DocumentationTok}[1]{\textcolor[rgb]{0.56,0.35,0.01}{\textbf{\textit{#1}}}}
\newcommand{\ErrorTok}[1]{\textcolor[rgb]{0.64,0.00,0.00}{\textbf{#1}}}
\newcommand{\ExtensionTok}[1]{#1}
\newcommand{\FloatTok}[1]{\textcolor[rgb]{0.00,0.00,0.81}{#1}}
\newcommand{\FunctionTok}[1]{\textcolor[rgb]{0.00,0.00,0.00}{#1}}
\newcommand{\ImportTok}[1]{#1}
\newcommand{\InformationTok}[1]{\textcolor[rgb]{0.56,0.35,0.01}{\textbf{\textit{#1}}}}
\newcommand{\KeywordTok}[1]{\textcolor[rgb]{0.13,0.29,0.53}{\textbf{#1}}}
\newcommand{\NormalTok}[1]{#1}
\newcommand{\OperatorTok}[1]{\textcolor[rgb]{0.81,0.36,0.00}{\textbf{#1}}}
\newcommand{\OtherTok}[1]{\textcolor[rgb]{0.56,0.35,0.01}{#1}}
\newcommand{\PreprocessorTok}[1]{\textcolor[rgb]{0.56,0.35,0.01}{\textit{#1}}}
\newcommand{\RegionMarkerTok}[1]{#1}
\newcommand{\SpecialCharTok}[1]{\textcolor[rgb]{0.00,0.00,0.00}{#1}}
\newcommand{\SpecialStringTok}[1]{\textcolor[rgb]{0.31,0.60,0.02}{#1}}
\newcommand{\StringTok}[1]{\textcolor[rgb]{0.31,0.60,0.02}{#1}}
\newcommand{\VariableTok}[1]{\textcolor[rgb]{0.00,0.00,0.00}{#1}}
\newcommand{\VerbatimStringTok}[1]{\textcolor[rgb]{0.31,0.60,0.02}{#1}}
\newcommand{\WarningTok}[1]{\textcolor[rgb]{0.56,0.35,0.01}{\textbf{\textit{#1}}}}
\usepackage{longtable,booktabs,array}
\usepackage{calc} % for calculating minipage widths
% Correct order of tables after \paragraph or \subparagraph
\usepackage{etoolbox}
\makeatletter
\patchcmd\longtable{\par}{\if@noskipsec\mbox{}\fi\par}{}{}
\makeatother
% Allow footnotes in longtable head/foot
\IfFileExists{footnotehyper.sty}{\usepackage{footnotehyper}}{\usepackage{footnote}}
\makesavenoteenv{longtable}
\usepackage{graphicx}
\makeatletter
\def\maxwidth{\ifdim\Gin@nat@width>\linewidth\linewidth\else\Gin@nat@width\fi}
\def\maxheight{\ifdim\Gin@nat@height>\textheight\textheight\else\Gin@nat@height\fi}
\makeatother
% Scale images if necessary, so that they will not overflow the page
% margins by default, and it is still possible to overwrite the defaults
% using explicit options in \includegraphics[width, height, ...]{}
\setkeys{Gin}{width=\maxwidth,height=\maxheight,keepaspectratio}
% Set default figure placement to htbp
\makeatletter
\def\fps@figure{htbp}
\makeatother
\setlength{\emergencystretch}{3em} % prevent overfull lines
\providecommand{\tightlist}{%
  \setlength{\itemsep}{0pt}\setlength{\parskip}{0pt}}
\setcounter{secnumdepth}{5}
\usepackage{booktabs}
\ifluatex
  \usepackage{selnolig}  % disable illegal ligatures
\fi
\usepackage[]{natbib}
\bibliographystyle{apalike}
\newlength{\cslhangindent}
\setlength{\cslhangindent}{1.5em}
\newlength{\csllabelwidth}
\setlength{\csllabelwidth}{3em}
\newenvironment{CSLReferences}[2] % #1 hanging-ident, #2 entry spacing
 {% don't indent paragraphs
  \setlength{\parindent}{0pt}
  % turn on hanging indent if param 1 is 1
  \ifodd #1 \everypar{\setlength{\hangindent}{\cslhangindent}}\ignorespaces\fi
  % set entry spacing
  \ifnum #2 > 0
  \setlength{\parskip}{#2\baselineskip}
  \fi
 }%
 {}
\usepackage{calc}
\newcommand{\CSLBlock}[1]{#1\hfill\break}
\newcommand{\CSLLeftMargin}[1]{\parbox[t]{\csllabelwidth}{#1}}
\newcommand{\CSLRightInline}[1]{\parbox[t]{\linewidth - \csllabelwidth}{#1}\break}
\newcommand{\CSLIndent}[1]{\hspace{\cslhangindent}#1}

\title{Malawi Open NAP}
\author{Open NAP Initiative}
\date{2021-05-10}

\begin{document}
\maketitle

{
\setcounter{tocdepth}{1}
\tableofcontents
}
\hypertarget{prerequisites}{%
\chapter{Prerequisites}\label{prerequisites}}

This is a \emph{sample} book written in \textbf{Markdown}. You can use anything that Pandoc's Markdown supports, e.g., a math equation \(a^2 + b^2 = c^2\).

The \textbf{bookdown} package can be installed from CRAN or Github:

\begin{Shaded}
\begin{Highlighting}[]
\FunctionTok{install.packages}\NormalTok{(}\StringTok{"bookdown"}\NormalTok{)}
\CommentTok{\# or the development version}
\CommentTok{\# devtools::install\_github("rstudio/bookdown")}
\end{Highlighting}
\end{Shaded}

Remember each Rmd file contains one and only one chapter, and a chapter is defined by the first-level heading \texttt{\#}.

To compile this example to PDF, you need XeLaTeX. You are recommended to install TinyTeX (which includes XeLaTeX): \url{https://yihui.org/tinytex/}.

\hypertarget{executive-summary}{%
\chapter{Executive Summary}\label{executive-summary}}

Geographically, Malawi is a landlocked country in southern Africa bordering Mozambique, Tanzania, and Zambia (Masi 2017). The country has a total area of 118,484
km2 of which 20\% is covered by Lake Malawi. The country's topography is varied. In the mountainous sections of Malawi surrounding the Rift Valley, plateaus rise
generally 800 m to 1,200 m above sea level, although some rise as high as 3,000 m in the north. Malawi experiences sub-tropical climate conditions and annual
changes between wet and dry seasons. The wet season generally occurs between November and April and the dry season between May and October. Average temperatures
range between 18° and 27°C, and the wet season can bring average monthly rainfall in the order of 150 mm to 300 mm (Masi 2017). Annual rainfall ranges from 500 mm
in low-lying areas such as the Shire Valley to above 3,000 mm in the northern highlands (USAID 2017a).

Malawi is characterized by widespread poverty, and a rapidly growing population with high population density, putting pressure on land, fisheries, water and other
natural resources (Masi 2017). Malawi is already experiencing some of the effects of climate change with observed rising temperatures and changes in the variability
of rainfall (Masi 2017). Adverse impacts have already resulted in considerable damage, disrupted economic activity and adversely affected the lives of large number
of people, particularly the poor who are the most vulnerable to weather related shocks (Masi 2017). Challenges resulting from climate change include (Masi 2017):
dry spells and seasonal droughts linked to crop failures, food security and nutrition availability; intense rainfall associated with severe riverine and flash
floods and damaging infrastructure including roads, bridges, schools and health facilities; soil erosion due to intense rainstorms combined with ongoing degradation
of upstream catchments causing high sediment deposition loads in rivers hence massive siltation in Lake Malawi that adversely affects hydropower energy generation;
heat stress and outbreaks of livestock diseases like Newcastle disease in chickens and African Swine Fever in pigs; degraded grazing fields resulting to low fodder
availability and quality; competition for resources like water and grazing land; denudation of forests and woodlands driven by biomass energy demand also causing
biodiversity loss; increase in disease incidence and transmission of cholera, schistosomiasis and malaria.

Malawi is experiencing climate related hazards and extreme events which are increasing vulnerability of the communities to climate change across all sectors
(Irish Aid 2018) with reports of extreme weather events (that is, droughts, heavy rains, and floods) increasing from just one during the 1970s to nineteen
between 2000 and 2006 (Hughes et al.~2019). Mean temperatures have risen by an average rate of 0.21°C per decade, with comparative increases in
evapotranspiration (Hughes et al.~2019). Extreme weather events that occur frequently in the country include dry spells, seasonal droughts, intense rainfall,
riverine floods and flash floods (Masi 2017). Impacts include the Phalombe flash floods in 1991 that killed over 1,000 people, and wiped out villages, crops,
livestock and property (REF) and an intensive 2015 flood event in XX area that left many lives and livelihoods destroyed (Irish Aid 2018). The effects of the
climate changes and extreme weather events are compounded by a number of other factors. Extensive land use, including the massive cutting down of trees on the
Middle and Upper Shire Valleys, has resulted in severe land degradation and soil erosion, leading to siltation of the Shire River and its tributaries, seriously
affecting hydro-electric power generation, human health and fisheries (UNFCCC 2006). Soil degradation which is a major challenge in Malawi has soil losses
averaged at 20 T/ha/year translating to a 4\% - 25\% annual yield loss (Irish Aid 2019). The average annual national soil loss rates were estimated at 29 tons per
hectare in 2014, with soil erosion and nutrient depletion reported to affect more than 60\% of Malawi's land area. Unsustainable farming practices, an increased
demand for agricultural land and wood fuels associated with a growing population have all been attributed to cause this degradation with chemical land
degradation, including soil pollution and salinization/ alkalization, leading to 15\% loss in the total arable land in Malawi in the last decade alone. Between
2008 and 2016, majority of urban households relied on biomass energy with a 35\% increased charcoal demand worth more than USD 66 million in 2016 providing
employment opportunities for over 235,000 people (Hughes et al.~2019). This has a huge impact on agriculture which is the main economic activity of the country
contributing to over 80\% of the country's GDP.

Malawi is among the dozen most vulnerable countries globally in terms of adverse effects of climate change, especially drought, but also floods/heavy rains.
Heavy dependence on rain-fed agriculture of both the national and local economies, and for the livelihoods of the majority (85\%) rural population makes Malawi
particularly vulnerable. The rains can start as early as October, especially in the south of the country and can end as late as May, especially in the north of
the country (Malawi, 2015). This early rains and extended rains disrupt the agricultural cycle hence having a negative impact on food production in the country.
Factors including high population density and poverty, small landholding sizes, and the low-input low-output farming systems exacerbate farmers' vulnerability
and reduce the resilience of agricultural systems and adaptive capacity of farming communities to effectively respond to adverse CC impacts or take advantage of
emerging opportunities (Zulu 2017). Malawi, with a 3.06\% annual growth rate (Masi 2017), has high incidences of poverty, violence, unemployment, malnutrition,
HIV and AIDS, high illiteracy rates, poor health, and psychological disorders which characterize the country's young population (MDGS II 2011-2016) (Irish Aid,
2018). About 85\% of the people live in rural areas and derive their livelihoods from natural resources and agriculture (from small land holdings of between 1.0
and 5.0 ha per household of five people), with the remaining 15\% residing in urban areas (Malawi Vision 2020). The changes in climate and land cover and use are
exacting significant adverse impacts on the economy of Malawi. A 1-in-10 year drought event would have an estimated adverse impact of 4\% on the annual GDP of
Malawi, with even larger impacts for 1-in-15 and 1-in-25 year events (Malawi 2015). The Government of Malawi (GoM) has estimated that 29 metric tons of soil per
hectare are lost each year, costing the country an estimated 8\% of its annual gross domestic product (GDP) (GOM 2001) (USAID 2017b) and for the period 2001 to
2009, the annual costs of land degradation have been estimated at USD 244 million per year, an amount equivalent to 6.8\% of Malawi's country's GDP. There has
been migration from rural to urban areas (at the rate of 3.6\% per year), and from densely populated to sparsely populated areas or districts over the decades
from areas adversely affected by climatic hazards (especially floods and drought) to safer upland areas or other districts (MoECCM 20181) and in search of income
earning opportunities (Malawi
Vision 2020).

National Adaptation Plans (NAPs) are generally important in several ways. For instance, if countries fail to build resilience of people, places, ecosystems and
economies to the impacts of climate change, they risk losing the hard won sustainable development gains. The most unfortunate part is that poor countries are
more vulnerable to the devastating impacts of climate with Malawi being one of the poorest countries in the world, ranked 170 of 188 countries on the global
United Nations Development Programme's HDI. Given the climate related challenges faced by Malawi, a NAP will identify and provide a roadmap on key adaptation
measures required to address key adaptation needs and processes to ensure that these measures are mainstreamed into the national planning and development
processes and programmes across systems and sectors. The country's Intended National Determined Contribution INDC noted the need to enhance resilience of
productive sectors like rain fed agriculture to the associated negative impacts of climate change. The 2016 Malawi National Climate Change Policy noted the need
to effectively manage the impacts of climate change through interventions that build and sustain the social and ecological resilience of all Malawians; with the
regulation of greenhouse gas emissions to the atmosphere at a level that would prevent dangerous human-induced interference with the climate system within a
timeframe that enables social, economic and environmental development to proceed in a sustainable manner. It notes that climate change needs to be integrated
into planning, development, coordination and monitoring of key relevant sectors in a gender sensitive manner and through an appropriate institutional framework.
The 2006 NAPA sought to increase the adaptive capacities of vulnerable communities to adverse effects of climate change through a number of initiatives, such as:
improving community resilience to climate change by the development of sustainable rural livelihoods; restoring forest in Upper, Middle and Lower Shire Valleys
catchments to reduce siltation and the associated water flow problems; improving agricultural production under erratic rains and changing climatic conditions;
improving Malawi's preparedness to cope with droughts and floods, and; improving climate monitoring to enhance Malawi's early warning capability and decision
making and sustainable utilization of Lake Malawi and lakeshore areas resources. The NAP process seeks to reduce vulnerability to the impacts of climate change
by building adaptive capacity and resilience while integrating climate change adaptation into relevant new and existing national development policies, programs
and activities.

\hypertarget{framework-for-the-nap}{%
\chapter{Framework for the NAP}\label{framework-for-the-nap}}

As indicated in the introduction section above, Malawi's geographical characteristics and the prevailing socioeconomic conditions among the majority of its
population, makes it one of the most vulnerable countries to the impacts of climate change globally. The country has been experiencing unpredictable weather
patterns characterized by poor distribution of rainfall, causing dry spells, droughts and floods. Devastating droughts and floods witnessed in recent years and high
temperatures cause food insecurity affecting millions of its population through low agricultural yields as a result of reduced soil moisture and inflated food
prices. Drought lowers hydroelectric power production in the Shire River by reducing the flow rates in the river as a result of complete drying up of some of the
tributaries that feed into Lake Malawi. Lake Chilwa, a notable wetland, is drying up. These have made agricultural production and the country's agro-based economy
extremely vulnerable. Land degradation and loss of soil fertility, decreasing availability of safe water for humans and livestock as water tables recede, forest
fires, floods resulting in severe crop loss and infrastructure damage including roads and the only rail line that links the south to the centre, all result in
serious socio-economic disruptions, food and water insecurity, and diseases such as diarrhoea, cholera and malaria. Increased temperatures, droughts, and floods
will also result in a range of direct and indirect impacts to health, with malaria being of particular concern to Malawi because as temperatures becomes warmer, it
will become more suitable for breeding of mosquitoes even at higher altitudes, which historically have not been exposed to the disease. All these changes among
others are depressing economic activities, with significant impact on national GDP, and diminishing the wellbeing of the large population of rural dwellers (85\%)
whose livelihoods depend on wetlands, livestock and natural resources, as well as the urban poor who have to contend with unemployment and inequality

The National Adaptation Planning process which was initiated during the seventeenth session of the Conference of the Parties (COP) to the United Nations Framework
Convention on Climate Change (UNFCCC) is today an essential component of planning at all levels because climate change is an issue that has to be addressed over the
long-term. The process enables developing and least developed country (LDC) parties to assess their vulnerabilities, mainstream climate change risks, and to address
adaptation across all key sectors that are impacted by climate change (LEG, 2012). Further, it is essential that developing country and LDC parties integrate
adaptation planning in the broader context of sustainable development planning2 because climate change risks disproportionately magnify development challenges in
these countries as compared to developed countries (LEG, 2012). The national adaptation plan (NAP) process was, therefore, established by the COP as a pathway by
which effective adaptation planning in LDCs and other developing countries can be facilitated. The Government of Malawi embarked upon the National Adaptation Plan
(NAP) process to adopt a medium-term approach for reducing vulnerability to climate change impacts, and to facilitate the integration of climate adaptation into
ongoing planning processes at national and subnational levels.

The agreed objectives of the national adaptation plan process are (LDC-EG, 2012): (a) To reduce vulnerability to the impacts of climate change, by building adaptive
capacity and resilience; (b) To facilitate the integration of climate change adaptation, in a coherent manner, into relevant new and existing policies, programmes
and activities, in particular development planning processes and strategies, within all relevant sectors and at different levels, as appropriate.

The implementation of the NAP process is intended to:

\begin{itemize}
\item
  build on existing CCA planning processes and initiatives in order to provide continuity with previous planning efforts;
\item
  build on past implementation successes;
\item
  eliminate duplication of effort; and
\item
  avoid repetition of implementation failures.
\end{itemize}

\hypertarget{essential-functions-of-the-nap-process}{%
\subsection{Essential functions of the NAP process}\label{essential-functions-of-the-nap-process}}

The NAP for Malawi will serve the following functions:

\begin{enumerate}
\def\labelenumi{\arabic{enumi}.}
\item
  Enhanced institutional coordination- Provision of oversight on climate change activity implementation by NSCCC and the NTCCC providing a platform for
  efficient and effective implementation of national, regional, and global partnerships on climate change.
\item
  Strengthen the capacity of Malawi's government at all levels to implement a NAP process. MDAs will provide the data and information needed at various stages
  of the NAP process.
\item
  Nationally agreed adaptation targets that are mainstreamed into sectoral strategies like the MGDS III and policies which will provide for building of climate
  change resilience through regular development budgets. National Climate Change Investment Plan will assist the NAP process in resource mobilization.
\item
  A timetable and a work-plan to harmonize the main policy inconsistencies across Malawi's policy and legal frameworks that are relevant to climate change
  adaptation, which again will provide for building of climate change resilience through regular development budgets.
\item
  Incentivized government technical officers through professional development strategies. Capacity development will entail holding regular working group
  meetings and developing training programs for working groups based on prior training needs assessment. Working group meetings will come up with terms of
  reference and a training program for climate risk and vulnerability assessments, economical appraisal and design of adaptation pathways.
\item
  Tools and mechanisms established to promote iterative adaptation planning. Relevant institutions, individuals and organizations involved in CCA will be
  encouraged to adopt and use this CCA blueprint to build climate change resilience and contribute to the sustainable socioeconomic development of the country.
\item
  Enhanced access to adaptation finance that delivers the country's adaptation targets effectively. The National Climate Change Investment Plan and the National
  Climate Change Fund both have stipulated how they will manage fiduciary risks in dealing with the financial resources. Financial integrity in the NAP process
  will be further assured by adhering to government operating procedures on financial management and procurement as contained in Malawi's Financial Management Act.
  In addition, the NAP budget will be tabled by the Minister of Finance to the National Assembly during presentation of the annual government budget for approval.
  All NAP work-plans will be presented to the National Technical Committee on Climate Change and the National Steering Committee on Climate Change for endorsement and approval. This will ensure accountability and transparency.
\item
  A promotion of private sector engagement in businesses that will meet market demand for adaptation technologies and services. This will be achieved through
  the engagement of the Malawi Confederation of Chambers of Commerce and Industry (MCCCI) as a go-between to coordinate and facilitate private sector
  engagement. There will have to be a clear plan/structure for regular and sustained engagement.
\item
  Identify and address capacity gaps and needs to ensure that adaptation strategies are properly designed and implemented.
\end{enumerate}

\hypertarget{the-nap-as-the-umbrella-programme-for-adaptation}{%
\subsection{The NAP as the umbrella programme for adaptation}\label{the-nap-as-the-umbrella-programme-for-adaptation}}

The National Adaptation Plan (NAP) addresses the effects of climate variability and climate change in Malawi with a systems approach -- a departure from a
sectoral approach. The framework prioritizes transformative investments for addressing the impacts of climate change on the national economy with a focus on
building the resilience of vulnerable communities. The NAP evolves from a background of experience in the National Adaptation Plan of Action (NAPA).
Contrastingly, the NAPA was designed to address urgent and immediate needs of the country, created to act as a channel through which the country could access
support quickly and take advantage of win-win measures that would avoid increased damages and be more expensive to implement in the future. The NAPA was designed
more than 10 years ago, when the country was experiencing heightened levels of vulnerability to floods, drought, and other adverse effects of climate change.
With emerging and additional science and knowledge about climate change and its impacts, this NAP provides a framework for awareness and capacity for medium- and
long-term adaptation in the various systems which support national socio-economic development. The current Malawi Vison 2063 (MW2063) -- aspires to embrace
ecosystem-based approaches in managing the environment. With climate change, Malawi has made commitment to develop systems to break the cycle of environmental
degradation and increase resilience, sustainable development and planning as well as the promotion of climate change adaptation, mitigation, technology transfer
and capacity building for sustainable livelihoods through Green Economy measures. The NAP framework is a direct contribution to the UNFCCC commitment and the
MW2063.

There are several development programmes and activities that are taking place in Malawi at national and local governments under national government ministries
and parastatals or through bilateral arrangements and partnerships with private sector entities, which need to be buttressed to be resilient to the impacts of
climate changes in order to be able to effectively contribute to targeted development outcomes. Among many others, these include, for example:

\begin{itemize}
\item
  Lilongwe Water and Sanitation Project Malawi by Lilongwe Water Board jointly with Lilongwe City Council to increase access to improved water services and
  safely managed sanitation services in Lilongwe City;
\item
  the Shire Valley Transformation Program in Chikwawa and Nsanje Districts in the south of Malawi to increase agricultural productivity and commercialization for
  targeted households and to improve the sustainable management and utilization of natural resources.
\item
  The Social Cash Transfer (SCT), locally known as Ntukula Pakhomo Programme by the Ministry of Gender, Children, Disability and Social Welfare to cushion the
  poor and marginalized;
\item
  The Public Works Programme (PWP) implemented by the Ministry of Local Government and Rural Development through the National Local Government Finance Committee
  (NLGFC) it provides regular payments to individuals in exchange for work, with the objective of decreasing chronic or shock-induced poverty and providing
  social protection.
\item
  The School Feeding Program implemented by the Ministry of Education to improve child nutrition, increase children's' ability to concentrate in class, promote
  enrolment and regular attendance'
\item
  The Fertilizer Input Subsidy Programme (FISP) implemented by the Ministry of Agriculture to reduce poverty and ensure the country's food security by fostering
  an increase in agricultural productivity levels.
\item
  The Cement and Malata Subsidy Programme that seeks to provide affordable access to building materials by the poor.
\item
  The Increasing Access to Clean and Affordable Decentralized Energy Services (IACADES) Project under the Ministry of Energy with funding from UNDP and GEF among
  other sources.
\item
  Community Energy, an energy company funded by the Scottish government aims to help support energy-inefficient countries and to implement new energy-based
  programs to provide electricity to rural areas, focusing on personal electricity and solar projects, as well as hydro and cooking stoves for communities in
  rural areas, with 104 rural communities benefitting so far from the installation of install personal renewable energy units. Twelve districts of Malawi have so
  far received direction and access to these units, and each will begin to produce and regulate their own energy, with Community Energy's support.
\end{itemize}

Given its cross-cutting nature which includes, inter alia, considerations of adaptive capacity and resilience at systems level while considering also the
individual, institutional, and systemic factors, and its mainstreaming into governance and development planning structures, the NAP offers an appropriate
umbrella under which national programmes for adaptation can be jointly framed, coordinated and implemented. The NAP will present an aggregate national adaptation
plan that will link to appropriate local, subnational, national, and sectoral activities and action plans, maximise on efficiencies, minimise duplication of
efforts, and leverage on cost constraints to programme implementation. The NAP process shall also add value to past and current activities by identifying
capacity gaps, especially for the design and implementation of medium-term climate change adaptation priorities, as well as by accessing opportunities for
international funding to develop more effective climate responsive planning and budgeting. The Malawi NAP coalesces all the discrete climate change adaptation
plans and programmes that are or shall be implemented in the country. It comprises of collated, synthesised and analysed data of climate change trends and its
impacts, aggregated from local level and downscaled from regional analyses, as well as related peculiar vulnerabilities at scale within and across regions and
systems, and identifies gaps and capacity needs that should be addressed. This information will be used to identify and prioritise adaptation options, and to put
in place plans to implement the proposed adaptation options, as well as how to finance them. Finally, a monitoring and evaluation framework is determined for the
different programmes to track progress and to make adjustments where necessary. Periodic updates (every four years) shall be undertaken to ensure that the NAP is
responsive to new and emerging needs and offers an effective mechanism for climate change adaptation at scale.

\hypertarget{coherence-with-national-development-context-sdgs-sendai-and-other-relevant-frameworks}{%
\subsection{Coherence with national development context, SDGs, Sendai and other relevant frameworks}\label{coherence-with-national-development-context-sdgs-sendai-and-other-relevant-frameworks}}

The Malawi Vision 2020 is anchored on six pillars, namely; Good governance and a capable state, Human resource development and a knowledge-based economy, Private
sector-led development, Infrastructure development, Productive high value and market-oriented agriculture, and Regional and international integration. The vision
noted that air pollution and climate change issues though then relatively small could become serious challenges if unchecked. The vision hence identified the
strategic challenges to prevent air pollution and climate change issues as: monitoring emissions of hydrocarbons nitrogen oxides and carbon monoxides; proper
management of hazardous substances and wastes; use of ozone friendly technology; establishing regulations and enacting legislation on air pollution; and
promoting education on climate change issues. The Malawi vision statement highlights Malawi as an `environmentally sustainable,' `self-reliant with equal
opportunities for and active participation by all,' and `having social services, vibrant cultural and religious values and being a technologically driven middle-
income country.' This aligns with the vision of the NAP vision for Malawi of ``a country with people, ecosystems and infrastructure that are resilient and have
adaptive capacity to the impacts of climate change.''

The government of Malawi developed the National Climate Change Management Policy 2017-2027, to assist the country achieve its long term goal for climate change
management, which is to reduce the socio-economic impacts of adverse effects of climatic change. The NCCM policy is in line with other national strategies and
plans. For example, the Malawi Growth Development Strategy II 2011-2016 recognizes that climate change, environment and natural resources management as key
priority areas that needs to be responded to using appropriate approaches because it contributes to lower land quality, heightens extreme weather conditions
(e.g.~recurrent droughts, heavy rain falls and floods) which sometimes lead to emergency relief efforts that divert much needed finances from development
projects, and has significant adverse consequences for agriculture, food security, poverty and vulnerability. The process of developing the MGDS III 2017-2022
considered all the international commitments that Malawi made which include the SDGs, African Union Agenda 2063, SADC RISDP, and other regional treaties. The
government advocated for alignment of the SDG to all sector and institutional programming. This guaranteed that all development intervention from the cooperating
partners are well aligned towards the SDG timely tracking and reporting of all the agreed indicators. On the other hand, Malawi is also committed to implement
the Sendai Framework for Disaster Risk Reduction 2015-2030 as it strives to achieve various SDGs since Malawi is suffering the impacts of disasters both from
climate change as well another natural causes. The commitment goes beyond the 2030 agenda as it is clear that resilience building is paramount importance if the
development gains being achieved in all the national efforts should be sustained. Malawi is therefore well placed to enact and mainstream a NAP to operationalize
its approach to adaptation to climate change and to monitor progress towards desired outcomes.

\hypertarget{approach-and-methodologies}{%
\chapter{Approach and Methodologies}\label{approach-and-methodologies}}

\hypertarget{guiding-principles}{%
\section{Guiding principles}\label{guiding-principles}}

In line with the principles established by the UNFCCC and also in line with Malawi's development goals, the guiding principles for the NAP process are as follows:
developing sustainably; uplifting the poor and the vulnerable; respecting the critical role of gender; encouraging participation and ownership; incorporating
traditional and Indigenous knowledge, and proceeding with financial accountability and integrity.

\begin{enumerate}
\def\labelenumi{\alph{enumi}.}
\item
  A country-driven approach.
  country-driven approaches inspire ownership and ensure that plans, programmes and activities are aligned with national priorities.
\item
  Sustainable development
  Sustainable development is defined as ``development that meets the needs of the present without compromising the ability of future generations to meet their own
  needs'' (United Nations, 1987).
\item
  Uplifting the poor and the vulnerable
  Poor people in Malawi, who are also the majority, are the most affected by climate change impacts and have the least means of adapting to these impacts. Rural,
  urban and peri-urban poor people bear the brunt of climate-related disasters such as floods because their communities suffer from weak infrastructure. When drought
  and famine occur, the poor can also cope because of low incomes and reliance on rain-fed agriculture. Malawi's NAP will, therefore, among other principles, be
  guided by pro-poor principles to ensure inclusiveness. The NAP will ensure that the poor and vulnerable, including women and children, are targeted and benefit from
  the planning and implementing climate change adaptation interventions. The main objective of this principle is poverty reduction. This principle is in line with
  Malawi's Vision 20/20 and SDG 1.
\item
  Gender and social inclusion, and particular consideration of marginalized groups such as women
  The NAP will ensure that Malawi's Gender Policy (2015) principles---gender parity, women's empowerment and upholding women's rights---guide the process. The process will include the youth who are already engaged through various climate change youth networks
\item
  Participation NAP Process and ownership
  This is a critical guiding principle for the NAP. It will allow full involvement of stakeholders and beneficiaries in the NAP activities, thereby enabling
  information sharing and minimizing efforts' duplication. Soliciting the views of stakeholders at each step of the NAP will ensure their ownership, which will
  positively affect the outcomes. There are many actors in the climate change adaptation field that are already carrying out various activities. These will now be
  engaged with the NAP process guided by the framework, which will result in increased focus in terms of planning and funding for adaptation activities. This is
  important because adaptation activities have long been underfunded at both the central and district level. Stakeholder participation is necessary for buy-in,
  ownership, involvement in, and support of planned activities.
\item
  Incorporating traditional and indigenous
  While scientific methods of weather forecasting have evolved in the last 100 years or so, rural communities the world over have traditionally relied on
  Indigenous forecasting methods. In Malawi, communities have used local Indigenous methods to predict good or bad years by using cloud observations (appearance),
  wind directions, stars, and the behaviour of animals, insects and plants. Indigenous local knowledge of weather forecasting is useful in decision making at the
  village level. The NAP process will encourage integrating Indigenous knowledge with the scientific knowledge of weather forecasting. The process requires that
  communities be engaged to identify knowledge integrated with science, which could then be further disseminated for use by scientists, practitioners and policy-
  makers.
\item
  Financial accountability and integrity
  Resources allocated to climate change adaptation programs can greatly increase over time if there is confidence that these resources will be spent prudently, be
  quickly accessed, and produce the intended results. This calls for good fiduciary governance of the resources. The National Climate Change Investment Plan and
  the National Climate Change Fund have stipulated how they will manage fiduciary risks in dealing with the financial resources. Financial integrity in the NAP
  process will be further assured by adhering to government operating procedures on financial management and procurement as contained in Malawi's Financial
  Management Act. Besides, the NAP budget will be tabled by the Minister of Finance to the National Assembly during the annual government budget presentation for
  approval. All NAP workplans will be presented to the National Technical Committee on Climate Change and the National Steering Committee on Climate Change for
  endorsement and approval. This will ensure accountability and transparency.
\item
  A multidisciplinary and complementary NAP approach, building upon relevant existing plans and programmes
  Multidisciplinary and complementary approaches are essential in the NAP approach because adaptation is itself multidisciplinary and cross-cutting. The country
  has mainstreamed climate change issues in its development plans because it has implications for employment creation and economic growth. Its impact on various
  economic sectors such as agriculture, health and nutrition, tourism, and natural resources has been well established.
\item
  Simplicity and flexibility of procedures based on the country's circumstances
  Simplicity is important where actions are planned in multidisciplinary and multi-institutional/multi-agency contexts coupled with strong involvement of the
  public and private sector, communities and individuals. Flexibility is important, as adjustments can be made to improve different aspects of implemented
  programmes.
\item
  Alignment with the GCF country programme.
  This alignment is important to improve access to funds such as the Green Climate Fund. Such alignment would include coherency with the national climate change
  policy and related strategies and plans, coherence with existing policies, the executing entity's capacity to deliver, and stakeholder consultations and
  engagement.
\end{enumerate}

\hypertarget{guidelines-used}{%
\subsection{Guidelines used}\label{guidelines-used}}

The main guidelines used included:

· The Technical Guidelines for the National Adaptation Plan Process, UNFCCC -- this was used as the primary document for framing of the structure and content of the NAP. It also requires that the NAP process: follows a country-driven, fully transparent, approach; is based on and guided by the best available science and, as appropriate, traditional and indigenous knowledge; and facilitates country-owned, country-driven action and not be prescriptive, nor result in the duplication of efforts undertaken in-country.

\begin{itemize}
\item
  Malawi National Climate Change Policy-2017-2027
\item
  Malawi Second National Communication-2011
\item
  Malawi Vision 2020
\item
  The Malawi Growth Development Strategy 2017-2022
\item
  Malawi Intended Nationally Determined Contribution
\item
  National Adaptation Plan of Actions-2006
\item
  Malawi NAP Stocktaking Report 2016
\item
  National Climate Change Investment Plan (2013-2018)
\item
  National Environment and Climate Change Management 2012-2016
\item
  National Strategy for Sustainable Development
\item
  Malawi Strategy on Climate Change Learning
\end{itemize}

In addition, and following the experiences gathered from the implementation of the NAPA process, the Technical Guidelines recommend:

\begin{itemize}
\item
  using locally defined criteria for ranking vulnerabilities and prioritizing project activities, which will build confidence and buy-in across all stakeholders;
\item
  using available data and assessments as a basis for more comprehensive assessments; and
\item
  engaging national experts, as this will also enhance the experience and capacity of the country.
\end{itemize}

These were supported with emerging new data from the published literature. The assessment of these documents together included:

\begin{enumerate}
\def\labelenumi{\alph{enumi}.}
\tightlist
\item
  Process of identification/stocktaking of desirable and available information
\end{enumerate}

\begin{itemize}
\item
  \begin{enumerate}
  \def\labelenumi{\roman{enumi}.}
  \tightlist
  \item
    Climate and socio-economic data and information
  \end{enumerate}
\item
  \begin{enumerate}
  \def\labelenumi{\roman{enumi}.}
  \setcounter{enumi}{1}
  \tightlist
  \item
    Current assessments: Exploring possibilities for further assessments
  \end{enumerate}
\item
  \begin{enumerate}
  \def\labelenumi{\roman{enumi}.}
  \setcounter{enumi}{2}
  \tightlist
  \item
    Policies, strategies, plans
  \end{enumerate}
\item
  \begin{enumerate}
  \def\labelenumi{\roman{enumi}.}
  \setcounter{enumi}{3}
  \tightlist
  \item
    Existing initiatives on adaptation
  \end{enumerate}
\end{itemize}

\begin{enumerate}
\def\labelenumi{\alph{enumi}.}
\setcounter{enumi}{1}
\tightlist
\item
  Resource mobilization for the process.
\end{enumerate}

\hypertarget{a-systems-approach-to-adaptation}{%
\subsection{A systems approach to adaptation}\label{a-systems-approach-to-adaptation}}

Systems are complex, and each system interacts to various degrees with other related systems. Sectoral interventions have not been as successful as desired
because they do not take into account the interactions of system components, including the fact that the mandate to manage some components of the system may lie
in a different sector, and hence come under a different institutional mandate whose primary goal is not necessarily in tandem with those of another sector, and
more often than not, there is very little synergy between sectoral programmes.

Urban areas, for example, are complex since many social, physical and economic systems meet and interact, with many of these extending well beyond its spatial
boundaries, e.g.~water and power supply systems, while other linkages may be transboundary. It is important, therefore, that National Adaptation Plans capture
these systems and their interlinkages, scale and stakeholder diversity, so that appropriate and synergistic adaptation measures can be devised and implemented.
Thus, the NAP process uses a systems approach which facilitates the integration of climate change adaptation, in a coherent manner, into relevant new and
existing policies, programmes and activities, in particular development planning processes and strategies, within all relevant sectors and at different levels,
as appropriate.

The framework to guide the assessment of vulnerabilities and risks included:
- i. Conceptual framework of vulnerability and risk at various levels: national, system level, local level, etc.
- ii. Boundary conditions for the assessment using the period 1971-2000 for baseline climate but also extended further back into time where data is
available.
- iii. Focus on key systems/sectors
- iv. Synergy with SDGs, Sendai Framework for DRR, and other relevant regional and national frameworks.

\hypertarget{other-unique-considerations}{%
\subsection{Other unique considerations}\label{other-unique-considerations}}

The emergence of the COVID-19 pandemic in early 2020 disrupted globally, established societal structures and ways of doing things, and has had devastating
impacts on human health, stressed health systems and severely disrupted national economies. A UNDP 2020 study ``Covid-19 Pandemic in Malawi Final Report June
2020'' shows the high levels of vulnerabilities of individuals, households and the whole Malawian economy affecting negatively on almost all sectors of the
economic growth in the country. The study projects that the negative impacts of COVID-19 on the economy are projected to persist for more than 10 years. The open
NAP initiative in Malawi has been developed with Malawi being one of the 11 Africa-Asia-Pacific region beneficiaries of GCF funding of a mitigation-themed
project named Climate Investor One. However, more funding of projects from other funding sources like LDCF, SCCF, GEF and Adaptation Fund among other sources has
not been materialized perhaps due to lockdowns and poor internet access across the country affecting personnel availability to apply for funding as well as
undertake the projects. It is however hoped that the situation will normalize and that COVID-19 will be properly managed so as more funding will be availed to
Malawi to continue developing this NAP and more funds can be used to develop the next NAP.

\hypertarget{road-map}{%
\subsection{Road Map}\label{road-map}}

This particular process was initiated in 2016 with a stakeholder engagement workshop (Figure XX). Milestones in the process are illustrated in Table XX below,
with the goal of mainstreaming the NAP into the Malawi Growth and Development Strategy (MDGS III).

insert diag here

\hypertarget{national-context}{%
\chapter{National Context}\label{national-context}}

\hypertarget{national-circumstances}{%
\section{National circumstances}\label{national-circumstances}}

\textbf{Environment:} Malawi's landscape has a varied topography and is dominated by the Great Rift Valley, which runs north to south and contains Lake Malawi and the Shire River Valley. To the west are the central plateaus, highlands (Nyika and Viphya in the north and Shire in the south) and isolated mountains (Mulanje and Zomba) (USAID, 2017a). In the mountainous sections of Malawi surrounding the Rift Valley, plateaus rise generally 800m to 1,200m above sea level, with some especially in the north rising as high as 3,000m. To the south of Lake Malawi lie the Shire Highlands, approximately 900m above sea level. The Shire River plays a very significant role in Malawi by providing water for generating hydropower (98\% of Malawi's electricity), agriculture, fisheries, transport, tourism, urban and rural water supply along its length, impacting the livelihoods of over 5.5 million people in the southern region of Malawi (Masi, 2017). Freshwater for irrigation in Malawi's plantations such as Illovo Sugar at Nchalo is obtained from the Shire River; as well as other domestic and industrial uses (UNFCC, 2006). Malawi has multiple important waterbodies including Lake Malawi, (the third largest African Rift Valley Lake), Lakes Malombe, Lake Chilwa, and Lake Chiuta(USAID, 2015). Other rivers in Malawi providing water comprise of North and South Rukuru and Songwe in the Northern Region, Linthipe, Bua and Dwangwa in the Central Region, and Shire and Ruo in the Southern Region (Global Water Partnership, 2016).

In 2005, forest area coverage was at 24.3\% while cultivated land covered 33.7\%, shrubs and savanna woodlands covered 19.9\% and the remaining 22.1\% of Malawi was
covered by water. In the upper Shire River catchment, there was an 18 \% increase in agricultural land in the 1989 to 2002 period (Mtilatila et al.~2020). Forests
and trees impacts livelihoods and the economy through the supply of biomass fuels, provision of habitats for wildlife and biodiversity, prevention of land
degradation, protection of watersheds and acts as sources of soil fertility (Hughes et al.~2019). Malawi has the highest deforestation rate in sub-Saharan Africa
with the government of Malawi estimating that the annual rate of deforestation in Malawi is 1.0--2.8\%. Estimation shows that the ratio of forest area decreased from 51\% to 33\% from 1990 to 2010 (Mapulanga and Naito, 2019). Malawi has very low greenhouse gas (GHG) emissions of around 1.4 tons CO2 equivalents (CO2e) per capita in 2015 by global standards (Hughes et al.~2019). According to Malawi's Nationally Determined Contribution (NDC), the main sectors contributing to GHG emissions are as at 2015, forestry at 78\% of the emissions, agriculture at 16\% and energy at 4\% (Irish Aid, 2018).

Soil degradation is a major challenge with soil losses averaged at 20 T/ha/year. This translate in a yield loss of 4\% - 25\% annually (Irish Aid 2019). In 2014, the average annual national soil loss rates were estimated at 29 tons per hectare, and soil erosion and nutrient depletion are reported to affect more than 60\% of Malawi's land area. The main causes of this degradation are unsustainable farming practices, increasing demand for agricultural land and wood fuels associated with a growing population. Chemical land degradation, including soil pollution and salinization/ alkalization, has led to 15\% loss in the arable land in Malawi in the last decade alone. The annual costs of land degradation between 2001 and 2009 have been estimated at USD 244 million per year-an amount equivalent to 6.8\% of Malawi's country's GDP. Between 2008 and 2016, urban household demand for charcoal increased by 35\% and was worth more than USD 66 million in 2016 and provided employment opportunities for over 235,000 people (Hughes et al.~2019).

\textbf{Climate:} The majority of the country experiences a cool tropical continental climate, characterized by two distinct seasons: a rainy season from November to
April and a dry season from May to October. Annual rainfall ranges from 500 mm in low-lying areas such as the Shire Valley to above 3,000 mm in the northern
highlands. Overall rainfall exhibits high inter-annual variability and is highly influenced by the El Niño Southern Oscillation (USAID, 2017a). The rains can
start as early as October, especially in the south of the country and can end as late as May, especially in the north of the country (Malawi, 2015). The warm-wet
season stretches from November to April, during which 95\% of the annual precipitation takes place. Malawi experiences large heterogeneity in rainfall regime, and
there are big differences between the North, Central and South regions. Annual average rainfall varies from 725mm to 2,500mm with Lilongwe having an average of
900mm, Blantyre 1,127mm, Mzuzu 1,289mm and Zomba 1,433mm (Masi, 2017). In the south of Malawi, the wet season normally lasts from November to February bringing
around 150‐300m per month, but rain continues into March and April in the north of the country as the ITCZ migrates northwards. Inter‐annual variability in the
wet‐season rainfall in Malawi is also strongly influenced by Indian Ocean Sea Surface Temperatures, which can vary from one year to another due to variations in
patterns of atmospheric and oceanic circulation. The most well documented cause of this variability is the El Nino Southern Oscillation (ENSO) (UNDP, n.d.).

Average daily temperatures vary with seasons and elevation, with the coldest temperatures (12--15°C) in July in the highlands and the hottest (25--26°C) in October
in the Lower Shire Valley (USAID 2017a). Mean annual temperature has increased by 0.9°C between 1960 and 2006, an average rate of 0.21°C per decade (Irish Aid
2018). A cool, dry winter season runs from May to August with mean daytime temperatures varying between 17 and 27°C, and temperatures falling between 4 and 10°C
at night. A hot, dry season lasts from September to October with daytime temperatures between 25 and 37°C. The wet season generally occurs between November
and April and the dry season between May and October. Average temperatures range between 18° and 27°C, and the wet season can bring average monthly rainfall in
the order of 150mm to 300mm (Masi, 2017; UNDP, n.d.). Between 1967 and 2003, the country experienced six major droughts and incidences of flooding. 2011-12
droughts had severe effects on food security in many districts in Malawi, with approximately 2 million people affected, particularly in the southern districts.
(Irish Aid, 2018). Floods in Malawi have been associated with heavy upstream rainfall resulting in too much water downstream that leads to the breaking-up of
river banks. This is a common feature on the North Rukuru in Karonga, Likangala in Zomba, and the Ruo/Shire Rivers in Chikwawa/NsanjeMalawi has also experienced
flush floods due to prolonged torrential rains, such as the Phalombe flush floods in 1991 that killed over 1,000 people, and wiped out villages, crops, livestock
and property (UNFCC, 2006). Intensive flooding in 2015 left many lives and livelihoods destroyed (Irish Aid, 2018).

\textbf{Political context:} The Republic of Malawi is a sovereign State with rights and obligations under the Law of Nations (Malawi Constitution, Chapter one). There
shall be a President of the Republic who shall be Head of State and Government and the Commander-in-Chief of the Defense Forces of Malawi (Article 78). The
President shall be elected by a majority of the electorate through direct, universal and equal suffrage (Article 80(2)). The National Assembly of Malawi is the
supreme legislative body of the nation. The National Assembly has 193 Members of Parliament (MPs) who are directly elected in single-member constituencies using
the simple majority system and serve five-year terms. Malawi is a member of the United Nations, the Commonwealth of Nations, the Southern African Development
Community (SADC) (Malawi 2017), the Common Market for Eastern and Southern Africa (COMESA), and the African Union (AU). Malawi was a one party state since
attaining her independence until 1993 when it became a multi-party state (Masi, 2017).

\textbf{Legislative context:} The GoM prioritizes climate change, natural resources, and environmental management in its development strategy, the Malawi Growth and
Development Strategy (MGDS II 2012--2016). The GoM has also invested in the Green Belt Initiative (GBI); an initiative which seeks to transform Malawi, through
irrigation, from a predominantly consuming and importing country to a producing and exporting country (USAID, 2013). In 2016, Malawi made an ambitious 4.5
million hectares restoration pledge to the Bonn Challenge and the African Forest Landscape Restoration Initiative (AFR100) by 2030 estimated at a cost of
approximately 279 billion MWK or approximately 62000 MWK per hectare (USAID,2017b). GoM in partnership with the World Bank and African Development Bank has
formulated this Strategic Program for Climate Resilience (SPCR) under the Pilot Programme for Climate Resilience (PPCR) to act as a framework for addressing the
challenges of climate change that impact on the national economy and community livelihoods. The SPCR will build on the available enabling frameworks and efforts
in climate resilience-building programs as stipulated in the Malawi Growth and Development Strategy III, National Climate Change Management Policy (2016),
National Agriculture Policy (2016), National Climate Change Investment Plan (2013), and Malawi's Nationally Determined Contribution under the UNFCCC (2015).

Malawi is a signatory to various international treaties, instruments and that cover climate change. These include the United Nations Framework Convention on
Climate Change (UNFCCC) and the Kyoto Protocol. These treaties and instruments oblige the country to take various actions to address climate challenges including
putting in place instruments such as climate change policies and legislation. Malawi is a member of the Least Developed Countries' (LDCs) Group, the LDC Expert
Group (LEG), and currently has a seat on the board of the Adaptation Committee and the Green Climate Fund (GCF) Board (Masi, 2017). The Government has put in
place a series of legislative sectoral frameworks and strategies to integrate environment and climate change management in socio-economic development activities.
Key ones include: Vision 2020; the Malawi Growth and Development Strategies; National Environmental Policy (NEP) 2004; NAPA 2007; National Climate Change
Investment Plan (2013); Malawi Energy Policy (2003); Food Security Policy (2006); Disaster Preparedness and Relief Act (DPRA) (1991); Environment Management Act
(1996) and the Disaster Risk Management Policy 2015 (Irish Aid, 2018).

\textbf{Social context:} According to the World Population review, January 2018, Malawi has a land area of 118,484 square kilometers, with an estimated population of
18,921,352 million which ranks 61st in the world. Malawi still has a fairly low population density of 129 people per square kilometer (86th in the world).
However, Malawi is growing rapidly with a 3.06\% (Masi, 2017) annual growth rate. High incidences of poverty, violence, unemployment, malnutrition, HIV and AIDS,
high illiteracy rates, abuse, poor health, and psychological disorders characterize the country's young population (MDGS II 2011-2016) (Irish Aid, 2018). About
85\% of the people live in rural areas and derive their livelihoods from natural resources and agriculture (from small land holdings of between 1.0 and 5.0 ha per
household of five people), with the remaining 15\% residing in urban areas. About 48\% of the population is below 15 years of age. The overall average life
expectancy as of 2008 statistics was 37 years with fertility rates declining from 7.6 in 1984 to 2.8\% in 2008 and later rising to 6.7 (Malawi Vision 2020). There
has been migration from rural to urban areas (at the rate of 3.6\% per year), and from densely populated to sparsely populated areas or districts over the decades
from areas adversely affected by climatic hazards (especially floods and drought) to safer upland areas or other districts (Ministry of Environment and Climate
Change Management Environmental Affairs Department, 2018) and in search of income earning opportunities (Malawi Vision 2020). 33150 cases and 981 deaths were
recorded in Malawi's worst Cholera outbreak. Waterborne infectious diseases are a leading cause of child mortality and contribute to forms of growth retardation,
including stunting and wasting with 48 to 53 percent of children under the age of five suffering from stunted growth (Republic of Malawi, 2012). Overall, records
as to disaster damage provided by Department of Disaster Management Affairs, DoDMA and the Prevention Web (by The United Nations Office for Disaster Risk
Reduction, UNISDR) give critical information related to human and economic losses resulting from the disasters that have occurred in Malawi within last three
decades. More than 47 natural disasters were recorded in the last three decades and these disasters range from droughts, earthquakes, epidemics, floods and
storms. In these natural disasters, a total of 2,775 people were killed with an average of 90 people killed per year. Most of these (60\%) died due to epidemics
(National Water Resources Masterplan- Part II masterplan). Malaria is the most common disease in the lake areas, followed by respiratory infections, diarrhea,
anemia, and bilharzia/schistosomiasis. HIV/AIDS and other sexually transmitted infections (STIs) are also common. Research conducted by Madsen et al.~between
1998 and 2007 found a high prevalence of schistosomiasis in communities living along the shores of Lake Malawi. They found that the prevalence of urinary
schistosomiasis ranged from 10.2\% to 26.4\% in inland villages and from 21.0\% to 72.7\% in lakeshore villages. Infection rates were higher among school age
children ranging from 15.3\% to 57.1\% in inland schools and from 56.2\% to 94.0\% in lakeshore schools. The HIV infection rate in Malawi as a whole was 10.3\% in
2010 (UNAIDS) (USAID, 2015).

\textbf{Economic context:} Agriculture is central to Malawi's economy, contributing nearly 40 percent of GDP and roughly 90 percent of the country's export earnings
(USAID, 2017a). Maize is a dominant crop in Malawi, accounting for 28.8 percent of agricultural GDP. Groundnut is an important smallholder food and cash crop in
Malawi contributing 1.6 percent to agricultural GDP. Soya and sunflower account for 13 percent of that sector's total intermediate input expenditure, and account
for 1.9 percent of agricultural GDP (Aragie et al.~2018). The agriculture sector is the driver of Malawi's economy and provides employment to 85\% of the
workforce, and contributes 85 to 90\% of foreign exchange earnings and 60 to 70\% of raw materials for the manufacturing sector (UNFCCC, 2006). Over half (51\%) of
Malawi's predominantly rural (86\%) population live below the national poverty line, most (85\%) dependent on agriculture for livelihood, and on only \$320 United
States Dollars (USD) per capita per year (Chinsinga, Chasukwa and Naess 2012; World Bank 2014; Zulu 2017). Average annual headline inflation in 2016 stood at
22.6\%, slightly lower than the 2015 figure of 21.0\%, with rising food inflation as the main driver. Power generation reduced by 30\% due to low levels in the
Shire River affecting economic activities in sectors such as manufacturing, which experienced low capacity utilization. Malawi's overall GDP grew at only 2.7\% in
2016, down from 2.9\% in 2015. According to the poverty statistics for 2010, 70.9\% of the people in Malawi are living on less than \$1.90 a day. The people living
below the national poverty line are 50.7\% and the country inequality trend (GINI Index) stands at 46.1 (Irish Aid 2018). According to the United Nations
Development Program's Human Development Report for 2014, about 62\% of the population in Malawi lives on less than US \$1.25 a day and 89\% lives below the US \$2 a
day threshold (USAID, 2015). Tobacco is Malawi's largest export cash crop, accounting for over half of export earnings, followed by tea and sugar (Purchase from
Africans for Africa. n.d.; FAO, n.d.; and World Bank, 2012). Fishing contributes about 4\% to Malawi's Domestic Product (GDP) and accounts for 60--70 percent of
Malawians' animal protein intake. An estimated 1.6 million Malawians derive at least some income from fishing, fish processing, marketing and trading, boat and
gear-making, and allied industries (Brummet and Noble, 1995; Andrew et al.~2003). Wildlife is a valuable tourism resource as it can contribute significantly to
incomes and employment. The sector, however, faces a number of challenges including poaching, poor supporting infrastructure, and low community participation in
wildlife conservation (USAID, 2013).

Malawi is one of the poorest countries in the world, ranked 170 of 188 countries on the global United Nations Development Programme's HDI. More than 70\% of the
population lives below the international poverty line of USD 1.90 per capita per day and GDP per capita is just USD 372 (2015). Both inequality and poverty rates
are high. About 20.7\% of the people are so poor that they cannot afford to eat a minimum daily recommended food intake, and at least 37\% of children under five
are chronically undernourished and stunted (low weight for age). Malawi's wealth per capita, USD 8,409 in 2014, is much lower than the average for other low-
income countries (USD 13,629) or for Sub-Saharan Africa as a whole (USD 25,562) (Hughes et al.~2019). Real gross domestic product (GDP) grew by 5.7\% in 2014, but
slowed down to 2.5\% in 2016 after floods in early 2015 followed by two consecutive years of drought, which has adversely affected the performance of agriculture,
which accounts for about a third of the country's GDP. The country has a GDP of US\$ 6.4 billion (2015 data), and per capita income (2015 data) is US\$ 34011.
Malawi is a low-income country with 74\% of Malawians earning US\$ 1.25 per day or less. Using national poverty headcount, approximately 50.7\% of the population
live below the national poverty line. About 24.5\% are considered ultra-poor, meaning that they cannot afford to meet the minimum standard of the daily
recommended food requirement. Levels of chronic malnutrition are very high at 42\%, wasting is at 4\% and underweight prevalence is at 13\%. The 2015 flood damage
cost event estimated at US\$ 335 million, equivalent to approximately 5\% of GDP. Land degradation is estimated to cost the equivalent of 5.3\% of GDP each year
with soil degradation a significant factor that contributes between 4 and 25\% to the loss of agricultural yields in Malawi (Masi, 2017).

Lake Malawi provides the main source of the country's fish production. Other important sources include Lake Chilwa, Lake Malombe and the Elephant Marsh. The
sector has experienced considerable decline of commercially important fish species like Chambo (Oreochromis spp.) from around 30,000 Mt a year in the late 20th
century to about 2,000 Mt annually in recent years from Lake Malawi caused by overfishing and climatic influences which result in reduced water levels and
disrupt fish breeding and nursery sites. Weak governance capacity to enforce fisheries regulations, and control of illegal fishing and destruction of habitats,
contribute to reduced abundance of fish stocks and fisheries resources in Malawi (Masi, 2017). The National Human Development Report of 2001 ranks Malawi as one
of the lowest in terms of Human Development Index (HDI), placing it at number 163 out of 173 countries in the world (United Nations Development Programme
(UNDP)/Malawi Government (MG), 2001). It is one of the poorest countries in Africa, with about 65\% of its population living below the poverty line in 1998, and
29\% living in extreme poverty (MG, 1995, 2000; MoA, 2005; UNFCCC 2006). The manufacturing sector currently makes a small contribution to national income (12\% of
GDP) and employment and there is limited industrial diversification. In addition, there are weak inter and intra-industry linkages (Malawi Vision 2020).

\textbf{Technological context:} It is the aspirations of Malawians through their Malawi's Vision 2020 to have a science and technology-driven economy. A developing
country such as Malawi needs information technology to achieve development in all spheres of human endeavor. However, promoting the use of IT is the main
challenge (Malawi Vision 2020). Greater use of geospatial technologies such as aerial surveys, satellite monitoring, and drone surveys could help address the
limited human resources at field levels. Mobile phone technologies are rapidly improving communication and services with growing opportunities for informing
Malawians on environmental issues (Hughes et al.~2019). The media plays a key role in raising public awareness on climate change issues especially in informing
rural communities who suffer most due to adverse impacts of climate change due to their low adaptive capacity (Ministry of Environment and Climate Change
Management Environmental Affairs Department, 2013-2018). The systematic use of new cell phone technologies, social media, video documentaries, radio and TV
programs, and other information-communication technologies can greatly accelerate the widespread knowledge of proven restoration interventions (Ministry of
Natural Resources, Energy and Mining, n.d.).

\hypertarget{legal-frameworks}{%
\subsection{Legal frameworks}\label{legal-frameworks}}

The vision of Malawi's NAP aligns with the Malawi Growth and Development Strategy III (2017-2022). MGDS III is anchored in Water Development, Agriculture and
Climate Change. The NAP process will address climate change management through improved community resilience to climate change through enhanced agricultural
production, infrastructure development and disaster risk management. The MGDS III adaptation strategies for Agriculture, Water Development and Climate Change
Management include increased agricultural production and productivity, increased land under irrigation; increased agricultural diversification, enhanced
agricultural risk management, enhanced integrated water resources management at all levels, and improved weather and climate monitoring for early warning,
preparedness and timely response. These will be the strategies the NAP will also prioritize. The MGDS III goals are premised on Malawi's long term development
aspirations well laid out in Vision 2020.Malawi has also prioritized climate change, environment and natural resources management among the priorities within
priorities of the Malawi Growth and Development Strategy (MGDS II). Government of Malawi has also developed the National Climate Change Management Policy (NCCMP)
whose goal is to promote climate change adaptation, mitigation, technology transfer and capacity building for sustainable livelihoods through Green Economy
measures for Malawi. The policy outlines six priority areas for climate change management in the country which include: Climate change adaptation, Climate change
mitigation, Capacity building, education, training and awareness, Research, technology development and transfer, and systematic observation, Climate change
financing, Cross-cutting issues like gender consideration, population dynamics and HIV and AIDS.

The NCCMP policy statements are:

\begin{enumerate}
\def\labelenumi{\arabic{enumi}.}
\item
  Reduce vulnerabilities of populations in Malawi and promote community and ecosystem resilience to the impacts of climate change;
\item
  Ensure that women, girls and other vulnerable groups are engaged and involved in planning and implementing climate change adaptation interventions; and
\item
  Ensure that communities are able to adapt to climate change by promoting climate change adaptive development in the long term.
\item
  Promote the reduction of greenhouse gas emissions; and
\item
  Enhance carbon sinks through re-afforestation and sustainable utilization of forest resources.
\item
  Build capacity in all sectors and at all levels in climate change to attain socio-economic development utilizing the principles of green economy; and
\item
  Address capacity gaps on investment in skills and capabilities for negotiations, mechanisms for reducing emissions while supporting prudent environmental
  management and sustainable economic growth.
\item
  Enhance research, technology and systematic observation for climate change management, supported by appropriate capacity development and dedicated financing
\item
  Encourage resource mobilization and commitment of government for the prioritized technologies.
\item
  Enhanced financing for implementation and coordination of climate change management activities through increased national budgetary allocation, establishment
  of a Climate Change Management Fund, improved access to international climate financing (both multilateral and bilateral) and private sector investments.
\item
  Mainstream gender and issues affecting the disadvantaged groups into all climate change strategies, plans and programmes.
\item
  Integrate population issues into climate change management in the development agenda through an integrated approach which would reduce poverty, protect
  natural resources and reduce inequality.
\item
  Incorporate HIV and AIDS as well as gender considerations in all climate change interventions including adaptation, mitigation, capacity building and
  technology development and transfer.
\end{enumerate}

The Government of Malawi through their Vision 2020 and the Malawi Constitution 1995 has put in place a series of legislative sectoral frameworks and strategies
to integrate environment and climate change management in socio-economic development activities. These include:

\begin{itemize}
\item
  \begin{enumerate}
  \def\labelenumi{\roman{enumi}.}
  \tightlist
  \item
    The Malawi Growth Development Strategies;
  \end{enumerate}
\item
  \begin{enumerate}
  \def\labelenumi{\roman{enumi}.}
  \setcounter{enumi}{1}
  \tightlist
  \item
    United Nations Development Assistance Framework for Malawi (UNDAF);
  \end{enumerate}
\item
  \begin{enumerate}
  \def\labelenumi{\roman{enumi}.}
  \setcounter{enumi}{2}
  \tightlist
  \item
    National Strategy for Sustainable Development 2004;
  \end{enumerate}
\item
  \begin{enumerate}
  \def\labelenumi{\roman{enumi}.}
  \setcounter{enumi}{3}
  \tightlist
  \item
    National Environmental Policy (NEP) 2004;
  \end{enumerate}
\item
  \begin{enumerate}
  \def\labelenumi{\alph{enumi}.}
  \setcounter{enumi}{21}
  \tightlist
  \item
    National Forestry Policy of Malawi, 1996;
  \end{enumerate}
\item
  \begin{enumerate}
  \def\labelenumi{\roman{enumi}.}
  \setcounter{enumi}{5}
  \tightlist
  \item
    National Land Resource Management Policy and Strategies (2000);
  \end{enumerate}
\item
  \begin{enumerate}
  \def\labelenumi{\roman{enumi}.}
  \setcounter{enumi}{6}
  \tightlist
  \item
    Wildlife Policy (2000);
  \end{enumerate}
\item
  \begin{enumerate}
  \def\labelenumi{\roman{enumi}.}
  \setcounter{enumi}{7}
  \tightlist
  \item
    Malawi Irrigation Policy and Development Strategy (2000);
  \end{enumerate}
\item
  \begin{enumerate}
  \def\labelenumi{\roman{enumi}.}
  \setcounter{enumi}{8}
  \tightlist
  \item
    National Fisheries and Aquaculture Policy (2001);
  \end{enumerate}
\item
  \begin{enumerate}
  \def\labelenumi{\alph{enumi}.}
  \setcounter{enumi}{23}
  \tightlist
  \item
    National Land Policy (2002);
  \end{enumerate}
\item
  \begin{enumerate}
  \def\labelenumi{\roman{enumi}.}
  \setcounter{enumi}{10}
  \tightlist
  \item
    National Environmental Action Plan 2002;
  \end{enumerate}
\item
  \begin{enumerate}
  \def\labelenumi{\roman{enumi}.}
  \setcounter{enumi}{11}
  \tightlist
  \item
    National Climate Change Investment Plan (2013);
  \end{enumerate}
\item
  \begin{enumerate}
  \def\labelenumi{\roman{enumi}.}
  \setcounter{enumi}{12}
  \tightlist
  \item
    National HIV and AIDS Policy, 2003;
  \end{enumerate}
\item
  \begin{enumerate}
  \def\labelenumi{\roman{enumi}.}
  \setcounter{enumi}{13}
  \tightlist
  \item
    Malawi Energy Policy (2003);
  \end{enumerate}
\item
  \begin{enumerate}
  \def\labelenumi{\roman{enumi}.}
  \setcounter{enumi}{14}
  \tightlist
  \item
    National Land Use Planning and Management Policy, 2005;
  \end{enumerate}
\item
  \begin{enumerate}
  \def\labelenumi{\roman{enumi}.}
  \setcounter{enumi}{15}
  \tightlist
  \item
    Food Security Policy (2006);
  \end{enumerate}
\item
  \begin{enumerate}
  \def\labelenumi{\roman{enumi}.}
  \setcounter{enumi}{16}
  \tightlist
  \item
    National Water Policy (2005);
  \end{enumerate}
\item
  \begin{enumerate}
  \def\labelenumi{\roman{enumi}.}
  \setcounter{enumi}{17}
  \tightlist
  \item
    Mines and Minerals Policy (2013);
  \end{enumerate}
\item
  \begin{enumerate}
  \def\labelenumi{\roman{enumi}.}
  \setcounter{enumi}{18}
  \tightlist
  \item
    National Transport Policy (2015);
  \end{enumerate}
\item
  \begin{enumerate}
  \def\labelenumi{\roman{enumi}.}
  \setcounter{enumi}{19}
  \tightlist
  \item
    National Construction Industry Policy (2015);
  \end{enumerate}
\item
  \begin{enumerate}
  \def\labelenumi{\roman{enumi}.}
  \setcounter{enumi}{20}
  \tightlist
  \item
    Water Resources Act (2013);
  \end{enumerate}
\item
  \begin{enumerate}
  \def\labelenumi{\roman{enumi}.}
  \setcounter{enumi}{21}
  \tightlist
  \item
    Mines and Minerals Act (1981);
  \end{enumerate}
\item
  \begin{enumerate}
  \def\labelenumi{\roman{enumi}.}
  \setcounter{enumi}{22}
  \tightlist
  \item
    Disaster Preparedness and Relief Act (DPRA) (1991);
  \end{enumerate}
\item
  \begin{enumerate}
  \def\labelenumi{\roman{enumi}.}
  \setcounter{enumi}{23}
  \tightlist
  \item
    Waterworks Act (1995);
  \end{enumerate}
\item
  \begin{enumerate}
  \def\labelenumi{\roman{enumi}.}
  \setcounter{enumi}{24}
  \tightlist
  \item
    Environment Management Act (1996);
  \end{enumerate}
\item
  \begin{enumerate}
  \def\labelenumi{\roman{enumi}.}
  \setcounter{enumi}{25}
  \tightlist
  \item
    Forestry Act (1997);
  \end{enumerate}
\item
  \begin{enumerate}
  \def\labelenumi{\roman{enumi}.}
  \setcounter{enumi}{26}
  \tightlist
  \item
    Fisheries Conservation and Management Act (1997);
  \end{enumerate}
\item
  \begin{enumerate}
  \def\labelenumi{\roman{enumi}.}
  \setcounter{enumi}{27}
  \tightlist
  \item
    Road Traffic Act (1997);
  \end{enumerate}
\item
  \begin{enumerate}
  \def\labelenumi{\roman{enumi}.}
  \setcounter{enumi}{28}
  \tightlist
  \item
    Local Government Act (1998);
  \end{enumerate}
\item
  \begin{enumerate}
  \def\labelenumi{\roman{enumi}.}
  \setcounter{enumi}{29}
  \tightlist
  \item
    Energy Regulation Act (2004);
  \end{enumerate}
\item
  \begin{enumerate}
  \def\labelenumi{\roman{enumi}.}
  \setcounter{enumi}{30}
  \tightlist
  \item
    National Parks and Wildlife Act (2004),
  \end{enumerate}
\item
  \begin{enumerate}
  \def\labelenumi{\roman{enumi}.}
  \setcounter{enumi}{31}
  \tightlist
  \item
    Gender Equality Act (2013).
  \end{enumerate}
\end{itemize}

\textbf{Table 1: National/sectoral policies, strategies and plans relevant for adaptation}

\begin{longtable}[]{@{}
  >{\raggedright\arraybackslash}p{(\columnwidth - 6\tabcolsep) * \real{0.01}}
  >{\raggedright\arraybackslash}p{(\columnwidth - 6\tabcolsep) * \real{0.06}}
  >{\raggedright\arraybackslash}p{(\columnwidth - 6\tabcolsep) * \real{0.01}}
  >{\raggedright\arraybackslash}p{(\columnwidth - 6\tabcolsep) * \real{0.92}}@{}}
\toprule
& \textbf{Title/Type} & \textbf{Year} & \textbf{Objective} \\
\midrule
\endhead
\textbf{Climate} & Malawi National Climate Change Policy & 2016 & The policy aims to effectively manage the impacts of climate changethrough interventions that build and sustain the social and ecologicalresilience of all Malawians; contribute towards the stabilization ofgreenhouse gas concentrations in the atmosphere at a level that wouldprevent dangerous human-induced interference with the climate systemwithin a timeframe that enables social, economic and environmentaldevelopment to proceed in a sustainable manner; Integrate climatechange into planning, development, coordination and monitoring of keyrelevant sectors in a gender sensitive manner; and Integrate cross-cuttingissues into climate change management through an appropriate institutional framework. \\
& Second National Communication & 2011 & The objectives include; strengthening the technical and institutional capacities of various public and private sector organizations to acquireskills and competencies in mainstreaming climate change issues into theirrespective sectoral programmes, policies and strategies, contributing toglobal efforts in better understanding the various sources and sinks ofgreenhouse gases, potential impacts of climate change and effective responsemeasures to achieve the ultimate goal of UNFCCC of stabilizing greenhouse gasconcentrations in the atmosphere to a level that would prevent dangerous anthropogenic interference with the climate system, proposing climate changeprojects aimed at finding solutions to climate change problems thatcommunities can adapt and/or use to mitigate climate change, enhancing generalawareness on climate change and climate change related issues, strengtheningdialogue, information exchange, networking and cooperation among variousstakeholders in the public and private sector organizations, including NGOs,and the university, involved in climate change studies in accordance with Article 6 of the UNFCCC. \\
& NAPA & 2006 & The NAPA seeks to increase the adaptive capacities of vulnerable communities toadverse effects of climate change. Five urgent activities were rated high andcombined into project clusters. These include: Improving community resilience toclimate change through the development of sustainable rural livelihoods; restoringforest in Upper, Middle and Lower Shire Valleys catchments to reduce siltation andthe associated water flow problems; improving agricultural production under erraticrains and changing climatic conditions; improving Malawi's preparedness to copewith droughts and floods; and improving climate monitoring to enhance Malawi'searly warning capability and decision making and sustainable utilization of LakeMalawi and lakeshore areas resources. \\
& National ClimateChange InvestmentPlan & 2013-2018 & The primary objective is to increase climate change investments in Malawi. \\
& National Environment and Climate ChangeManagement & 2012-2016 & To inform, educate and communicate the public and ensure popular participation in themanagement of environment, natural resources and climate change. This will be achievedthrough the following specific objectives: increase public awareness, knowledge,understanding and participation on environment and climate change among various targetgroups, specifically rural communities and disadvantaged groups including women andyouth, promote popular participation in the implementation of the environment andclimate change ENRM and CC, enhance institutional and individual capacity forcommunication in environment and climate change, foster collaboration, coordinationand networking of NECC communication interventions, enhance monitoring and evaluationof NECC Strategy \\
\textbf{Environment} & National Forestry Policy of Malawi & 1996 & The objective of the policy is to improve the quality of life of the Malawi population,particularly rural smallholders, and provide a stable local economy in order to reducethe degenerative impact of development on the environment that often accompaniespoverty. The forest policy provides an enabling environment for making forests and treeresources available to communities on a sustainable basis thereby promoting ruraldevelopment. \\
& National EnvironmentalPolicy (NEP) & 2004 & To minimize the adverse impact of climate change and variability to reduce air pollutionand greenhouse. However, the guiding principles and the strategies for achieving thisobjective suggests that the policy orientation is focused on mitigation and notadaptation. \\
& National EnvironmentalAction Plan & 2002 & To document and analyse all environmental issues and measures to alleviate them, topromote sustainable use of natural resources in Malawi, to develop an environmentalprotection and management plan \\
& National Biodiversityand Action Plan & & Outlines strategies for species monitoring and recovery, conservation of traditionalagro-biodiversity resources, conservation of aquatic and mountain biodiversity thatprovides local communities with significant livelihood options for food security,medicine and other uses. \\
\textbf{Agriculture} & Malawi Irrigation Policyand Development Strategy & 2000 & Increase land under sustainable irrigation farming, Facilitate crop diversificationand intensification, Create an enabling environment for irrigated agriculture, Optimize investment in irrigation development taking into account climate change,Enhance capacity for irrigated agriculture, Promote a business culture in thesmall-scale irrigated agriculture sector \\
& Food Security Policy & 2006 & Increasing agricultural productivity as well as diversity and sustainableagricultural growth and development, Guarantee that all Malawians have at all timesphysical and economic access to sufficient nutritious food required to lead ahealthy and active life \\
& Draft National Agricultural Policy & Draft & The draft policy seeks to promote adaptation and mitigation technologies andinterventions to minimize future adverse effects of climate change on agriculturalproduction and rural livelihoods. Some of the proposed ac2ons will support climatechange adaptation and mitigation in agriculture. \\
\textbf{Energy} & Malawi Energy Policy & 2013 & To improve the security and reliability of energy supply; Increase access toaffordable and modern technologies; Stimulate economic development and ruraltransformation for poverty reduction; Improve the energy sector and governance; andMitigate environmental, safety and health impacts of energy production and utilization. \\
\textbf{Health} & National HIV andAIDS Policy & 2003 & To improve the provision and delivery of prevention, treatment, care and supportservices for PLWAs, to reduce individual and societal vulnerability to HIV/AIDS bycreating an enabling environment, to strengthen the multi-sectoral and multi-disciplinary institutional framework for coordination and implementation ofHIV/AIDS programs in the country \\
\textbf{Economic} & Wildlife Policy & 2000 & The goal of this policy is to ensure proper conservation and management of wildliferesources. The policy also increases sustainable utilization and equitable access tothe resources and fair sharing of the benefits from the resources for both present andfuture generations of Malawi. \\
& United Nations DevelopmentAssistance Framework forMalawi (UNDAF); & 2019-2023 & It incorporates the goals and principles that underpin Agenda 2030 and the 17 SDGs thatlie at its heart. It further guides the UN Agency programs ensuring UN wide coherence and represents a strong collaborative link with the Government of Malawi's developmentaimsexpressed in Malawi Growth and Development Strategy MGDS III \\
& The Malawi GrowthDevelopment Strategy III & 2017-2022 & Improved weather and climate monitoring for early warning, preparedness and timelyresponse. The strategy will promote effective and efficient generation, analysis andutilization of reliable, responsive, high quality, up to date and timely climate services; and Improving spatial (by area and agro-ecological zone) weather and climate monitoringand prediction systems through automation and other IT advances. \\
& National Strategy forSustainable Development & 2004 & Seeks to reduce damage to property and loss of life caused by weather and climate naturaldisasters and contributes to sustainable industrial production or meets the UNFCCCobligations \\
& Mines and Minerals Policy & 2013 & To attract investment in the mining sector, to formalize and improve small scale mining, to create employment opportunities and economic diversification, to incorporate socialdimensions and empower women in mining, to promote measures to protect the environment,as well as increase foreign exchange earnings. \\
\textbf{Land} & National Land ResourceManagement Policy andStrategies & 2000 & The policy seeks to: improve and sustain the productivity of land for agricultural andother uses through use of sound technologies to conserve soil and water resources, soilfertility improvements and respecting livestock stocking capacities of land; Promoterehabilitation of degraded lands for both agriculture and other uses with the aim ofsustaining the usability of these lands; and Control the dangers of surface run-off water such as soil erosion and all its associated causative factors. \\
& National Land Policy & 2002 & The goal of the policy is to ensure tenure security and equitable access to land by allcitizens of Malawi in order to facilitate ecologically balanced use of land resources.The policy deals with issues of access to land, tenure security and sustainableenvironmental management. The key focus of the policy is on issues of land ownership,land use, land registration, national physical development plans, and establishing legalframework for land use. \\
& National Land UsePlanning and ManagementPolicy & 2005 & To secure social and economic development through optimum and ecologically balanced useof land and land based resources \\
\textbf{Water} & National Water Policy & 2005 & The objectives of the policy are to: Promote sustainable and integrated water resourcemanagement and development to make water readily available and equitably accessible byall Malawians; Ensure water of acceptable quality for all needs; Provide water supplyand sanitation services to all at affordable cost; Promote efficient and effectiveutilization, conservation and protection of water resources for sustainable agricultureand irrigation, fisheries, navigation, eco-tourism, forestry, hydropower and disastermanagement and environmental protection. \\
\textbf{Education} & Malawi Strategy onClimate Change Learning & 2013 & The objective of the Strategy is to strengthen human resources and skills developmentfor the advancement of green, low emission and climate resilient development \\
\textbf{Socio/Cultural} & Gender Equality Act & 2013 & The act seeks to promote gender equality, equal integration, and influence, andempowerment, dignity in all functions of society, to prohibit and provide redress forsex discrimination, harmful practices and sexual harassment, to provide publicawareness on promotion of gender equality. \\
& National Gender Policy & 2000 & To strengthen gender mainstreaming and women empowerment at all levels in order tofacilitate attainment of gender equality and equity in Malawi, to reduce genderinequalities and enhance participation of women, men, girls and boys in socio-economicdevelopment processes. \\
\textbf{Fisheries} & National Fisheries andAquaculture Policy & 2001 & The policy provides clear guidelines for the development of the fisheries sector. Thepolicy also stipulates roles and responsibilities of public and private sector and civilsociety organisations in the development of the fisheries industry \\
& Fisheries Conservationand Management Act & 1997 & The act provides for the regulation, conservation and management of the fisheries ofMalawi. \\
\bottomrule
\end{longtable}

\textbf{Table 2: General environmental laws and their policy relevance for adaptation}

\begin{longtable}[]{@{}
  >{\raggedright\arraybackslash}p{(\columnwidth - 4\tabcolsep) * \real{0.07}}
  >{\raggedright\arraybackslash}p{(\columnwidth - 4\tabcolsep) * \real{0.01}}
  >{\raggedright\arraybackslash}p{(\columnwidth - 4\tabcolsep) * \real{0.92}}@{}}
\toprule
\textbf{Title/Type} & \textbf{Year} & \textbf{Objective} \\
\midrule
\endhead
\textbf{EnvironmentManagement Act} & 2017 & This Act concerns the conservation and management of the environment in Malawiand prescribes environmental standards. It also concerns the conservation andmanagement of biological (genetic) resources. The Act consists of 119 sectionsdivided into 17 Parts, covering main areas of environmental concern, some ofthe areas covered are: The Environment Protection Authority; EnvironmentalPlanning; Environmental and social impact assessment, audits and monitoring;Environmental standards; Management of the Environment and Natural Resources;Pollution Control; The Environment Fund. Climate Change is addressed in partVIII concerning management of the environment and natural resources. \\
\textbf{Energy RegulationAct} & 2004 & This Act establishes the Energy Regulatory Authority to regulate the energysector, defines its functions and powers, and provide for licensing of energyundertakings. Members of the Authority should have adequate knowledge relatedrenewable energy. The Authority shall notably promote energy efficiency and theexploitation of renewable resources. The Authority is charged in art. 9.2.(i) topromote the exploitation of renewable energy resources, and (e) to promote energyefficiency and energy savings. \\
\textbf{Rural ElectrificationAct} & 2004 & This act makes provision for the promotion, funding, management and regulation ofrural electrification. Specific, more favourable rules are laid out for renewableenergy resources, including in terms of finance. \\
\textbf{Disaster Preparednessand Relief Act(Cap. 33:05)} & 1992 & This Act makes provision for the prevention of disasters in Malawi and for disasterpreparedness and disaster mitigation. There shall be a Commissioner for DisasterPreparedness and Relief who shall, among other things: supervise the establishmentof civil protection organizations and civil protection areas and control and directpersonnel, materials and services for the purposes of this Act. The Act providesfor the establishment of the National Disaster Preparedness and Relief Committeeof Malawi. \\
\textbf{Wildlife Policy} & 2000 & The policy embraces the following objectives: Ensure adequate protection ofrepresentative ecosystems and their biological diversity by promoting and adoptingappropriate land management practices that are in line with sustainable utilizationconsiderations; Create public awareness and understanding on the need for wildlifeconservation and management and also their relationship to other land use issues;Create a conducive environment for wildlife-based enterprises; Facilitate developmentof necessary legislation and enforcement mechanisms in order to eliminate illegalwildlife use; and Develop a cost effective legal, administrative and institutionalframework for managing wildlife resources without compromising the resources'ecological attributes. \\
\bottomrule
\end{longtable}

\hypertarget{institutional-arrangements-for-climate-change-adaptation}{%
\subsection{Institutional arrangements for climate change adaptation}\label{institutional-arrangements-for-climate-change-adaptation}}

Malawi has several existing institutional structures to support climate change mitigation and adaptation policies (Malawi \& Environmental Affairs Department,
2016). The Malawi Constitution explicitly calls for environmental support, and the Malawi government has addressed climate change at the national, ministerial
and departmental level (Amadu et al., 2020). Coordination between government agencies is a significant challenge for implementing the climate change policy
components since climate change is a cross-cutting issue affecting most sectors, such as agriculture, human health, energy, fisheries, wildlife, water, forestry
and gender (Hughes et al., 2019). Table 3 presents the evolution of the climate change agenda.

Table 3. Evolution of the national climate change agenda in Malawi: policies, programmes, institutions and linkage to political leadership {[}source{]}

insert table 3 \ldots{}

\hypertarget{vision-goals-and-objectives-of-the-nap}{%
\chapter{Vision, Goals and Objectives of the NAP}\label{vision-goals-and-objectives-of-the-nap}}

\hypertarget{vision-for-adaptation-for-the-country}{%
\section{Vision for Adaptation for the Country}\label{vision-for-adaptation-for-the-country}}

The vision is ``a country that is resilient to adverse socio-economic impacts of climatic change''.

\hypertarget{goals-and-objectives-of-the-nap}{%
\section{Goals and Objectives of the NAP}\label{goals-and-objectives-of-the-nap}}

The main goal and objectives of the Malawi NAP, in line with the country's National Climate Change Management Policy, are:

\textbf{\emph{Goal}}
- create an enabling policy and legal framework for a pragmatic, coordinated and harmonized approach to climate change management

\textbf{\emph{Objectives}}

\begin{itemize}
\item
  Effectively manage the impacts of climate change through interventions that build and sustain the social and ecological resilience of all Malawians;
\item
  Integrate climate change into planning, development, coordination and monitoring of key relevant sectors in a gender sensitive manner; and
\item
  Integrate cross-cutting issues into climate change management through an appropriate institutional framework
\end{itemize}

\hypertarget{climate-change-adaptation-assessment}{%
\chapter{Climate Change Adaptation Assessment}\label{climate-change-adaptation-assessment}}

\hypertarget{observed-climate-impacts}{%
\section{Observed climate impacts}\label{observed-climate-impacts}}

The general aspects of the climate and environment of Malawi have been covered in sections 1.1 and 5.1.1 above. In this section, emphasis is on the impacts of
observed climate on the identified systems, for which the impacts of climate extreme events are summarised in Table XX below.

Table XX. Examples of notable past extreme climate events, impacts and impacted systems in Malawi

\begin{longtable}[]{@{}
  >{\raggedright\arraybackslash}p{(\columnwidth - 6\tabcolsep) * \real{0.02}}
  >{\raggedright\arraybackslash}p{(\columnwidth - 6\tabcolsep) * \real{0.04}}
  >{\raggedright\arraybackslash}p{(\columnwidth - 6\tabcolsep) * \real{0.74}}
  >{\raggedright\arraybackslash}p{(\columnwidth - 6\tabcolsep) * \real{0.20}}@{}}
\toprule
\textbf{Hazard} & \textbf{Event Data} & \textbf{Impacts} & \textbf{Impacted ecosystems} \\
\midrule
\endhead
Drought & 1960-2014 & - Increasing drought frequency and intensity- Seven severe droughts experienced since 1991- The droughts of 1991/92 and 1993/94 impacted very severely on agriculture - maize production declined by 60\% in 1991/92 -equivalent of only 45\% of average production levels for the previous five years - Annual loss of maize production by 4.6 \% - Malawi food crisis marked in 2002 - Depleted food reserves - Deforestation - Water scarcity and elevated poverty levels - Increased vegetation loss - Increased irrigation by rural small holder farmers, 90 \% of thefarmers in Malawi - Disruption of hydroelectric power generation at Lake Malawi 1991-1994 & - Crop production - Livestock production - Rural water supply - Health - Agriculture market and trade - Forestry - Urban food planning - Ecosystems \\
Floods & 1967-2014 & - A total of 19 floods recorded, causing decreased food suppled - Shire River flooding and property destruction -- it registered almostthree-quarters of the total flood-related economic losses in the past ten years - Shortage in agricultural produce and food crisis - Damage to infrastructure -- resulting is \$ 60 million loss - Decreased annual maize production by 12 \% - Average loss of 0.7 \% of the annual GDP - Agricultural losses are estimated at 3.5 to 8.2 \% of GDP during RP5and RP50 floods - Mzuzu city floods -- Fifteen settlements were affected, 19,000 peopledisplaced, seven people killed, and seven camps were set up for thedisplaced - Chronic and acute respiratory diseases reported - 4,901,344 confirmed cases of malaria and 3,614 deaths were reported by the World Health Organization in 2017 - Increased poverty rate by 0.9\% per annum & - Crop production - Health - Energy - Urban water supply (Mzuzu City) - Agriculture market and trade - Transport - River bank flood planning - Sewerage waste management system (Mzuzu city) \\
Heavy storms & La Niña2000/01 & - Unusually high rainfall events - Very low crop yields -- causing waterlogging. - Increased leaching of soil nutrients & - Crop production - Agriculture market and trade - Ecosystem \\
Land degradation & 2001-2014 & - The average annual national soil loss rates in 2014 were estimated at29 tons per hectare. - The annual land degradation costs between 2001 and 2009 is estimated atUSD 244 million per year --equivalent to 6.8\% of Malawi's GDP & - Forestry - Ecosystem - Crop production \\
Intense heat & & - Decreasing lake levels - Decreased fish catches - Dry river and reduced surface water flow & - Ecosystems - Health -- link to fish nutritionvalue - Agriculture market and trade - Energy \\
Increased temperature & 1960-2006 & - Increased temperatures of 0.9°C between 1960 and 2006, with an average rate of 0.21°C per decade - Reduced hydropower generation at the Lujeri Micro-Hydropower Scheme insouthern Malawi during 1980--2011 & \\
Low rainfall & Mid 1980s-- 2014El Niño 1997/1998 & - Decreased precipitation in Chilwa basin - Reduced surface water flows & - Crop production - Livestock production \\
\bottomrule
\end{longtable}

Figure 2. Common shocks experienced from flooding in 2016-2017 Mzuzu town according to the IHS3

The average temperature in Malawi ranges from 8°C in the northern highlands to 38 °C in the lowland regions around Lake Malawi and the Lower Shire Valley (Nhamo
et al., 2019). Since the 1960s, Malawi has recorded an annual mean temperature rise of 0.9°C (Parrish et al., 2020). Analysed data from 1960 to 2007 showed
increasing drought frequency and intensity and the variability of rainfall, contributing to regional (SADC) insecurity of food and water (Godwell Nhamo \&
Muchuru, 2019). Malawi suffered seven severe droughts and 19 floods between 1967 and 2014 that adversely affected smallholders' production and food security
(Haug \& Wold, 2017). As a result, trends in people in need of food assistance (Figure 1) have increased between 2012 and 2016 (Haug \& Wold, 2017).

\emph{Figure XX: Trends in people in need of food assistance}

Besides the severe droughts, frequent flood events remain a dominant impact of climate extremes in Malawi (Adeloye et al., 2015; IFPRI, 2020). Although local
rainfall patterns are challenging to model accurately, an increase in the frequency and magnitude of drought and floods has been observed (Parrish et al., 2020).
Malawi received the highest record rainfall for the country in 2015, causing severe flooding, especially in the Southern Region (Haug \& Wold, 2017). During
floods, agricultural losses are estimated at 3.5 to 8.2 \% of GDP during RP5 and RP50 floods, respectively (IFPRI, 2020). According to Adeloye et al.~(2015),
rainfall in Shire of Malawi is about 2000 mm due to orographic influences. As a result, flooding is more prominent in the Shire valley. According to Malota and
Mchenga (2019), the rural lower Shire Valley is most prone to flooding in the entire country - it registered almost three-quarters of the total flood-related
economic losses in the past ten years. The Shire Basin is characterized by medium risk communities based on vulnerability and hazard - accounting for 8 of the 12
communities assessed (Adeloye, 2015). In the last decade alone, Malawi recorded an annual average loss of 12 \% in maize yield production due to flood-related
crop damage (Malota \& Mchenga, 2019). The Mzuzu city experienced the worst floods ever recorded since its establishment in April 2016. Fifteen settlements were
affected, 19,000 people were displaced, seven people were killed, and seven camps were set up to house the displaced (Kita, 2017). Figure 2 shows the percentage
of households that suffered various flood-related shocks between 2016 and 2017 (Kita, 2017).

In Malawi, agriculture is the foundation of the economy (Ajefu \& Abiona, 2020). It employs 85\% of the workforce and generates one-third of the gross domestic
product (GDP) and 90\% foreign exchange earnings (Msowoya et al., 2016). Under the dominant land-use practice of ridge tillage, maize-based farming systems
cultivated up to 80 \% of the land area of the Lilongwe-Kasungu plains (Lark et al., 2020). Maize farming covers over 92\% of Malawi's agricultural land and
accounts for over 54\% of the national caloric intake (Msowoya et al., 2016). In the last decade alone, Malawi recorded an annual average loss of 12 \% in maize
yield production due to food-related crop damage (Malota \& Mchenga, 2019). Malawi recorded a deficient national average maize production of 0.76 tons per hectare
(t / ha) in 2005, 40 \% below the expected average (Msowoya et al., 2016). There was also overall maize (the staple crop in the region) deficit of 5.1 million t,
a 10\% decrease in production compared to the previous year and a 15\% drop compared to the 5-year average (Nhamo \& Muchuru, 2019). The average national production
growth rate in the 1980s was 3\%, followed by a production decline rate of 2 \% per annum from 1990 to 1994 (Msowoya et al., 2016). Figure 3a shows the
agricultural contribution of Malawi to the total GDP, indicating a good correlation between the economic growth of Malawi and agricultural production. In
contrast, the changes in maize production in Malawi between 1980 and 2011 are shown in Figure 3b, illustrating the instability of maize production and the steady
rise in maize prices over the past decades. Because of climate change, the Lilongwe District, Malawi's largest maize growing district, may decrease by up to 14 \%
by mid-century, rising to as much as 33 \% by the end of the century (Msowoya et al., 2016). Small- and medium-scale farmers bear the brunt of floods, with
recorded average annual production losses of 2.7 and 2.2 \% respectively, compared to the small gains realized by large-scale farmers (IFPRI, 2020).

Figure 3. (a) Malawi's per worker agricultural value-added and national per capita GDP adjusted for constant 2005 prices for 1980-2008 (World Bank 2015); and (b) per capita maize production and price for the same period (Chirwa et al.~2006; FAOSTAT 201

Climate impacts on biodiversity have also been recorded in Malawi. Declining lake levels in Lake Malawi has resulted in a subsequent decline in terrestrial and
aquatic biodiversity (Aragie et al., 2018; GCF, 2017). It has been observed that an increase in temperature by 5 °C can reduce the lake level by 1.42 m
(Mtilatila et al., 2020). Besides, eutrophication of lakes has led to reduced biodiversity (Hughes et al., 2019). Malawian fisheries, particularly for most
people living in rural areas, are a source of animal protein (Limuwa et al., 2018). However, between the 1970s and 2015, Malawians' fish consumption decreased by
60 \% due to low fish catches (Limuwa et al., 2018).

Droughts and floods, the leading climate impacts in Malawi result in elevated poverty levels (Actionaid, 2002). On average, poverty is 1.3 \% higher due to
droughts and 0.9\% higher each year due to floods (IFPRI, 2020). Lack of access to water, food insecurity, and low-income levels at the household level
accelerates poverty during extreme climates (Hughes et al., 2019). Table 1 demonstrates household livelihoods' exposure to climate risks in the past ten years
(Abdi et al., 2020).

\emph{Table 1. Exposure of household livelihoods to climatic risks in the past ten years in Malawi (N=1582) (Abdi et al., 2020)}

\begin{longtable}[]{@{}llllll@{}}
\toprule
\textbf{Climatic shocks} & \textbf{Exposure} & \textbf{Exposure frequency} & & & \\
\midrule
\endhead
& Household (\%) & None & 1-2 times & 3-5 times & More than 5 times \\
\textbf{Droughts} & 66.0 & 34.0 & 43.2 & 21.3 & 1.5 \\
\textbf{Floods} & 41.2 & 58.8 & 32.1 & 8.3 & 0.8 \\
\textbf{Crop pests and diseases} & 48.0 & 52.0 & 32.0 & 11.7 & 4.3 \\
\textbf{Hailstorms} & 33.3 & 66.7 & 27.0 & 5.7 & 0.6 \\
\bottomrule
\end{longtable}

Source: CIMMYT-led project on Sustainable Intensification of Maize and Legume Cropping Systems for Food Security in Eastern and Southern Africa (SIMLESA) (Abdi
et al., 2020)

Changes in rainfall patterns are highly variable (Hughes et al., 2019). Northern and Southern Malawi has experienced a drying trend since the early 2000s, while
Malawi's centre has seen slightly increased rains. Reports of extreme weather events that is, droughts, heavy rains, and floods) increased from just one during
the 1970s to between 2000 and 2006. Figure 4 presents the historical climate variability in Malawi between 1905 to 1998.

\emph{Figure 4. Historic climate variability in Malawi. (Source: The International Resources Institute for Climate and Society at Columbia University, derived from the Climate Research Unit at the University of East Angalia, the United Kingdom). Note:} Yellow-red shading (drought) shows the country's percentage that would
experience lower than normal rainfall (to different degrees). Blue shading (floods) indicates the country's percentage that would experience higher than normal
rainfall linked to floods.

\emph{Figure 5. Seasonal rainfall time series Zomba. Source: (Jørstad \& Webersik, 2016)}

Severe floods have been reported to damage infrastructure in Malawi (Hughes et al., 2019). For example, the 2015 floods highlighted the transport sector's
vulnerabilities with severe damage caused to the infrastructure. For the transport sector, the total damages and losses due to the disaster were approximately
USD 60 million, while the cost of recovery was approximately USD 130 million, the highest among all the sectors. Flood events have also reduced electric
generation from hydropower energy (Mtilatila et al., 2020). Malawi is close to the bottom of the United Nations Human Development Index league, and one of the
world's lowest electricity access rates (Dauenhauer et al., 2020) exacerbates its extreme poverty levels. An estimated 16 million Malawians currently live
without electricity or 91 \% of the country's population (Dauenhauer et al., 2020). Reduced hydropower generation at the Lujeri Micro-Hydropower Scheme in
southern Malawi during 1980--2011 was attributed to increased temperature (Mtilatila et al., 2020).

Malawi has recorded 4,901,344 confirmed malaria cases and 3,614 deaths due to the disease were reported by the World Health Organization in 2017 (Frake et al.,
2020). Areas with the highest risk of infection are concentrated in the Shire River valley along the Lake Malawi lakeshore and the central plains, which are
characteristically hotter and wetter than other parts of the country. According to Frake et al.~(2020), for Anopheles gambiae s.s., 7.25 \% of Malawi exhibits
suitable water conditions (water only), approximately 16 \% for water plus another factor, and 8.60 \% is maximally suitable, meeting thresholds of suitability for
water presence, terrain features, and climate conditions. Almost 21\% of Malawi is suitable for breeding based on land characteristics alone, and 28.24\% is
suitable for breeding based on climate and land characteristics alone (Frake et al., 2020).

\hypertarget{projected-future-climate}{%
\subsection{Projected future climate}\label{projected-future-climate}}

The temperature is anticipated to continue to rise by 1.1 to 3.0 ᵒC by the 2060s and by 1.5 to 5.0 ᵒC by the 2090s (Jørstad \& Webersik, 2016). On the other
hand, by the 2090s, annual rainfall is projected to decrease throughout Malawi by -14\% (Future Climate for Africa, 2019; GCF, 2017; Msowoya et al., 2016).
Modelling of climate change scenarios predicts significant medium- and long-term changes to Malawi's climate, in terms of both temperature and rainfall (Jørstad
\& Webersik, 2016). Extremes in temperatures (that is, hot and very hot days) are also more likely to occur more frequently (Hughes et al., 2019). Figure 6 shows
the potential highs and lows Malawi may face during the 2030s and 2040s. These extremes in temperatures can negatively affect the vulnerable, such as the old,
the young, people living in poverty, and those with health issues. Extreme temperatures can also reduce water quality, cause surges in algal growth, and
negatively affect aquatic ecosystems, including fish. Analysis of 34 climate change models projecting up to 2090 suggests more frequent dry spells and a
reduction in the number of rainy days and the amount of rainfall each day (Figure 7). It also shows a greater likelihood of flooding. These changes are likely to
threaten livelihoods, increase the risk of food insecurity, and negatively affect economic growth.

Figure 6. Changes in heat extremes in the 2030s and 2040s (Source: World Bank. 2017(c). Multi-Sectoral Investment Plan for Climate and Disaster Risk Management in Malawi)

Figure 7. Time series of mean annual temperature (°C) for 34 CMIP models (Source: UMFULA. 2017. Malawi Country Climate Brief: Future Climate Change Projections
for Malawi). Note: CMIP = Climate Model Intercomparison Project

Rain-fed agriculture contributes 90 \% of Malawi's food production. However, the incidence of extreme droughts and floods and extreme heat events is expected to
increase (Hughes et al., 2019). According to Cacho et al., (2020), crop yields are expected to be below the no climate change reference scenario for most crops
by 2050, with average yields as low as 0.83 compared to 2010 (Table 2). For the period 2020-2050, climate change's total cost to smallholders is \$1.6 (± 1.3)
billion in present-value terms (Cacho et al., 2020).

Table 2. Relative yield projections in 2050 for main crops grown by smallholders in Malawi expressed as the ration of yields under each RCP to yields under no
climate change reference scenario with CO2 fertilizer (± SD)

Figure 8. Figure 3: Average changes in the Lilongwe District's mid-and late century maize yields (\%). Compared to the baseline period (1971-2000), the average
maize yield reduction ranges for the 2050s and 2090s are 7\%-14\% and 13\% -33\%, respectively (Source: (Msowoya et al., 2016).

Climate change impacts on infrastructure have been projected to reduce the growth rate of Malawi's GDP. Based on a comprehensive analysis using median climate
scenarios directly related to changes in temperature and precipitation up to 2050, it has been estimated that, without adaptation measures to the planning,
construction and maintenance of ad infrastructure, Malawi faces a potential annual average total cost of USD 165 million. The capacity installed and electricity
generated at the hydropower plants in Malawi account for 80.2\% and 98\%, respectively, of the country's total electricity power (Mtilatila et al., 2020).
Reduction in annual hydropower production between 1\% (RCP8.5) and 2.5\% (RCP4.5) during 2021--2050 and between 5\% (RCP4.5) and 24\% (RCP8.5) during 2071--2100 has
also been projected (Mtilatila et al., 2020).

Table 3. Sub-national climate projection data for 2020 -- 2099 period

\begin{longtable}[]{@{}
  >{\raggedright\arraybackslash}p{(\columnwidth - 6\tabcolsep) * \real{0.01}}
  >{\raggedright\arraybackslash}p{(\columnwidth - 6\tabcolsep) * \real{0.08}}
  >{\raggedright\arraybackslash}p{(\columnwidth - 6\tabcolsep) * \real{0.17}}
  >{\raggedright\arraybackslash}p{(\columnwidth - 6\tabcolsep) * \real{0.73}}@{}}
\toprule
\textbf{Hazard} & \textbf{Northern Malawi} & \textbf{Central and Southern Malawi} & \textbf{Country-level projections} \\
\midrule
\endhead
Temperature & Extreme temperature increase & Temperatures are projected to increase from the 1971--2000 baseline by between 1.4 and 1.6\,°C by 2035 and 1.9 and 2.5\,°C by 2055 under Representative Concentration Pathways (RCPs) 4.5 and 8.5 respectively, & - A clear warming trend is apparent in annual temperature throughout the country - The temperature is anticipated to continue to rise by 1.1 to 3.0 ᵒC by the 2060s and by 1.5 to 5.0 ᵒC by the 2090s - The greatest increases in temperature are projected for the2080s - Mean annual temperature will rise by\,1.99°C (1.38°C to 2.80°C)\,in\,2040-2059\,(RCP 8.5, Ensemble) - As climate evolves, overall increase in the frequency and intensity of heatwaves is observed. - All models show an increase in the number of days with temperatures above 30°C\,(a\,threshold sometimes used to examine the sensitivity of maize to heat stress) \\
Total monthly rainfall & Drying\,is pronounced in all the seasons in northern Malawi. A prolonged drying trend has occurred in December to February & Precipitation projections are more uncertain Moderate wetting trends appear in central part of Malawi. & - There is a lot of variability in rainfall amounts and seasonality - Annual precipitation will decrease by -63.59mm (-350.81mm to 271.99mm) in 2040-2059 (RCP 8.5, Ensemble) - By the 2090s, annual rainfall is projected to decrease throughout Malawi by -14\% - In the 2020s there could be more rainfall at the start of the rainy season (December) but lessrainfall thereafter (January to April). - In the 2050s, we can expect more rainfall during some months (December to March) of the rainy seasonunder RCP4.5 and relatively high rainfall variability under RCP8.5. - At the end of the century, in the 2080s, we project the rainy season to be shorter for both RCP4.5 and RCP8.5 because of less rainfall at both the beginning and end of the season. - All models consistently project increases in the proportion of rainfall that falls in heavy events inthe annual average of up to 19 \% by the 2090's. - For September-October-November (spring/early summer) we see a\,likelihood of drying of up to 10\%. - For December-January-February (summer), we see an overall small increase in rainfall in the country of up to 4\%. By the 2090s, we\,see larger projected\,changes. \\
Drought & Below normal crop yields between 2045-2090 & - Increased hunger and food insecurity - -Increased droughts in southern Malawi & - More severe droughts expected between with approximated GDP declines of 21.53\% in agriculturalsector driven largely by the large fall in maize production - Decline in food availability expected - Model results estimate that droughts, on average, cause GDP losses of almost 1 percent every year \\
Floods & Increased frequency of flood events & In southern Malawi, the following are expected: - Increased flood events - Increased flood related losses The average annual GDP loss due to floods is about 0.7percent or US\$9 million, thus making the averageimpact of floods slightly less than that of droughts. & - Flash floods projected to increase throughout Malawi - Floods mainly affect small and medium-scale farmers. \\
& & & \\
\bottomrule
\end{longtable}

\hypertarget{assessment-by-key-systems}{%
\subsection{Assessment by key systems}\label{assessment-by-key-systems}}

There are a number of key systems in Malawi (Table XX and Figure XX below on Malawi Systems). Each of these are addressed below and interlinkages between them
are explored. Local and national economic development in Malawi depends on livelihoods from natural resources and food security. Systems that provide for
livelihoods are vulnerable to the impacts of climate change and overexploitation due to growing demand from rapid population growth. The NAP has been designed to
capture the need to adapt in the critical subsystems of food production including crop, livestock, fisheries, and water resources. The food production system is
strategically linked with economic value chains as broader market interventions at the macro level. The fisheries subsystem is complements with crop and
livestock in livelihoods - as a source of income as well as a source of nutrition. Water resources support both agriculture (crop and livestock) and fisheries
and has strong linkages across productivity and adaptation needs. Other systems under consideration include health, the hydropower, the sewage and waste, urban
planning and development, and transport (Table 1).

Table XX: Systems affected by climate change and its impacts, mapped to national development goals, SDGs and Sendai Framework using the NAP SDG iFrame. KPA1 --
Agriculture, Water development and Climate Change Management, KPA2- Education and Skills Development, KPA3- Energy, Industry and Tourism Development, KPA4-
Transport and ICT Infrastructure, KPA5- Health and Population

\begin{longtable}[]{@{}
  >{\raggedright\arraybackslash}p{(\columnwidth - 10\tabcolsep) * \real{0.02}}
  >{\raggedright\arraybackslash}p{(\columnwidth - 10\tabcolsep) * \real{0.21}}
  >{\raggedright\arraybackslash}p{(\columnwidth - 10\tabcolsep) * \real{0.09}}
  >{\raggedright\arraybackslash}p{(\columnwidth - 10\tabcolsep) * \real{0.23}}
  >{\raggedright\arraybackslash}p{(\columnwidth - 10\tabcolsep) * \real{0.09}}
  >{\raggedright\arraybackslash}p{(\columnwidth - 10\tabcolsep) * \real{0.36}}@{}}
\toprule
Identified Systems & Key System Stressors/Weaknesses & Key Interacting/Nexus Systems & Development Goal (MGDS III) & Related SDGs & Sendai Framework (Annex III) \\
\midrule
\endhead
Crop production & Rising temperatures Increased intensity and frequency of heavy rainfall Increased frequency of droughts and length of dry spells. High costs of inputs Low adoption of agriculture technologies. Limited access to efficient markets. Limited irrigation systems. Poor land management practices. & Water resources and supply Livestock production Marketing and Transport Forests Fisheries Ecosystems and biodiversity Energy Health Social-cultural Governance & KPA1: Outcome -- Inclusive agricultural transformation adaptive to climate change KPA2: Outcome -- Improved and accessible quality and relevant education and skills for all KPA4: Outcome -- Enhanced equitable access to social and economic services, local and international markets through safe, reliable and affordable transport and ICT infrastructure KPA5: Outcome -- Improved health, nutrition status and population management and development & SDG1-No Poverty {[}1.1, 1.2, 1.4{]} SDG2-Zero Hunger {[}2.1, 2.2, 2.3, 2.4, 2.5, 2.a, 2.c{]} SDG3-Good Health and Well-Being {[}3.8, 3.9{]} & Developing, testing or introducing practices or techniques that are more resilient to disasters and climate variability in farming systems or plant breeding. Development of irrigation or drainage networks to reduce vulnerability to disasters. Developing or introducing strategies to intensify crop production to mitigate rising food prices that result from drought. \\
Livestock production & Limited pasture due to human population pressure Inadequate storage and breeding technologies in feed and breeding programs -- unavailability of artificial insemination services Insufficient health support infrastructure and services such as dip tanks. Increasing temperatures Diseases and high cost of drugs Low milk prices Poor markets for milk Land degradation & Water resources and supply Rangelands Crop production Market and transport Health Ecosystems and biodiversity Social cultural Governance Energy & KPA1: Outcome -- Inclusive agricultural transformation adaptive to climate change KPA2: Outcome -- Improved and accessible quality and relevant education and skills for all KPA4: Outcome -- Enhanced equitable access to social and economic services, local and international markets through safe, reliable and affordable transport and ICT infrastructure KPA5: Outcome -- Improved health, nutrition status and population management and development & SDG1-No Poverty {[}1.1, 1.2, 1.4{]} SDG2-Zero Hunger {[}2.1, 2.2, 2.3, 2.4, 2.5, 2.a, 2.c{]} SDG3-Good Health and Well-Being {[}3.8, 3.9{]} & Integration of disaster resilience into extension services and programmes. Introducing or strengthening soil management practices to adapt to climate hazards. \\
Agriculture Markets and Trade & Over reliance on rainfed agriculture and limited irrigation infrastructure Limited access to inputs and services Low productivity Post-harvest losses Poor or lack of processing and weak marketing strategies Inadequate and/or lack of domestic markets Poor credit repayment discipline. Low prices for some crop produce High transportation costs & Crop production Livestock production Water resources Transport system & KPA1: Outcome -- Inclusive agricultural transformation adaptive to climate change KPA3: Outcome -- Sustainable energy for spurring socio-economic development KPA4: Outcome -- Enhanced equitable access to social and economic services, local and international markets through safe, reliable and affordable transport and ICT infrastructure KPA5: Outcome -- Improved health, nutrition status and population management and development & SDG2-Zero Hunger {[}2.3, 2.b, 2.c{]} SDG8-Decent Work and Economic Growth {[}8.a,{]} SDG10-Reduced Inequalities {[}10.5, 10a{]} SDG17-Partnerships for the Goals {[}17.10, 17.11, 17.12{]} & Establish a risk management framework integrating natural hazard risk mitigation strategies Disaster risk insurance schemes for productive sectors such as agriculture, fishing etc. \\
Fisheries & Unsustainable fishing practices - Overfishing Increased surface water temperatures. Increased frequency and intensity of heavy rainfall events. Increased drought conditions & Water resources and supply Ecosystems and biodiversity Forests Crop production Market and trade Health Social cultural & KPA1: Outcome -- Inclusive agricultural transformation adaptive to climate change KPA4: Outcome -- Enhanced equitable access to social and economic services, local and international markets through safe, reliable and affordable transport and ICT infrastructure KPA5: Outcome -- Improved health, nutrition status and population management and development & SDG1-No Poverty {[}1.1, 1.2, 1.4{]} SDG2-Zero Hunger {[}2.1, 2.2, 2.3, 2.4, 2.5, 2.a, 2.c{]} SDG3-Good Health and Well-Being {[}3.8, 3.9{]} & Fishing sector policy, planning and programmes, and institution of capacity building integrating DRR. \\
Forestry & High human population growth -- increased biomass energy demands Agriculture expansion. Tobacco farming -- which requires significant number of trees for curing Brick burning Urbanization Inadequate funding Poor law enforcement - corruption & Water resources and supply Crop production Livestock production Health Social cultural Energy Market and trade & KPA2: Outcome -- Improved and accessible quality and relevant education and skills for all KPA3: Outcome -- Sustainable energy for spurring socio-economic development & SDG15-Life on Land {[}15.1, 15.2, 15.7, 15a, 15b, 15c{]} SDG6-Clean Water and Sanitation {[}6.6{]} & Introducing the use of forest systems to reduce vulnerability to landslides, flooding or other natural hazards. Reforestation and afforestation with species less vulnerable to climate variability and natural hazards. Forest fire prevention measures. Forestry sector policy, planning and programmes, and institution of capacity building integrating DRR. \\
Water Resources & Deforestation in catchment areas. Rising temperatures. Increased intensity and frequency. Increased drought conditions and dry spell length. & Crop production Livestock production Energy Health Tourism Fisheries Forests Ecosystem and Biodiversity & KPA1: Outcome -- Inclusive agricultural transformation adaptive to climate change KPA3: Outcome -- Sustainable energy for spurring socio-economic development KPA4: Outcome -- Enhanced equitable access to social and economic services, local and international markets through safe, reliable and affordable transport and ICT infrastructure & SDG6-Clean Water and Sanitation {[}6.1, 6.2, 6.3, 6.4, 6.5, 6.6, 6.a ,6.b{]} SDG3-Good Health and Well-Being {[}3.9{]} SDG15-Life on Land {[}15.1, 15.8{]} & Reducing the vulnerability of public drinking water supply and distribution systems. Strengthening of hydrometeorology capacity and early warning systems. Reducing the vulnerability to natural hazards of wastewater treatment and disposal designs. Integration of DRR measures in river basin's development and management. \\
Rural water supply & Inadequate human resources to ensure an adequate and sustainable water supply. Cultivation along river banks and streams threatens the quality of water supplied to rural households and communities. Land cover changes due to deforestation causes drying up of some streams and rivers. Droughts Increased use of agro-chemicals. & Crop production Livestock production Forests Ecosystem and Biodiversity Social Cultural & KPA1: Outcome -- Inclusive agricultural transformation adaptive to climate change KPA4: Outcome -- Enhanced equitable access to social and economic services, local and international markets through safe, reliable and affordable transport and ICT infrastructure KPA5: Outcome -- Improved health, nutrition status and population management and development & SDG6-Clean Water and Sanitation {[}6.1, 6.2, 6.3, 6.4, 6.6, 6.b{]} SDG1-No Poverty {[}1.4{]} SDG3-Good Health and Well-Being {[}3.9{]} SDG11-Sustainable Cities and Communities {[}11.1{]} & Reducing the vulnerability of public drinking water supply and distribution systems. Multi-sector aid for basic social services (including basic education, basic health, basic nutrition, population/reproductive health and basic drinking water supply and basic sanitation) that integrate DRR. \\
Urban water supply system -- Lilongwe City & Rapid human population growth. High rates of urbanization The increased emergence of low-income areas and informal settlements -- make it difficult for Lilongwe Water Board (LWB) to improve water on water service coverage. Deforestation of the Lilongwe River catchment. Inadequate infrastructure & Crop production Livestock production Forests Ecosystem and Biodiversity Social Cultural & KPA1: Outcome -- Inclusive agricultural transformation adaptive to climate change KPA3: Outcome -- Sustainable energy for spurring socio-economic development KPA4: Outcome -- Enhanced equitable access to social and economic services, local and international markets through safe, reliable and affordable transport and ICT infrastructure KPA5: Outcome -- Improved health, nutrition status and population management and development & SDG6-Clean Water and Sanitation {[}6.1, 6.2, 6.3, 6.a, 6.b{]} SDG11-Sustainable Cities and Communities {[}11.1{]} & Reducing the vulnerability of public drinking water supply and distribution systems. Multi-sector aid for basic social services (including basic education, basic health, basic nutrition, population/reproductive health and basic drinking water supply and basic sanitation) that integrate DRR. \\
Urban water supply system -- Mzuzu City & Rapid human population growth -- High demand for water Inadequate storage capacity of the Lunyangwa dam Over-reliance on loans for expansion of services -- results in loss of revenue. Interrupted power supply -- which causes machines to fail to pump water at times. & Crop production Livestock production Forests Ecosystem and Biodiversity Social Cultural & KPA3: Outcome -- Sustainable energy for spurring socio-economic development KPA4: Outcome -- Enhanced equitable access to social and economic services, local and international markets through safe, reliable and affordable transport and ICT infrastructure KPA5: Outcome -- Improved health, nutrition status and population management and development & SDG6-Clean Water and Sanitation {[}6.1, 6.2, 6.3, 6.a, 6.b{]} SDG11-Sustainable Cities and Communities {[}11.1{]} & Reducing the vulnerability of public drinking water supply and distribution systems. Multi-sector aid for basic social services (including basic education, basic health, basic nutrition, population/reproductive health and basic drinking water supply and basic sanitation) that integrate DRR. \\
Urban water supply system -- Blantyre City & Poor governance/management practices -- lack or low incentives to workers. Low tariffs collection -- due to illegal connections, and vandalism Rapid rates of urbanization Inadequate finances to expand the water supply system. Power outages Droughts and low rainfall & Crop production Livestock production Forests Ecosystem and Biodiversity Social Cultural & KPA3: Outcome -- Sustainable energy for spurring socio-economic development KPA4: Outcome -- Enhanced equitable access to social and economic services, local and international markets through safe, reliable and affordable transport and ICT infrastructure KPA5: Outcome -- Improved health, nutrition status and population management and development & SDG6-Clean Water and Sanitation {[}6.1, 6.2, 6.3, 6.a, 6.b{]} SDG11-Sustainable Cities and Communities {[}11.1{]} & Reducing the vulnerability of public drinking water supply and distribution systems. Multi-sector aid for basic social services (including basic education, basic health, basic nutrition, population/reproductive health and basic drinking water supply and basic sanitation) that integrate DRR. \\
Energy -- Hydroelectricity generation system & Encroachment of hydro- power dams for sand winning -- Affect stability of dams. Low water levels due to inadequate and/or erratic rainfall. High maintenance costs. Operational losses due to power theft or informal power connections & Water resources Ecosystem and Biodiversity Market and Trade & KPA1: Outcome -- Inclusive agricultural transformation adaptive to climate change KPA2: Outcome -- Improved and accessible quality and relevant education and skills for all KPA3: Outcome -- Sustainable energy for spurring socio-economic development KPA4: Outcome -- Enhanced equitable access to social and economic services, local and international markets through safe, reliable and affordable transport and ICT infrastructure & SDG7-Affordable and Clean Energy {[}7.1, 7.2, 7.3, 7a, 7b{]} SDG 15- Life on Land {[}15.1{]} SDG6-Clean Water and Sanitation {[}6.6{]} SDG8-Decent Work and Economic Growth{[}8.1, 8.2, 8.3{]} & Incorporation of the potential impacts of disasters in the design standards of generation, transmission and distribution lines and power system reliability assessments. Integration of DRR considerations in energy sector planning and institution capacity building. Supporting the increased production of climate smart sources of energy. \\
Energy -- Electricity distribution infrastructure & Low distribution capacity. Poor transmission infrastructure Wildfire burning high transmission poles in forests Windstorms pulling down transmission lines Bushfires causing explosion of transformers. Increasing air temperatures affecting carrying capacity and transmission potential. & Market and Trade Transport Governance Forests & KPA2: Outcome -- Improved and accessible quality and relevant education and skills for all KPA3: Outcome -- Sustainable energy for spurring socio-economic development KPA4: Outcome -- Enhanced equitable access to social and economic services, local and international markets through safe, reliable and affordable transport and ICT infrastructure & SDG7-Affordable and Clean Energy {[}7.1, 7.2, 7.3, 7a, 7b{]} SDG 15- Life on Land {[}15.1{]} SDG6-Clean Water and Sanitation {[}6.6{]} SDG8-Decent Work and Economic Growth{[}8.1, 8.2, 8.3{]} & Incorporation of the potential impacts of disasters in the design standards of generation, transmission and distribution lines and power system reliability assessments. Integration of DRR considerations in energy sector planning and institution capacity building. Supporting the increased production of climate smart sources of energy. \\
Urban waste management system - Landfills & Poor public attitude towards waste disposal. High rates of urbanization which have increased demand for landfills. Lack of proper waste management plans. Low frequency of waste collection by the city assembly due to limited number of vehicles. Lack of machinery for digging trenches and compacting waste in landfills. Poor effluent discharge into rivers. Agriculture chemicals Heavy rains and /or floods which washes waste from an around landfills into river systems & Health Water resources and supply Crop production Livestock production Social cultural & KPA1: Outcome -- Inclusive agricultural transformation adaptive to climate change KPA3: Outcome -- Sustainable energy for spurring socio-economic development KPA4: Outcome -- Enhanced equitable access to social and economic services, local and international markets through safe, reliable and affordable transport and ICT infrastructure & SDG11-Sustainable Cities and Communities{[}11.6{]} SDG12-Responsible Consumption and Production{[}12.3, 12.4, 12.5{]} SDG3-Good Health and Well-Being{[}3.9{]} SDG6-Clean Water and Sanitation {[}6.2, 6.3{]} & Preventive measures to counteract increased exposure to diseases related to disasters. \\
Medical waste management & Rapid human population growth which has increased amount of waste generated. High maintenance cost for medical waste disposal systems. Expansion of health facilities has resulted in high-cost biological waste disposal & Health Water resources and supply & KPA1: Outcome -- Inclusive agricultural transformation adaptive to climate change KPA3: Outcome -- Sustainable energy for spurring socio-economic development KPA4: Outcome -- Enhanced equitable access to social and economic services, local and international markets through safe, reliable and affordable transport and ICT infrastructure KPA5: Outcome -- Improved health, nutrition status and population management and development & SDG3-Good Health and Well-Being {[}3.9{]} & Incorporating disaster-related health risks into clinical practice guidelines, and curricula for continuous medical education and training. Preventive measures to counteract increased exposure to diseases related to disasters. \\
Sewage waste management systems & Generation of high amounts of sewage beyond the capacity of treatment plants -- due to high human population growth. Heavy rains which flood sewer reservoirs. Rapid expansion of urban settlements & Health Water resources and supply Crop production Livestock production Social cultural & KPA1: Outcome -- Inclusive agricultural transformation adaptive to climate change KPA4: Outcome -- Enhanced equitable access to social and economic services, local and international markets through safe, reliable and affordable transport and ICT infrastructure KPA5: Outcome -- Improved health, nutrition status and population management and development & SDG3-Good Health and Well-Being{[}3.9{]} SDG6- Clean Water and Sanitation{[}6.2, 6.3{]} SDG11-Sustainable Cities and Communities{[}11.3, 11.6{]} & Preventive measures to counteract increased exposure to diseases related to disasters. Reducing the vulnerability to natural hazards of wastewater treatment and disposal designs.  \\
Transport system & Inadequate road networks Inadequate freight and rail capacity Inadequate financial resources Delayed maintenance of various roads Heavy rains -- which cause floods or increased runoff which degrade roads Poor road surface conditions Weak enforcement of town planning regulations. & Water resources and supply. Energy Markets and Trade & KPA1: Outcome -- Inclusive agricultural transformation adaptive to climate change KPA4: Outcome -- Enhanced equitable access to social and economic services, local and international markets through safe, reliable and affordable transport and ICT infrastructure & SDG7-Affordable and Clean Energy {[}7a,7b{]} SDG9-Industry, Innovation and Infrastructure {[}9.1, 9a{]} SDG11-Sustainable Cities and Communities {[}11.2{]} SDG3-Good Health {[}3.6{]} & Embedding disaster-resilient elements in the existing transportation network. Assessing economic, environmental, or social impacts of natural hazards on transportation, as well as disaster risk impacts of new transport and infrastructure investments. Introducing disaster resilient building codes in road construction projects. \\
River bank flood planning & Extensive deforestation Heavy rainfall High poverty levels Limited funding Limited community participation Over-reliance on aid and lack of ownership by local communities of flood management projects Increased settlements in flood prone areas Lack of flood protection infrastructure & Water resources Energy Agriculture Health & KPA1: Outcome -- Inclusive agricultural transformation adaptive to climate change KPA4: Outcome -- Enhanced equitable access to social and economic services, local and international markets through safe, reliable and affordable transport and ICT infrastructure KPA5: Outcome -- Improved health, nutrition status and population management and development & SDG15-Life on Land {[}15.2, 15.3{]} & Development of flood prevention / control measures: floods from rivers or the sea; including sea water intrusion control and sea level rise related activities. Construction of evacuation shelters for communities to use in times of natural disasters. Development of disaster helplines \\
Urban flood planning & Poor drainage systems High human population density -- difficult to install or modify drainage systems Lack or untimely maintenance of drainage systems Informal settlements & Water resources Energy Agriculture Health & KPA1: Outcome -- Inclusive agricultural transformation adaptive to climate change KPA3: Outcome -- Sustainable energy for spurring socio-economic development KPA5: Outcome -- Improved health, nutrition status and population management and development & SDG11-Sustainable Cities and Communities {[}11.5, 11.a, 11.b{]} & Construction of evacuation shelters for communities to use in times of natural disasters. Development of disaster helplines. Development of flood prevention / control measures: floods from rivers or the sea; including sea water intrusion control and sea level rise related activities. \\
Ecosystems & Land use/cover change due to deforestation Rising temperature Erratic rainfall Increased drought conditions & Forests Water resources and supply Crop production Livestock production Crop production Market and Trade Health Social cultural Tourism Energy & KPA1: Outcome -- Inclusive agricultural transformation adaptive to climate change KPA3: Outcome -- Sustainable energy for spurring socio-economic development KPA4: Outcome -- Enhanced equitable access to social and economic services, local and international markets through safe, reliable and affordable transport and ICT infrastructure KPA5: Outcome -- Improved health, nutrition status and population management and development & SDG15-Life on Land {[}15.1, 15.2, 15.3, 15.4, 15.5, 15.6, 15.7, 15.8, 15.9, 15.a, 15.b, 15.c{]} & Reforestation and afforestation with species less vulnerable to climate variability and natural hazards.  \\
Social-Cultural System & Over-exploitation of natural resource for housing, clothing, medicine, energy, livelihoods Socio-cultural importance of livestock ownership Chronic poverty Low adaptive capacities & Ecosystems and biodiversity Forestry Crop production Livestock production Fisheries Rangelands Water resources and supply Energy Tourism Cities and human settlements Health & KPA1: Outcome -- Inclusive agricultural transformation adaptive to climate change KPA2: Outcome -- Improved and accessible quality and relevant education and skills for all KPA3: Outcome -- Sustainable energy for spurring socio-economic development KPA4: Outcome -- Enhanced equitable access to social and economic services, local and international markets through safe, reliable and affordable transport and ICT infrastructure KPA5: Outcome -- Improved health, nutrition status and population management and development & SDG8-Decent Work and Economic Growth {[}8.5, 8.6, 8.7{]} SDG5-Gender Equality {[}5.1, 5.2, 5.3{]} SDG11-Sustainable Cities and Communities {[}11.4{]} & Specific targeting of groups vulnerable to natural hazards for social protection programmes. Development of social protection strategies / safety nets to respond to natural disasters. \\
Health & Rising temperature Increased heavy rainfall events -- cause floods Increased drought conditions Increased human population Low number of health workers & Crop production Livestock production Ecosystems Water resources and supply Forests & KPA1: Outcome -- Inclusive agricultural transformation adaptive to climate change KPA4: Outcome -- Enhanced equitable access to social and economic services, local and international markets through safe, reliable and affordable transport and ICT infrastructure KPA5: Outcome -- Improved health, nutrition status and population management and development & SDG3-Good Health and Well Being {[}3.1, 3.2, 3.3, 3.4, 3.5, 3.6, 3.7, 3.8, 3.9, 3.a, 3.b, 3.c, 3.d{]} SDG2- Zero Hunger {[}2.2{]} & Training of health care providers in disaster preparedness and response. Retrofitting existing health infrastructure such as health centres and hospitals with disaster resilient building codes. Assessing changes in risk (exposure and sensitivity to disaster-related diseases, including in respect of vulnerable groups and post-disaster incidence. Incorporating disaster-related health risks into clinical practice guidelines, and curricula for continuous medical education and training. Preventive measures to counteract increased exposure to diseases related to disasters. Strengthening health management information systems related to disaster risk management. Strategies that aim to improve the disaster risk management of the health and insurance system. Including disaster related diseases in basic benefits of insurance policies. \\
\bottomrule
\end{longtable}

\hypertarget{food-production-megasystem}{%
\subsubsection{Food Production Megasystem}\label{food-production-megasystem}}

The Food Production System in Malawi is complex -- highly fragmented and dependent on many small-scale producers who are often non-market oriented and vulnerable to
climate change. This is on a background of many environmental risks that impose limits to systems productivity. A recent report on the Malawi's Agri-food System
(White, 2019) demonstrates the country's agri-food system's complexity in two fundamental ways (i) the country's agri-food system comprises 80\% of the population of
about 18.1 million people consisting of smallholder farmers and many who work as food retailers, transporters, and small-scale processors. (ii) the country's food
production system operates in a complex policy debate about the role of subsidies such as the Farm Input Subsidy Program (FISP) against priorities for Greenhouse
Gas (GHG) mitigation. For the purpose of the National Adaptation Plan (NAP), this section limits description of the country's food production system within the
scope of crop, livestock and the fisheries and aquaculture subsystems from climate change adaptation context.

The performance of agri-food system in Malawi is vulnerable to a high degree of uncertainty and volatility compounded with limited adaptive capacities, especially
among smallholder farmers. This is because, the country's agri-food system relies on rainfed annual agriculture production thriving in highly variable climate,
compounded with the fall armyworm (Spodoptera frugiperda). Many people, both urban and rural are perpetually vulnerable to an annual hungry season when the previous
year's harvest has been poor. Trends have shown that, malnutrition was steadily declining from a high of 26.3\% in 1998, down to 12.1\% in 2009, back up to 16.7\% in
2014, with undernourishment in 2015 staggering at 20.7\% and declines remained slow in recent years.

The volatility and precarity of the agri-food system is exacerbated by the combined impacts of poor infrastructure, uneven and deteriorating power access, expensive
fuel, and poverty. Across the region, climate change is yet, expected to have widely variable impacts that generally exacerbate variability and extremes. The future
impacts of the agri-food system can be reflected in the response of other subsystems such as energy and infrastructures (Table 1.1a-b).

Table 1.1a: Observed climate impacts at Country Level Impact of Climate Change (CLICC) in Malawi.

\begin{longtable}[]{@{}
  >{\raggedright\arraybackslash}p{(\columnwidth - 12\tabcolsep) * \real{0.04}}
  >{\raggedright\arraybackslash}p{(\columnwidth - 12\tabcolsep) * \real{0.60}}
  >{\raggedright\arraybackslash}p{(\columnwidth - 12\tabcolsep) * \real{0.08}}
  >{\raggedright\arraybackslash}p{(\columnwidth - 12\tabcolsep) * \real{0.09}}
  >{\raggedright\arraybackslash}p{(\columnwidth - 12\tabcolsep) * \real{0.06}}
  >{\raggedright\arraybackslash}p{(\columnwidth - 12\tabcolsep) * \real{0.08}}
  >{\raggedright\arraybackslash}p{(\columnwidth - 12\tabcolsep) * \real{0.04}}@{}}
\toprule
Sector & Observed climate impacts & Global impactrating & National impactrating & Confidence rating & Data qualityrating & Time period \\
\midrule
\endhead
Agriculture & Reduced crop yield associated with heat and drought stress. & Low-High & Low-High & High & Low & 1992-2017 \\
Crops & - Changes in crop suitability due to shifts in agroecologicalzones. - Economic losses due to severe reductions in crop yieldscaused by frequent floods and droughts. & Medium-High & Medium-High & Medium-High & Low & 1992-2017 \\
Livestock & Increased animal mortality associated with intense heat,frequent droughts and floods. & Low- Medium & Low- Medium & Low & Low & 1992-2017 \\
\bottomrule
\end{longtable}

Table 1.1b: Projected climate change impacts on the agri-food system. With medium rating based on medium quality data, agricultural yield will exhibit declining trends

\begin{longtable}[]{@{}
  >{\raggedright\arraybackslash}p{(\columnwidth - 10\tabcolsep) * \real{0.13}}
  >{\raggedright\arraybackslash}p{(\columnwidth - 10\tabcolsep) * \real{0.51}}
  >{\raggedright\arraybackslash}p{(\columnwidth - 10\tabcolsep) * \real{0.09}}
  >{\raggedright\arraybackslash}p{(\columnwidth - 10\tabcolsep) * \real{0.09}}
  >{\raggedright\arraybackslash}p{(\columnwidth - 10\tabcolsep) * \real{0.12}}
  >{\raggedright\arraybackslash}p{(\columnwidth - 10\tabcolsep) * \real{0.06}}@{}}
\toprule
Sector & Projected climate impacts & Impactrating & Confidence rating & Data qualityrating & Time period \\
\midrule
\endhead
Agriculture & Declining yield among all types of crops in all parts ofthe country except in the northern region. & Medium & Low-Medium & Medium & 2007-2050 \\
Energy & Hydropower production negatively affected by highrainfall variability. & Medium & Medium & Low & 2007-2050 \\
Built Infrastructure & Increased damage to infrastructure and human settlementdue to intense flooding. & High & High & Low & 2007-2050 \\
\bottomrule
\end{longtable}

\textbf{Source:} CLICC Phase 2 Project (2019).

\hypertarget{crop-production-subsystem}{%
\subsubsection{Crop Production Subsystem}\label{crop-production-subsystem}}

Approximately, 90\% percent of the crops are rainfed, and most farmers cultivate on small parcels of land of approximately 0.5 to 1.5 ha, although Lea \& Hanmer
(2009) note that many farmers in some parts of the country leave portions of their plots fallow, which is partially due to labour constraints (Bezner-Kerr \&
Patel, 2014). It is estimated that 11\% of farmers are landless and only 13\% of households cultivate on more than 2ha (Mangelsdorf, Hoppe, Kirk, \& Dihel, 2014).
Household land farms vary across the country -- larger in the northern region than further south due to lower population density. Maize occupies at least 60\% of
cultivated land and is farmed by 97\% of farming households. It makes up 60-70\% of total food intake and 48\% of protein consumption (Kampanje-Phiri, 2016).
Average maize yields in Malawi are around 1.2 MT/ha, which is lower than the average for Africa, 1.8 MT/ha, also considered far below the average potential
(Abate et al., 2017; Mango et al., 2018).

Total cultivatable land is not fully explored in Malawi. Currently total land cultivated is about 2.5 million hectares (Agric-policy) but total suitable area for
agriculture is about 4.7 million hectares. The cash crops like, tobacco, tea, sugarcane, and macadamia are cultivated in estate subsector. The estate subsector
also provides contract farming opportunities for smallholder farmers. Out of the cultivated land 90\% is under rain fed agriculture despite that there are 407,
862 hectares of land that have the potential for irrigation farming. Out of the 400,000 hectares of land suitable for irrigation, only 14,000 hectares are under
smallholder farmer irrigation while 48,000 hectares are under estate irrigation. This indicates a huge gap that can be addressed through investment.

With quantitative data available for eight soil and terrain factors, a recent study (Li et al., 2017)4 has indicated that highly suitable, moderately suitable,
marginally suitable, and unsuitable agricultural areas account for 8.2\%, 24.1\%, 28.0\%, and 39.7\% of the total land area, respectively. The majority of suitable
lands are currently used for agriculture, but more than half (57.4\%) of Malawi's total cropland exists on marginally suitable or unsuitable land categories and
is likely a candidate for rehabilitation through sustainable agricultural practices, if the crop production subsystem is to adapt to climate change.

Over the years the government has implemented agricultural input programs to improve agriculture production in the country amidst of the challenges. Most of
these input programs have focused on Maize production the staple food. The main aim of these programs has been to improve the productivity of the smallholder
maize farms so as to ensure food security. Since the early seventy's the government has implemented six agricultural input programs which include:

\begin{itemize}
\item
  Agricultural Input Subsidy Programme: -- subsidized seed and fertilizer for smallholder farmers (1970-1995)
\item
  Supplementary Input Programme: -- Input kit distribution to vulnerable households (1995-1997).
\item
  Starter Pack Programme: -- Universal distribution of fertilizer and seed (1998-99).
\item
  Targeted Input Programme: -- Targeted fertilizer and seed distribution (2000-04).
\item
  The 2005 Extended Target Input Programme: -- Expanded targeted fertilizer and seed distribution.
\item
  Farm Input Subsidy Programme: -- Targeted voucher based maize seed and fertilizer subsidies (2006 to present).
\end{itemize}

The main aim of these programs has been to improve the productivity of the smallholder maize farms so as to ensure food security. However, some of these programs
did not achieve the intended goals hence they were phased out. For instance, despite having the Targeted input programme (2000-04) and the 2005 extended target
input programme the country still experienced severe food crises in 2002 and 2005. Currently, the Farm Input Subsidy Programme (FISP) is being implemented where
smallholder famers are provided with coupons which allow them to purchase hybrid maize seed and fertilizers at relatively low prices. The FISP programme has
positive impacts on maize production and net crop income but limited impact on food consumption and household income (references). Furthermore, weaknesses of the
programme have been pointed out including its financial sustainability and identification of beneficiaries (IFPRI, 2013), as there is high support to the middle
income than the poorest.

The constraints to expanding irrigation for agricultural production have been:

\begin{itemize}
\item
  Focusing of the agricultural economy on rainfed agriculture and existing irrigation schemes, where emphasis was on funding extension activities.
\item
  Reluctance of donors to fund irrigation development.
\item
  Replacement of irrigation services under the Ministry of Agriculture, which has focused on rainfed agriculture.
\item
  Price setting for crops not viable for irrigation.
\item
  Almost no irrigation technology training facilities within the country.
\item
  A poorly funded and understaffed Department of Irrigation.
\end{itemize}

\hypertarget{livestock-production-subsystem}{%
\subsubsection{Livestock Production Subsystem}\label{livestock-production-subsystem}}

The\,livestock industry\,in\,Malawi\,is underdeveloped and contributes only 8\% of total GDP and about 36\% the value of total agricultural products.\,Both smallholders
and estate farmers are involved in livestock production, but due to various production bottlenecks, intensive livestock production systems are largely dominated
by estate farms. Despite that, livestock\,provides food, income, manure, animal traction and social security to some smallholder farmers. Considering all this,
livestock\,may account more than 11\% to Gross Domestic\,Production. Major livestock production comprise beef, dairy, goat, sheep, pigs, chicken and eggs, with
registered small increases in recent years. Population of cattle has been increasing by 3\% annually. In 2014 there were over 1.3 million cattle and over 6.3
million goats in the country. Livestock production experiences varying challenges including: (i) limited pasture due to human population pressure (ii) inadequate
production and storage technologies in feed and breeding programmes (iii) Insufficient health support infrastructure and services such as dip tanks.

Spatial coverage of floods responsible for economic losses are generally localized in watershed areas. This limited coverage creates localized impacts such as
crop and soil losses; hence the rating is medium. However, the impacts are felt on the national economy hence the rating is high. For example, on average, Malawi
loses US\$9 million or 0.7\% of the GDP each year due to floods in the southern region of the country. Taken together, drought and floods cost the Malawian economy
about 1.7\% of its GDP every year. This is equivalent to almost US\$22 million.

Available literature agrees on the devastating impacts of frequent and severe floods and droughts on the agriculture sector in Malawi. The sector suffers the
greatest losses, effecting declines in GDP ranging from 1.1 to 21.5\% during Return Period of 5 years (RP5) and Return period of 25 years (RP25) for droughts,
respectively. Furthermore, the literature and experts agree that low agriculture productivity resulting from climate change result in food shortages, cause
domestic grain prices to rise while grain imports increase rapidly to cover the shortfall. Maize imports, for example, increase by between 6 and 256\% during RP5
and RP25 droughts, respectively. The possibility for high rating of the impacts implies that spatial coverage is at times wide, and frequency of the impact
increases to high. For example, maize is by far the dominant crop produced in Malawi, occupying more than 70 per cent of available agricultural land and is
critically important to livelihoods. The average\,land holding\,size per\,household\,for smallholders in\,Malawi\,in the period under consideration was 1.2 hectares.
Over 90\% of the total agricultural value-added came from about 1.8 million smallholders who on average owned only 1 hectare of\,land. Flooding in low-lying areas
where productivity is inherently high affected almost the whole country in terms of food availability stability and accessibility. The alternation with droughts
in the high areas complicates crop productivity. Floods wash away livestock in the low-lying areas.

\hypertarget{agriculture-markets-and-trade-subsystem}{%
\subsubsection{Agriculture Markets and Trade Subsystem}\label{agriculture-markets-and-trade-subsystem}}

Malawi is an export-led economy and agriculture comprises 80\% of exports, with major export crop being tobacco, but sugar, tea, and coffee. The role of tobacco
as an export crop is continuously declining. Smallholder rainfed maize production dominates and comprises about 25\% of the agricultural GDP, of which agriculture
as a whole makes up around 30\% of the overall GDP (Pauw, Beck, \& Mussa, 2016). At the farm level, net revenue varies widely and may be influenced by multiple
factors such as soil conditions, farm size, infrastructure, distance to market, composition of the household, education levels, agro-climatic variability, and
other variables.

Transportation of agricultural produce/seeds and agriculture markets constitutes one of the components of the food production system. Agriculture produce is
transported differently from the farm to the storage facilities or from the farm to the market depending on distance to be covered and on the financial capacity
of the farmer. Among the modes of transport employed are; transportation by foot, bicycle, oxcarts and vehicles. Cereal crops are sold in different ways some are
sold to the Malawi government through the Agricultural Development and Marketing Corporation (ADMARC). The ADMARC sells and buys produce from farmers at
standardized prices. Other farmers who do not prefer to sell their crop produce to ADMARC usually sell at local markets or sell to vendors who usually move
around villages searching for crop produce to buy. For cash crops like tobacco the government has established structures like the Tobacco Control Commission
which regulate the sales of tobacco and facilitates exports of the produce.

Sales of livestock and fish are usually done at local markets. For African smallholder farmers to sustain the yield increases they seek, they are reliant on a
seed industry. On the other hand, a hybrid-based maize sector also requires large-scale commercial seed enterprises whose profits can be sustained only by strong
seasonal demand by farmers for renewing their seed (Haggblade \& Hazell, 2010). Leading Seed Companies in Malawi Multinational seed companies carry out seed
breeding, production, multiplication, processing, and distribution of mainly hybrid maize. Local seed companies are involved only in seed multiplication and
distribution. Malawi's main seed companies are Seed Co, Monsanto (Bayer), DowDuPont (Pannar), Demeter, and MUSECO.

\_\textbf{The Country Vision on Trade.\_}

The Malawi Vision 2063 shows the country's commitment to have an agricultural development and marketing entity running on commercial principles, promoting the
commercialization of agriculture and providing local and international structured market linkages farmers. The parastatals shall operate under a strong alliance
with the private sector, in a transparent and accountable manner and independent of political interference.

Malawi is a member of WTO since 1995 and, in recent years has targeted trade-led development through trade expansion instruments, including regional trade
agreements. The country also a Member State of the Common Market for Eastern and Southern Africa4 (COMESA) and the Southern African Development Community5
(SADC), with each one accounting for less than a quarter of the country's trade. Malawi is primarily a resource exporting country and features in the lowest
quartile among its regional trade agreement (RTA) partners in terms of GDP per capita -- in 2017, the COMESA average was US\$2,900 and the SADC average was
US\$3,720. It is also a Signatory Party to the Protocol on Free Movement of Persons of the Kigali Declaration (2018) and to the 2018 African Continental Free
Trade Agreement (AfCFTA).

This active regional trade policy is remarkable and provides several trade-led opportunities for development. However, a variety of challenges and constraints
continue to impede trade, such as licensing requirements and a system of trade permits.6 Efforts, such as single window, are underway to simplify border or
certification procedures but, overall, there is a great deal of paperwork and specific certification regulation. In addition, standards-related regulations, and
implementation, notably Sanitary or Phytosanitary measures (SPS) and other Technical Barriers to Trade (TBT), can also be an impediment to the export of
agricultural and agriculture-related products. Malawi also faces several infrastructure-related constraints including poor transport links and lack of access to
electricity for a large proportion of the population - only 10 per cent have access - mirroring a trend in many African countries. On the other hand, over half
of the population have access to radio and mobile phone services.

Figure xxx displays the value-added breakdown of the Malawian GDP. The significant change in the aggregate economy since 2017 is characterized by a decline in
the share of the industrial sector from 29\% in 1990 to 15\% in 2017 and an increase in the share of the services sector. from 26\% in 1990 to 56\% in 2017. The
former can be explained by the small size of the sector and the relative expansion of world demand for certain basic agricultural products, which has led to a
shift towards greater specialization in raw or semi-processed products. The increase in the contribution from services is striking at first, given that the
sector's predominance in generating value-added is generally a phenomenon of developed countries. The most dynamic over the past decade have been construction
and sub-sectors such as wholesale and retail trade, real estate, information and communication and financial services. Growth in the services sector is believed
to be driven by government expenditure as well as development assistance.3 The share of agriculture as a percentage of GDP has also seen a steady decline from
45\% in 1990 to about 28\% in 2017.

\textbf{FIGURE xxx:} Value-added breakdown

Source: UNCTAD secretariat calculations based on World Bank World Development Indicators data

\hypertarget{ecosystems}{%
\subsubsection{Ecosystems}\label{ecosystems}}

Ecologically sensitive and fragile ecosystems are facing threats due to poor land use practices and deforestation. Habitat fragmentation and loss threaten
biodiversity. Deforestation is leading to the loss of mechanisms for adaptation from the increased impacts of severe flooding and excessive heat waves.
of the population have access to radio and mobile phone services.

Table 1: Selected ecological systems of special significance in Malawi

\begin{longtable}[]{@{}
  >{\raggedright\arraybackslash}p{(\columnwidth - 4\tabcolsep) * \real{0.10}}
  >{\raggedright\arraybackslash}p{(\columnwidth - 4\tabcolsep) * \real{0.07}}
  >{\raggedright\arraybackslash}p{(\columnwidth - 4\tabcolsep) * \real{0.83}}@{}}
\toprule
\textbf{Name} & \textbf{Type} & \textbf{Brief Description} \\
\midrule
\endhead
Chongoni Rock Art Area & UNESCO world heritage site & Listed as a world heritage site in 2006. Covers an area of about 126.4km2. Comprises rock art which has paintings by BaTwa hunter-gathers who inhabitedthe area from the late stone age \\
Lake Malawi National Park & UNESCO world heritage site & Became a heritage site in 1984. The only It is considered of global importancefor biodiversity conservation due to its fish diversity. \\
Mulanje mountain biospherereserve & UNESCO site & Gazetted as a forest reserve in 1927. Designated as a UNESCO site in 2000. Ithas rich biodiversity with a high level of endemism. The Mulanje cedar found inthis reserve is considered the national tree of Malawi. It is also an ImportantBird Area (IBA) \\
Lake Chilwa wetland biosphere reserve & UNESCO site & Designated as a UNESCO site in 2006. It is the second largest lake in Malawi andhome to one of the most diverse populations of birds \\
Nyika National Park & National Park & Nyika National park is one of the important Afro-montane centers of biodiversity.It is included in the global 200 ecoregions which comprises the most outstandingand representative habitats for biodiversity on the planet. \\
Vwaza Marsh wildlifereserve & Wildlife Reserve & Covers an area of 1000km2 it lies along the Zambia border. It has a rich habitatwhich attracts a range of birds and has nearly 300 species of birds. It also hasa high diversity of smaller mammals. \\
Dzalanyama forest reserve & Forest Reserve & It was gazette in 1922 with the aim of protecting the forest's ecosystem andLilongwe's catchment. It is dominated by the miombo woodlands and it is alsocategorized as an important bird area (IBA). \\
Thuma forest reserve & Forest Reserve & Gazetted in 1926 and covers 197km2 in the great rift valley escarpment of LakeMalawi. The upper parts of the forest are covered with miombo woodlands while thelower parts are covered with mixed low altitude woodland with patches of bamboo.Thuma is one of a few forest reserves in Malawi which is still home to elephantand buffalo. \\
Kasungu National Park & National Park & It is the second largest National park in Malawi and covers an area of 2100km2.Vegetation mostly comprises miombo woodlands and grassy river channels. In recentyears it has faced heavy poaching which has reduced the number of animals in thispark \\
Nkhotakota wildlife Reserve & Wildlife Reserve & Is the largest and oldest wildlife reserve in Malawi and it covers 1800km2. Itexperienced high levels of poaching in the past which reduced the number of animalsin the park. However, currently it has improved in terms of animal abundance due tothe management activities of African Parks \\
Liwonde National Park & National Park & Established in 1973 and covers about 548km2. It is home to the Big five. In recent years it has witnessed a huge increase in the number of elephants such that in 2016 over 250 elephants were translocated from Liwonde National Park to Nkhotakota wildlife reserve. It is also home to high variety of bird species; habouring over 300 bird species. \\
Majete wildlife reserve & Wildlife Reserve & It is a big five reserve and an important wildlife destination in Malawi. Nearly 5,000animals of 16 species have been reintroduced including black rhino, elephant, lion,leopard, cheetah, sable antelope, and buffalo. It harbors over 400 elephants. \\
Lengwe National Park & National Park & It was designated as a national park in 1970. It consists of open deciduous forests anddense thickets. It is home to the Nyala antelope \\
\bottomrule
\end{longtable}

The Lake Malawi ecosystem is particularly an area of high freshwater biodiversity that plays a crucial role in the local economy of people living around the
lake. It is however also under increasing threat from development, deforestation, hydropower development, oil exploration and multiple other interconnected
factors. There is currently a lack of information and awareness of freshwater biodiversity within the region so existing conservation actions fail to recognize
its importance and vulnerability.\,

The country is committed to protection of ecosystems and the services. The national parks, wildlife reserves, and forest reserves cover 18\% (1.7 million
hectares) of the land mass of Malawi and a substantial proportion of the Shire Valley. Effective protection of these resources will continue to contribute
significantly to address the drivers of climate change.

Degraded ecosystems need restoration to maintain carbon storage and sequestration, and through best practice land management to combat degradation. Currently,
forests are being lost and degraded at alarming rate, driven by a range of factors, including conversion for agriculture, overharvesting of firewood, cutting for
charcoal production and increasing frequency of forest fires. Consequently, the country is experiencing unprecedented loss of habitat and their biodiversity.

These factors are likely to intensify as population pressures continue to grow. Remnant forests decline in both quality and coverage and as changing climatic
factors influence regeneration, forest fire frequency etc. Investing in the sustainable management and conservation of these remaining natural habitats, with
strategies and interventions that are informed by climate modelling, offers a potentially cost-effective way of protecting ecosystem services and contributing to
resilience. The Government of Malawi has developed a National Biodiversity Strategy and Action Plan to deal with threats to biodiversity including ecosystems.

\hypertarget{fisheries-system}{%
\subsubsection{Fisheries System}\label{fisheries-system}}

Malawi is endowed with wild fish resources with fish farming is predominantly based on finfish for both commercial and noncommercial purposes. The fisheries and
aquaculture provide essential nutrition, support livelihoods and contribute to national development in Malawi. The aquaculture sector is important to the
country's economic growth and will remain so in many years to come. As the human population grows so too will the demand for animal protein. Fish provides over
70 per cent of the dietary animal protein intake among Malawians and 40 per cent of the total protein supply. Fish also provides vital vitamins, minerals and
micronutrient.5 Much of the fish is consumed in rural areas thereby contributing significantly to daily nutritional requirements to some of the vulnerable groups
such as HIV and AIDS victims, orphans and the poor (Economic Report, 2011). Fishing is the main source of livelihood to 37,089 out of 3,984,981 households in
Malawi (NSO, 2018).

The sector directly employs nearly 59,873 fishers and indirectly over 500,000 people who are involved in fish processing, fish marketing, boat building and
engine repair. Furthermore, nearly 1.6 million people in lakeshore communities derive their livelihood from the fishing industry. The main provision of the
fishery resource comes from capture fisheries. Sustainable fisheries contributes 3 percent to the national GDP, and government has set a target of 3.8\% to be
achieved by 2022 in partial fulfilment of MGDS Key Priority Area 1: To achieve sustainable agricultural transformation that is adaptive to Climate Change (GoM,
2017).

Over the past few years, the sector has displayed signs of growth6. Total annual production volumes reached an all-time high of 164,940 tonnes in 2016, up from
about 81,400 tonnes in 2005 and 100,900 tonnes in 2010. While the bulk of fish caught, sold and consumed has traditionally been produced by capture fishery,
capture fishery production has declined in some years. This has been particularly the case for the commercially-oriented, high-value species such as the
Oreochromis karongae - locally known as `chambo' -- the average annual production of which declined from more than 10,000 tonnes between 1980 and 1990 to around
4,000 tonnes between 2000 and 20157. The annual fish production under aquaculture increased from about 800 tonnes in 2005 to about 4,900 tonnes in 2015 and 7,672
tonnes in 2016. The bulk of fish produced by aquaculture are commercially oriented, high-value species, which are being caught less by capture fishery (\textbf{Table
1.2)}.

\textbf{Table 1.2:} Trends in Malawi's Annual Fish Production and Growth for Capture Fisheries And Aquaculture

\begin{longtable}[]{@{}
  >{\raggedright\arraybackslash}p{(\columnwidth - 12\tabcolsep) * \real{0.05}}
  >{\raggedright\arraybackslash}p{(\columnwidth - 12\tabcolsep) * \real{0.13}}
  >{\raggedright\arraybackslash}p{(\columnwidth - 12\tabcolsep) * \real{0.19}}
  >{\raggedright\arraybackslash}p{(\columnwidth - 12\tabcolsep) * \real{0.15}}
  >{\raggedright\arraybackslash}p{(\columnwidth - 12\tabcolsep) * \real{0.16}}
  >{\raggedright\arraybackslash}p{(\columnwidth - 12\tabcolsep) * \real{0.12}}
  >{\raggedright\arraybackslash}p{(\columnwidth - 12\tabcolsep) * \real{0.21}}@{}}
\toprule
\textbf{Year} & \textbf{Capture (tonnes)} & \textbf{\% Growth in capture fisheries} & \textbf{Aquaculture (tonnes)} & \textbf{\% Growth in aquaculture} & \textbf{Total (tonnes)} & \textbf{\% Growth in capture \& aquaculture} \\
\midrule
\endhead
2005 & 80,609 & & 813 & & 81,422 & \\
2006 & 72,929 & (9.5) & 907 & 11.6 & 73,836 & (9.3) \\
2007 & 67,818 & (7.0) & 1,252 & 38.0 & 69,070 & (6.5) \\
2008 & 75,867 & 11.9 & 1,318 & 5.3 & 77,185 & 11.7 \\
2009 & 76,045 & 0.2 & 1,600 & 21.4 & 77,645 & 0.6 \\
2010 & 98,300 & 29.3 & 2,632 & 64.5 & 100,932 & 30.0 \\
2011 & 82,336 & (16.2) & 2,815 & 7.0 & 85,151 & (15.6) \\
2012 & 120,328 & 46.1 & 3,232 & 14.8 & 123,560 & 45.1 \\
2013 & 109,889 & (8.7) & 3,705 & 14.6 & 113,594 & (8.1) \\
2014 & 116,289 & 5.8 & 4,742 & 28.0 & 121,031 & 6.5 \\
2015 & 144,315 & 24.1 & 4,918 & 3.7 & 149,234 & 23.3 \\
2016 & 157,268 & 9.0 & 7,672 & 56.0 & 164,940 & 10.5 \\
\bottomrule
\end{longtable}

\textbf{\emph{Source}}: Department of Fisheries

Lake Malawi has potential for fisheries expansion. The various targeted species found in Lake Malawi alone have an estimated catch potential in the range of
120,000 to 200,000 tonnes, as estimated by the ODA/SADC Pelagic Resources project (M. Banda, pers. comm.).

Figure 1.1- Potential sites of aquaculture investment in Malawi {[}source{]}

Other water bodies are overfished (Lake Malombe), prone to desiccation (Lake Chilwa) or threatened by water hyacinth, Eichhornia crassipes (Lower Shire).
Identified virgin stocks in Lake Malawi, however, require expensive deep-water trawls. It is unlikely that these developments will benefit small-scale operators
and economic viability remains to be demonstrated. Malawi is now a net importer of fish to supplement its needs. It all suggests that there remains unexploited
potential increasing productivity in the aquaculture subsector to meet the growing demand.

Current observations indicate that the aquaculture sector in Malawi is vulnerable to the impacts of climate change, but smallholder fish farmers have limited c
capacity to adapt. Recent field observation shows that fish farmers in the country have been experiencing climate change in many ways depending on geographical
location. In Blantyre the Chambo Fisheries Limited has been experiencing extreme cold temperatures which eventually affect fish production and fingerling growth.
Contrastingly, in Salima and Balaka observations have shown that farmers are increasingly facing extreme hot weather conditions resulting in water shortages and
drying up of dams before harvesting the fish stocks reach harvesting stage.

Atmospheric warming could change water temperatures, which might impact production. Droughts could decrease the availability of freshwater to fill ponds or
tanks. In other areas such as Mzimba, Rumphi and Phalombe, farmers face torrential rains which at times result in heavy flooding of fishponds, and consequently
losing fish stocks. In March 2019, for example, floods triggered by Cyclone Idai washed away two Chonona Fish Farms fishponds, along with catfish stock that was
about to be harvested, resulting in significant sunk costs. Generally, the Lower Shire valley has significant potential in aquaculture production, but the region
is vulnerable to extreme climatic events which alternately occur between floods and droughts. The seasonal impact on production depends on the specific weather
conditions in agroecological zones. For instance, farmers in high-altitude areas such as the northern region city of Mzuzu are not able to produce fingerlings
during the cold months from May to July, which also restricts the production of grow-outs to a single cycle per year. On the other hand, fingerling production
and production of grow-out fish can be undertaken throughout the year in low-altitude warm areas such as the Lower Shire and most of the Lake Shore districts of
Nkhatabay, Nkhotakota and Salima.

Projections indicate also that climate change will invariably heighten risks and vulnerabilities to existing levels of variability of temperature and rainfall.
Even with the levels of uncertainty linked to climate modelling, all recent studies of Malawi's future climate broadly agree that over the next decades:
temperatures will rise, causing higher evaporation and consequent water stress, and; high levels of rainfall variability will remain. While there exists less
confidence in the exact future patterns of extremes, there is higher likelihood of dry spells and higher likelihood of intense rainfall events.

\hypertarget{forestry-system}{%
\subsubsection{Forestry System}\label{forestry-system}}

Forest cover in Malawi has declined significantly over the past years mainly due to charcoal production and agriculture expansion. The remnant forests face
pressures from human population increase and climate changes. Corrupt practices among the forestry officials also pose a threat to the survival of the remaining
forests (Table 1). Planting more trees, reducing charcoal production and curbing corrupt practices will ensure the sustainability of these forests.

\hypertarget{water-resources-megasystem}{%
\subsubsection{Water Resources Megasystem}\label{water-resources-megasystem}}

\textbf{(Subsystems: River bank flood planning; Urban flood planning)}

Malawi relies on both surface- and ground-water sources, with an extensive river system covering 20 percent of the country's surface area, comprising the Shire,
Ruo, Bua, Rukuru, and Songwe Rivers, and numerous lakes such as Malawi, Chilwa, Chiuta, and Malombe. Water resource distribution exhibits dramatic spatiotemporal
variation. Approximately, 90 percent of the runoff in major rivers occurs between December and June. The country's vast network of streams, rivers and lakes
provide water for various uses including drinking and agriculture. Lake Malawi plays a particularly important role in surface-water supply in the socio-economic
development of the country, but decreased water levels adversely affect power generation from hydro power plants and water supply in towns. These water resources
have been affected by droughts, erratic rainfall and poor agriculture practices, affecting the quantity and quality of available water (Table 1). Topographically
low-lying areas and cities are increasingly faced with severe flash floods; more generally, floods in Malawi occur widely and cause huge economic losses. The
main causes of these floods have been heavy rainfall and poor catchment management practices. With climate change projected to increase incidences of heavy
rainfall these floods are expected to increase. In urban areas, poor drainage systems and the rapid increase of informal settlements have contributed to an
increase in the frequency of floods. Frequent floods and droughts are the most severe effects of climate change in Malawi which highly impact the water system.
Apart from causing the lack of access to water supply, drought derails the economic progress for communities (Table 2.1).

\textbf{Table 2.1:} Observed impacts of climate change on the water system in Malawi. Water supply, water treatment, water collection and surface water management are all affected by the impacts of climate change.

\begin{longtable}[]{@{}
  >{\raggedright\arraybackslash}p{(\columnwidth - 12\tabcolsep) * \real{0.15}}
  >{\raggedright\arraybackslash}p{(\columnwidth - 12\tabcolsep) * \real{0.57}}
  >{\raggedright\arraybackslash}p{(\columnwidth - 12\tabcolsep) * \real{0.04}}
  >{\raggedright\arraybackslash}p{(\columnwidth - 12\tabcolsep) * \real{0.04}}
  >{\raggedright\arraybackslash}p{(\columnwidth - 12\tabcolsep) * \real{0.07}}
  >{\raggedright\arraybackslash}p{(\columnwidth - 12\tabcolsep) * \real{0.07}}
  >{\raggedright\arraybackslash}p{(\columnwidth - 12\tabcolsep) * \real{0.06}}@{}}
\toprule
Water & & High & High & Medium-High & Medium-High & 1992-2018 \\
\midrule
\endhead
Water supply & Water quantity and quality disrupted by increasing frequency of droughts and floods. & High & High & High & Low & 1992-2018 \\
Water treatment & Increased sediment, nutrient, and pollutant loadings from heavy rainfall and floods and droughts. & High & High & High & Low & 1992-2018 \\
Water collection & Damage to water infrastructure and contaminated ground and surface water sources & High & High & High & Low & 1992-2018 \\
Surface water management & Increased frequency and magnitude of floods associated withtorrential rains. & High & High & High & Low & 1992-2018 \\
& Reduction in waterflow in major rivers of the country due toreduction in rainfall & & & & & \\
\bottomrule
\end{longtable}

A study by Adhikari\,and\,Nejadhashemi (2019) examined climate change impacts on water resources in Malawi. Downscaled outputs from six general circulation models,
for the most extreme Representative Concentration Pathway (RCP 8.5), were used as inputs to the soil and water assessment tool to assess the impacts of climate
change on evapotranspiration, surface runoff, water yield, and soil moisture content at the country, watershed, and sub-basin levels by the 2050s. At the
country level, the results showed a\,--5.4\%--5.4\%\,to\,+24.6\%+24.6\%\,change in annual rainfall, a\,−5.0\%−5.0\%\,to\,+3.1\%+3.1\%\,change in annual evapotranspiration, from\,--
7.5\%--7.5\%\,to over\,+50\%+50\%\,change in annual surface runoff and water yield, and up to an 11.5\% increase in annual soil moisture. At the watershed level, results
showed an increase in annual rainfall and evapotranspiration in the north and a gradual decline towards the south. Sub-basin-level analysis showed a large
probability of increase in the annual precipitation, surface runoff, water yield, and soil moisture, especially in the north. Overall, the northern region was
found to be more prone to floods, while the southern region was found to be more prone to droughts.

\hypertarget{water-supply-megasystem}{%
\subsubsection{Water Supply Megasystem}\label{water-supply-megasystem}}

\textbf{Subsystems: Rural water supply; Urban water supply system -- Lilongwe City; Urban water supply system -- Mzuzu City; Urban water supply system -- Blantyre City}

Malawi is water stressed and the per capita water availability continues to decline due to human population growth especially in the urban and peri urban areas\\
(World Bank, 2007); thus, water withdrawal for agriculture/irrigation as well as for municipal purposes has been rising concurrently with population growth.
However, in the past decades Malawi has made significant progress in increasing water supply coverage. In 2015 WHO/UNICEF Joint monitoring Programme (JMP)
estimated that coverage for improved water supply was 90\% nationally; 89\% in rural areas and 96\% in urban areas (WHO/UNICEF, 2015), surpassing its Millennium
Development Goal water supply target. In 2014 over 80\% of people had access to improved water sources within a distance of 200 m for urban and 500 m for rural
areas and 93\% had an average time to collect drinking water (return trip) of less than 30 minutes (MoIWD, 2014). In rural areas water source options include
piped water and community hand pumps as well as household point of -use water treatment (Holm et al 2016). Nevertheless, water supply is being affected by
climate change as evidenced by the increased frequency of droughts and floods (Pauw et al 2010; Chidanti-Malunga et al., 2011). However, the water supply
services in the country experiences several challenges which makes water access in the country not equitable. One of the main challenges is the low functionality
of the rural water supply services; with an estimate of about 25\% water points not working at a given time (MoIWD, 2014).

Piped water supply in Malawi falls under water boards; the Northern Region Water Board (NRWB), Central Region Water Board (CRWB) and the Northern Region Water
Board (NRWB). These regional water boards cater for the northern, central and southern region of Malawi. However, cities like Lilongwe and Blantyre have other
water boards namely the Lilongwe Water Board (LWB) and the Blantyre Water Board (BWB) which aim at catering for the needs of population of their respective
cities. The water boards in all parts of the country experience similar challenges which are a result of increasing human population which result in increased
demand for water (Table 1). Most of the water treatment plant built in different districts were designed for a smaller human population than the current
population. Hence, in recent years the country has been experiencing water intermittent water supply. Some water boards have made an effort to construct water
storage infrastructure and dams to meet the growing demand for water. Even though these initiatives will solve water challenges in the short run, there is still
need for more funding to the various water board to prepare for long term challenges.

\hypertarget{energy-megasystem}{%
\subsubsection{Energy megasystem}\label{energy-megasystem}}

\textbf{Subsystems: Hydroelectricity generation system; Electricity distribution infrastructure}

Most of the energy demands in Malawi are met by biomass energy, with biomass energy satisfying over 90\% of the energy needs. The increasing human population in
the country is exerting huge pressure on biomass energy sources like forests. This has led to wide spread deforestation in the country. Climate change which has
increased the frequency of droughts also poses a threat to energy needs in the country. Droughts imply that the regeneration potential of trees is lowered due to
inadequate water. Further, droughts result in lower water levels in rivers consequently affecting hydro-power production.

\hypertarget{waste-management-megasystem}{%
\subsubsection{Waste management megasystem}\label{waste-management-megasystem}}

\textbf{Subsystems: Urban waste management system - Landfills; Medical waste management; Sewage waste management systems}

\textbf{Urban Waste Management System}
Increased human population has led to an increase in waste generated, for instance Lilongwe city accumulates over 200 tons of waste per day. These wastes are
usually dumped in landfills around residential areas and markets waiting for the city/town assembly to collect them. However, due to inadequate finances, the
collection by the city/town assembly is irregular resulting in accumulation of wastes. consequently, the wastes produce bad smell and sometimes find their way
into water bodies; thus, posing a high risk of causing diseases.

In Major Cities of the country, councils are responsible for waste collection, transportation, and disposal at designated dumping sites. However, the quantity of
solid waste collected remains smaller than solid waste generated. In most areas of the cities where settlement is unplanned, waste collection is absent leading
to environmental hazards in the form of air pollution from burning, direct contact and vermin. The system of waste management remains rudimentary in the urban
areas. Many townships of Malawi dispose waste in pits dug within their plots, while some throw waste on the roadside, the riverside and very few utile community
skips. As there are no properly designed sanitary landfills, waste collected by cities is dumped in the designated open dumpsites with huge implications on
health for those living adjacent these sites.

It is clear that waste management, pollution, inadequate access to sanitation services and poor urban conditions are some of the major challenges to development
in Malawi. The MW 2063 recognizes that the environment and the Vision pillars have overlapping effects on each other, with unplanned urbanization often
associated with environmental downsides. Industrial growth has for long been associated with increased pressure and demand on land and pollution of water and
air. It is also water intensive with heightened demand for fuel which is not necessarily clean. Poor Industrial waste management coupled with loss of forest
cover have increased the destruction risks on flora and fauna and endangered species. Industrial activity associated with unregulated disposal of waste.

Increased human population has led to an increase in waste generated, for instance Lilongwe city accumulates over 200 tons of waste per day. These wastes are
usually dumped in landfills around residential areas and markets waiting for the city/town assembly to collect them. However, due to inadequate finances, the
collection by the city/town assembly is irregular resulting in accumulation of wastes. consequently, the wastes produce bad smell and sometimes find their way
into water bodies; thus, posing a high risk of causing diseases.

\hypertarget{health}{%
\subsubsection{Health}\label{health}}

Climate change has the potential to worsen the health situation in Malawi by increasing infant mortality, and waterborne diseases, as well as increase pest and
diseases that affect the crop and livestock production systems, resulting in low yields which contribute to food insecurity. With increase in temperatures
incidences of Malaria are expected to increase and spread to higher altitudes. Increased incidences of heavy rainfall and floods are likely to put vulnerable
communities under poor sanitation and great risk of diseases like cholera. Additionally, climate change has the potential to increase incidences of malnutrition
due to low agriculture productivity which affects food availability.

The Second National Communication assessed the associations between weather and malaria, cholera, diarrhea, and undernutrition, with additional climate change
expected to increase the levels of risk. Undernutrition is one of the most important health and welfare problems facing Malawi. Agriculture is predominantly
subsistence, so droughts and floods (regular occurrences) severely reduce crop yields and food security. There is a significant relationship between climate
change and undernutrition - which has been described as a `hunger risk multiplier'. Climate change exacerbates existing rates of undernutrition through three
causal pathways: (i) impacts on household access to sufficient, safe and adequate food; (ii) impacts on care and feeding practices; and (iii) impacts on
environmental health and access to health services. Declines and variability of crop yields could have significant negative implications for nutrition and
stunting, and even when calorie consumption is adequate there can still be micronutrient deficiencies.

There is a strong relationship between temperature and diarrhea, where the incidence is related to food-borne diseases caused by high temperatures. In addition,
diarrheal outbreaks are frequently associated with the aftermath of floods, due to contamination of water supplies. As an example: WHO analysis suggests that,
under a high emissions scenario, diarrheal deaths attributable to climate change in children under 15 years old is projected to be about 10.6\% of the almost
5,800 diarrheal deaths projected in 203069. Although diarrheal deaths are projected to decline to about 3,100 by 2050 the proportion of deaths attributable to
climate change will rise to approximately 14.9\%. Cholera epidemics have been occasionally reported, with the 2001/2 epidemic associated with over 33,000 infected
and over 1,000 deaths. These epidemics occurred more often in dry years when people are forced to rely of contaminated water, although it can also be exacerbated
by floods, when these contaminate water sources. Malaria is increasingly being reported in high altitude plateaus and hilly areas that were malaria free four to
five decades ago. In part this is due to changes in rainfall patterns and increase in temperature, although socio-economic determinants also account for spatial
variations in malaria risk. Temperature was not associated with malaria incidence over the period 1974-2006, and there was a negative relationship between
rainfall and malaria.

Currently, Malawi operates a three-tier health system. The first tier is primary healthcare. This sector is in effect to meet the needs of general medical care,
which includes community and rural hospitals and maternity units. The second tier consists of district hospitals. These see patients who receive a referral from
their primary care physician to receive specialized services. This includes laboratory work and rehabilitation services. The final tier is tertiary care provided
by central hospitals. This tier covers extreme conditions that require highly specialized care such as treatment for specific diseases. The linkage for these
services comes through an elaborate referral system that trickles down the health system. Although the 2008 doctrine worked to lay out different measures to
ensure the quality of health service delivery in Malawi, major health concerns still persist. HIV/AIDS continues to be the number one cause of death in Malawi:
21.7 percent of deaths in 2012 were linked to HIV/ AIDS. Acute Respiratory Infections account for 8.6 percent of deaths, while Malaria accounts for 40 percent of
22.hospitalized individuals.

The government of Malawi developed The Health Sector Strategic Plan II (2017-2022), whose goal is to achieve universal health coverage of quality, equitable and
affordable health care with the aim of improving health status, financial risk protection and client satisfaction. HSSP II has one of the objectives being to
reduce environmental and social risk factors that have direct impact on health. The MW2063 envisions a healthy population with improved life expectancy working
towards the socioeconomic transformation of Malawi. The goal is to attain universal health coverage with quality, equitable and affordable health care for all
Malawians. This will be achieved by providing a comprehensive health care system through interventions that will address shortfalls in the recruitment,
distribution and retention of health workers; strengthening reproductive, maternal, neonatal, child and adolescent health; improving the availability and quality
of health infrastructure, medical equipment, medicines and medical supplies; and exploring innovative and sustainable financing for health while focusing on
efficiency enhancing measures such as strengthening governance, among other interventions. Every constituency in the country shall have well-equipped and staffed
hospitals and health centers with commensurate investment in public health and medical health programmes, including E-health. Malawi shall have a health sector
with advanced data capturing and management systems to support decision-making and policy formulation. Malnutrition has a significant bearing on our children's
future development and health with wider implications on socio-economic development.

\hypertarget{transport-system}{%
\subsubsection{Transport System}\label{transport-system}}

The transport system in Malawi comprises two key players the private transport and public transports. Public transport is dominated by minibus and buses. The
transport system plays a key role in the economy of the country, Nevertheless, it is characterized by several challenges like the poor road network and
inadequate finances for road construction and maintenance. These challenges are worsened by increased frequency of floods which washes some bridges and roads
resulting to high maintenance costs.

The Malawi Vision 2063 envisages an integrated transport system that will not only support domestic economic activity but also build global linkages for the
national economy. The country has a multi-modal but underdeveloped transport system consisting of road, rail, air and inland water transport.

\begin{itemize}
\item
  \textbf{Road transport subsystem:} The country's transport system is dominated by roads which carry more than 70 percent of internal freight and close to 90 percent of international freight. The Malawi Vision 2063 further strives to have a world-class, well maintained and expanding road network connecting the urban and rural areas to local and international markets. This will be done through development of transport masterplans at the national, city, town and council levels and adhered to.
\item
  \textbf{Rail transport subsystem:} Experience has shown that efficient rail and water transport is cheaper than road transport, especially for bulk freight over long distances. However, the country has a rail route which remains unreliable because of poor infrastructure and water transport is not fully developed with dilapidated ports infrastructure. The MW2063 commits to have an expanded and modernized railway system as an attractive alternative transport mode.
\item
  \textbf{Air transport subsystem:} In terms of aviation the MW 2063 commits to create an aviation sector that is internationally competitive and expanded to attract more competition from global players.
\item
  \textbf{Water transport subsystems:} To facilitate trade, the country shall have a water transport system that is expanded to generate wealth for the economy.
\end{itemize}

\hypertarget{socio-cultural-system}{%
\subsubsection{Socio-Cultural System}\label{socio-cultural-system}}

Culture in Malawi is embedded in the dominant modes of production, consumption, lifestyles and social organization that give rise and relevance to adaptation to
climate change. For example, the belief in disasters linked to anger of spirits has been culturally embedded in many traditions. This has implications on
adaptive responses.

The large proportion of the Malawi population leaves in the rural areas depend on natural resources for livelihoods. Culture plays a critical rural in natural
resources management. The preservation, promotion and retention of our cultural values that promote sustainable natural resources management will remain vital
for adaptation to climate change. Upcoming generations must be encouraged to patriotically embrace our culture and tradition, especially those values that
promote sustainable natural resources management.

Figure XX: Malawi systems August 2020 LEG mmap.

\hypertarget{national-adaptation-priorities}{%
\chapter{National Adaptation Priorities}\label{national-adaptation-priorities}}

\hypertarget{key-risks-and-adaptation-options}{%
\section{Key risks and adaptation options}\label{key-risks-and-adaptation-options}}

The key risks and adaptation options are presented based on analysis and summary of past and current data and reports up to 2020. The risk levels are divided into three temporal periods: near future (2020-2040 which is the period for which most of the granular {[}sub-regional{]} climate projections are based; mid-future (MF -- covering the period 2040-2069) and far future (FF -- the period 2070 to 2099). Risk level is assigned based on the criteria outlined below and expert judgment as presented in the reports in the framework outlined in the NAP technical guidelines report. It is evident that there are inadequately projected risks,particularly beyond 2040 for most of the systems/sectors and how to address this gap is addressed in section 8 of this report.

\textbf{Risk assessment criteria (scores are provided in brackets, with a possible highest score of 24, and ranked as follows: high (20 or more; medium (15-19), low (14 and below):}
\emph{The probability of a given climate hazard} -- The general probability for change in a climate hazard (such as temperature or extreme precipitation events) occurring.

\begin{itemize}
\item
  \textbf{High} probability of the climate hazard occurring (3);
\item
  \textbf{Medium} probability of the climate hazard occurring (2);
\item
  \textbf{Low} probability of the climate hazard occurring (1).
\end{itemize}

\emph{The likelihood of impact occurrence} -- The likelihood that a change in a given climate hazard (e.g.~temperature rise) will result in a particular impact (e.g.~material failure). Examples of likelihood categories include:

\begin{itemize}
\item
  \textbf{Virtually certain/already occurring} -- Nearly certain likelihood of the impact occurring over the life of the infrastructure, and/or the climate hazard may already be impacting infrastructure (3);
\item
  \textbf{High} likelihood of the impact occurring over the life of the infrastructure (2);
\item
  \textbf{Moderate} likelihood of the impact occurring over the life of the infrastructure (1);
\item
  \textbf{Low} likelihood of the impact occurring over the life of the infrastructure (0).
\end{itemize}

\emph{The magnitude of the consequence} -- The combined impacts, should a given hazard occur, taking into account such factors as:

\begin{itemize}
\item
  \textbf{Internal operations}, including the scope and duration of service interruptions, reputational risk, and the potential to encounter regulatory problems (1 - low to 3 -- high);
\item
  \textbf{Capital and operating costs}, including all capital and operating costs to the stakeholder and revenue implications caused by the climate change impact; (1 - low to 3 -- high);
\item
  \textbf{Number of people impacted}, including considerations related to any impacts on vulnerable populations (including, but not limited to seniors, low-income communities, mentally or physically disabled citizens, homebound residents, and children); (1 - low to 3 -- high);
\item
  \textbf{Public health}, including worker safety; (1 - low to 3 -- high);
\item
  \textbf{Economy}, including any impacts to the city's economy, the price of services to customers, and clean-up costs incurred by the public; (1 - low to 3 -- high);
\item
  \textbf{Environment}, including the release of toxic materials and impacts on biodiversity, the state's ecosystems, and historic sites. (1 - low to 3 -- high).
\end{itemize}

\textbf{Table XX: Projected climate changes and impacts. NF -- near future (2020-2040), MF -- mid-future (2041-2069), FF -- far future (2070-2099)}

\begin{longtable}[]{@{}
  >{\centering\arraybackslash}p{(\columnwidth - 14\tabcolsep) * \real{0.16}}
  >{\centering\arraybackslash}p{(\columnwidth - 14\tabcolsep) * \real{0.04}}
  >{\raggedright\arraybackslash}p{(\columnwidth - 14\tabcolsep) * \real{0.21}}
  >{\raggedright\arraybackslash}p{(\columnwidth - 14\tabcolsep) * \real{0.43}}
  >{\centering\arraybackslash}p{(\columnwidth - 14\tabcolsep) * \real{0.10}}
  >{\centering\arraybackslash}p{(\columnwidth - 14\tabcolsep) * \real{0.02}}
  >{\centering\arraybackslash}p{(\columnwidth - 14\tabcolsep) * \real{0.02}}
  >{\centering\arraybackslash}p{(\columnwidth - 14\tabcolsep) * \real{0.02}}@{}}
\toprule
\textbf{Parameter} & \textbf{Hazard/Threat} & \textbf{Impacts} & \textbf{Vulnerabilities at Scale} & \textbf{Affected Systems} & \textbf{Risk (NF)} & \textbf{Risk (MF)} & \textbf{Risk (FF)} \\
\midrule
\endhead
Below normal rainfall (Actionaid, 2002; Aragie et al., 2018, 2018; Future Climate for Africa, 2019; GCF, 2017; Hughes et al., 2019; IFPRI, 2020; Mwanaleza, 2017) & Drought & Decreased crop production Increased food crisis Depleted food reserves and unaffordable food prices Maize export ban By the 2090s, annual rainfall is projected to decrease throughout Malawi by -14\% Water scarcity & Several hundred hunger-related deaths Erosion of social capital and informal social support systems in poor communities Malawi loses 4.6 \% of its maize production each year due to droughts Droughts cause poverty to increase further Lack of access to water in rural areas Donor reliance & Crop production Livestock production Agriculture markets and trade Social-Cultural Health & High & High & Low \\
(Actionaid, 2002; Aragie et al., 2018; Future Climate for Africa, 2019; GCF, 2017; Hughes et al., 2019; IFPRI, 2020) & Biodiversity loss & Declining lake levels Terrestrial and aquatic biodiversity decline Decreasing fish catches Water pollution Increased loss of vegetation & Substantial lake levels decline below the Lake Malawi Outflow Threshold Decreased irrigation water supply in the Shire River Basin Decreasing fish catches Increasing human pressure and poor governance of natural resources Increasing biological and chemical pollution of water from urban areas and industrial waste Food insecurity in rural areas Eutrophication of lakes leading to reduced biodiversity Loss of biodiversity and degeneration of the ecosystem & Energy - hydropower Fisheries Ecosystems Forestry Water resources & Medium & High & High \\
(Hughes et al., 2019; Japan International Cooperation Agency (JICA), 2020; Mapulanga \& Naito, 2019) & Land degradation & Declining soil fertility and soil loss Water depletion Solid water disposal Deforestation Erosion and sedimentation along slopes, river and stream banks & Pollution and increased vulnerability to climate change Destruction of crop land Sedimentation of water sources Threat to social and economic development Weak land tenure security Poor environment health Decreased access to piped water leading to the use of unprotected sources Control erosion mitigation and reduce sedimentation Protect source water from sedimentation & Ecosystems Water resources Health & High & Med & low \\
Above normal rainfall (Actionaid, 2002; Hughes et al., 2019; IFPRI, 2020; Malawi, 2018; Ministry of Natural Resources Energy and Mining, 2018) & Floods, storms, Intense runoff, Hail & Localized floods Shire River flooding Chronic and acute respiratory diseases Generally wetter days for Malawi Disruption to infrastructure Increased landslides & A shortage of agricultural commodities Reduced maize production -- about 12 \% per year Maize production losses in the southern Malawi Average loss of 0.7 \% of the annual GDP due to the flooding of lakes and the overflowing of rivers. Increased risk of contracting pneumonia in children Increased risk of flooding from groundwater and surface water Disruption of rail and road transportation & Crop production Livestock production Health Energy Water supplies Transport & High & Med. & low \\
High rainfall variability (GCF, 2017; Mtilatila et al., 2020; Pauw et al., 2011) & & Changes in surface water flows including flooding Lake fluctuations -- below normal rainfall & The combined effects of temperature increase and rainfall decrease result in significantly lower flows in the Shire River Reduced hydropower production Increased poverty among urban and nonfarm households Increased national food shortages and higher domestic prices. & Energy Water supplies Social-Cultural Food & High & Med & Med \\
Shorter rainfall seasons, Late onset of rainy season (Hughes et al., 2019; Mwanaleza, 2017) & & Water shortages & Increased irrigation in rural areas Increased poverty among the vulnerable rural communities & Food production Water supplies & High & Medium & Low \\
High temperatures (Global Facility for Disaster Reduction and Recovery (GFDRR), 2015; Hughes et al., 2019; Mtilatila et al., 2020) & Heat & Water availability Vermin and pests Reduction in cloud cover Increased irrigation Increased mean annual temperature by 0.21°C per decade over the last 30 years & Decreasing water access trends during dry spells Uncertain water availability Increased risk of changes in distribution of vermin and pests Skin conditions due to increased exposure to sunshine & Crop production Livestock production Water resources Water supplies Health & High & Med & Low \\
& Biodiversity loss & Decreasing lake levels & An increase in temperature of 5 ◦C reduces the lake level by 1.42 m & Fisheries Energy production & Med & Low & Low \\
\bottomrule
\end{longtable}

\hypertarget{ranking-adaptation-actions}{%
\subsection{Ranking adaptation actions}\label{ranking-adaptation-actions}}

The adaptation options listed below have been ranked using a set of criteria that is partly modified from Sinay and Carter (2020) to make it simple for a large
group of diverse stakeholders to come to consensus easily on the priority adaptation actions which will be unpacked in the project development plans. The
adaptation options are clustered under over-arching adaptation themes which are the most likely to generate synergistic and wide-reaching co-benefits for the
country as a whole. The project development plans will take into consideration other specific criteria that will assess aspects such as alignment with SDGs,
Sendai Framework and Country GCF programmes, and inclusion of cross-cutting factors such as gender, vulnerable groups, policy and legislative reforms, and
knowledge and capacity building at individual, community, institutional and systemic levels.

\textbf{\emph{Criteria}}:

\begin{longtable}[]{@{}
  >{\raggedright\arraybackslash}p{(\columnwidth - 8\tabcolsep) * \real{0.17}}
  >{\raggedright\arraybackslash}p{(\columnwidth - 8\tabcolsep) * \real{0.08}}
  >{\raggedright\arraybackslash}p{(\columnwidth - 8\tabcolsep) * \real{0.08}}
  >{\raggedright\arraybackslash}p{(\columnwidth - 8\tabcolsep) * \real{0.06}}
  >{\raggedright\arraybackslash}p{(\columnwidth - 8\tabcolsep) * \real{0.60}}@{}}
\toprule
\textbf{Criteria} & \textbf{Indicator} & \textbf{States/Score} & \textbf{Values} & \textbf{Observations} \\
\midrule
\endhead
\textbf{Uncertainty} & Scenario & 1.5ᵒC & 1 & The state of this indicator relates to the average temperature increase used for planning. 1 -- Near future \\
\, & \, & 3ᵒC & 2 & \,2 -- Mid future \\
\, & \, & 5ᵒC & 3 & \,3 -- Far future \\
\textbf{Costs} & Costs & Low & 3 & Low in comparison to other responses \\
\, & \, & Moderate & 2 & Moderate in comparison to other responses \\
\, & \, & High & 1 & High in comparison to other responses \\
\textbf{Decision-Making time horizons} & Timing & Urgent & 3 & If the implementation of the adaptation option can avoid life threatening situations. \\
\, & \, & Convenient & 2 & When the implementation of the adaptation option is not urgent, but is in synchrony with ongoing development. \\
\, & \, & Inconvenient & 1 & Implementation of the adaptation option is not urgent and may significantly impact the existing development plans \\
\textbf{Co-benefits} & Natural Systems & Low & 1 & No natural system co-benefits \\
& & Moderate & 2 & Some benefits to natural systems \\
& & High & 3 & Many benefits to natural systems \\
\, & Human Systems & Low & 1 & Few people benefit (social, economic/livelihoods, inclusivity, gender) \\
& & Moderate & 2 & Moderate number of people benefit \\
& & High & 3 & Large number of people benefit \\
\textbf{Positive Systems Synergies} & Positive impacts & Low & 1 & Largely confined within a single system \\
\, & \, & Moderate & 2 & Links strongly to 2-3 systems \\
\, & \, & High & 3 & Links strongly to 4 or more systems \\
\textbf{Negative Systems Synergies} & Negative impacts & Low & 3 & Low potential of negative impacts on another system (e.g.~aquaculture in dams can increase nutrient levels in water supply systems) \\
\, & \, & Moderate & 2 & Medium potential of negative impacts on another system \\
\, & \, & High & 1 & High potential of negative impacts on another system \\
\bottomrule
\end{longtable}

\textbf{\emph{Adaptation Options -- Ranked:}}

\begin{enumerate}
\def\labelenumi{\arabic{enumi}.}
\item
  Sustain and protect ecosystems and ecosystem goods and services including the integrity of water resources through: general tree planting and creation of
  buffer zones along river banks, lakes and wetlands in rural and urban areas; encouraging use of alternative energy sources and energy efficient appliances;
  capacity building and implementation of community based catchment and natural resources management; support agricultural intensification, and; enhanced law
  enforcement.
\item
  Promote a climate-resilient food production system by implementing a range of strategies including: growing crops that are resilient to projected higher
  temperatures and early maturing to cope with the shifting growing season; use of organic manure to buttress high costs of fertilizers; provision of subsidies on
  farm inputs and access to loans; enhancing the infrastructure for and management of irrigation to reduce dependence on rain fed agriculture; strengthening and
  increasing the reach of extension services and training for farmers; access to drought and flood early warning information; provision of storage facilities for
  various crop and livestock products, and; creation of local markets through establishment of more agricultural companies.
\item
  Forecasting and early warning systems for droughts and floods should be in place at sub-national scale and relevant for adoption and application in the
  different systems and increase funding to the national disaster risk management programs.
\item
  Promotion of alternative economic opportunities for lakeshore people to alleviate fishing pressure and enhanced collaboration between the fishermen and the
  department of fisheries to promote sustainable fish utilization.
\item
  In order to sustain energy production and distribution under climate change, the following measures should be undertaken: adopt alternative renewable energy
  sources like solar energy to diversify the energy mix and reduce environmental pressures from woodfuel demand; expand and reinforce water storage in dams by
  planting trees around dams; relocate settlement areas demarcated for dam construction, and; ensure that dams comply with all environmental regulations and laws.
\item
  In order to ensure rural and urban water supplies under climate change, the following should be undertaken: restoration of the Lilongwe River catchment and
  dam catchments through re-afforestation; expansion of the Lunyangwa and Kamuzu dams; expansion of the water transmission and distribution networks; reduction
  of non-revenue water, and; exploration for new water sources such as the Likhubula project which extracts water from Mulanje mountain.
\item
  Climate proof supply distribution systems (water/power), waste management systems (sanitation) and transport systems (roads, railways, bridges) and improve
  the connectivity within the different systems.
\item
  Implement effective waste management in urban areas by: increasing the capacity of city or town assemblies to collect waste from residential areas; enhancing
  regular treatment of solid and liquid waste from industries and biomedical facilities before disposal and discharge; effecting automation and regular maintenance
  of the sewer systems to detect leakages and prevent blockages and obstruction; constructing landfills in areas less prone to floods, and; carrying out civic
  education of the public to change their attitudes such as towards waste segregation and recycling.
\item
  Improve general population health through provision of sanitation and hygiene infrastructure and awareness creation, increasing the number of health workers,
  and relocation of people from flood prone areas.
\end{enumerate}

\hypertarget{implementation-strategy-for-the-nap}{%
\chapter{Implementation Strategy for the NAP}\label{implementation-strategy-for-the-nap}}

\hypertarget{projects-for-implementation-and-guidelines}{%
\section{Projects for implementation and guidelines}\label{projects-for-implementation-and-guidelines}}

Table XX: Existing country programmes for climate change adaptation

\begin{longtable}[]{@{}
  >{\raggedright\arraybackslash}p{(\columnwidth - 14\tabcolsep) * \real{0.19}}
  >{\raggedright\arraybackslash}p{(\columnwidth - 14\tabcolsep) * \real{0.44}}
  >{\raggedright\arraybackslash}p{(\columnwidth - 14\tabcolsep) * \real{0.05}}
  >{\raggedright\arraybackslash}p{(\columnwidth - 14\tabcolsep) * \real{0.07}}
  >{\raggedright\arraybackslash}p{(\columnwidth - 14\tabcolsep) * \real{0.03}}
  >{\raggedright\arraybackslash}p{(\columnwidth - 14\tabcolsep) * \real{0.07}}
  >{\raggedright\arraybackslash}p{(\columnwidth - 14\tabcolsep) * \real{0.10}}
  >{\raggedright\arraybackslash}p{(\columnwidth - 14\tabcolsep) * \real{0.06}}@{}}
\toprule
Project / Programme & Objectives & Funders & Implementing Agency & Type of Project & Duration & Priority Sectors & Geographic Focus \\
\midrule
\endhead
Building Climate Change Resilience in the Fisheries Sector in Malawi & To improve Lake Malawi and coastal area community resilience to climate change through the development of an early warning system, and sustainable fisheries and aquaculture, in order to ensure food and livelihood security. & Least Developed Countries Fund & FAO & Full Size & 09/11/2016 to 31/07/2021 & Climate Change Adaptation & Malawi \\
Shire Valley Transformation Program - I & To provide access to reliable gravity fed irrigation and drainage services, secure land tenure for smallholder farmers, and strengthen management of wetlands and protected areas in the Shire Valley. & GEF Trust Fund & The World Bank & Full-size Project & Project Approved for Implementation 08- Aug 2017 & Climate Change, Biodiversity & Malawi \\
Technology Needs Assessments - Phase III (TNA Phase III) & Provide participating countries targeted financial and technical support to prepare new or updated and improved TNAs, including Technology Action Plans (TAPs), for prioritized technologies that reduce greenhouse gas emissions, support adaptation to climate change, and are consistent with Nationally Determined Contributions and national sustainable development objectives & GEF Trust Fund & United Nations Environment Programme & Full-size Project & Project approved for implementation, 13/03/2018 & climate change & Global \\
Food-IAP: Enhancing the Resilience of Agro-Ecological Systems (ERASP) & Enhance the Provision of Ecosystem services and improve the Productivity and Resilienceof Agricultural Systems of Vulnerable Rural Poor. & GEF Trust Fund & International Fund for Agricultural Development & Full-size Project & Project Approved for Implementation 02/04/2017 & Climate Change, Biodiversity, Land degradation & Malawi \\
Global Partnership on Wildlife Conservation and Crime Prevention for Sustainable Development (PROGRAM) & Promote wildlife conservation, wildlife crime prevention and pro-conservationsustainable development to reduce impacts to known threatened species from poaching andillegal trade. & GEF Trust Fund & The World Bank & Full-size Project & Concept proposed, 01 Jun 2015 & Climate Change, Biodiversity, Land degradation & Global \\
Food-IAP: Fostering Sustainability and Resilience for Food Security in Sub-Saharan Africa - An Integrated Approach (IAP-PROGRAM) & Support countries in target geographies for integrating priorities to safeguard and maintain ecosystem services into investments improving smallholder agriculture and foodvalue chains(Target 10-12 countries; 10 million ha of production landscapes; 2-3 millionbeneficiary households) & GEF Trust Fund & International Fund for Agricultural Development & Full-size Project & Concept approved, 01 Jun 2015 & Climate Change, Biodiversity, Land degradation & Regional \\
Climate Adaptation for Sustainable Water Supply & Sustain availability of water supply in the river courses \& climate proof waterresources outputs of the Sustainable Rural Water \& Sanitation Infrastructure forImproved Health \& Livelihood project in five districts: Rumphi, Nkhotakota, Ntcheu,Mangochi and Phalombe & Least Developed Countries Fund & African Development Bank & Full-size Project & Project Approved for Implementation, 10 Jan 2019 & Climate Change & Malawi \\
Umbrella Programme for Biennial Update Report to the United NationalFramework Convention on Climate Change (UNFCCC) & To support thirty nine (39) Least Developed Countries (LDCs) and Small Islands DevelopingStates (SIDS) prepare and submit good quality initial biennial update reports to theUNFCCC that comply with the convention's reporting obligation & GEF Trust Fund & United Nations Environment Programme & Full-size Project & Project Approved for Implementation, 23 Jun 2015 & Climate Change & Global \\
Building Climate Change Resilience in the Fisheries Sector in Malawi & To improve the resilience of fishing communities around Lake Malombe to the effects ofclimate change & Least Developed Countries Fund & Food and Agriculture Organization & Full-size Project & Project Approved for Implementation, 28 Aug 2016 & Climate Change & Malawi \\
Strengthening Climate Information and Early Warning Systems in Malawi toSupport Climate Resilient Development and Adaptation to Climate Change & To strengthen the weather, climate and hydrological monitoring capabilities, earlywarning systems and available information for responding to extreme weather and planningadaptation to climate change in Malawi. & Least Developed Countries Fund & United Nations Development Programme & Full-size Project & Project closed, 02 Jul 2019 & Climate Change & Malawi \\
Shire Natural Ecosystems Management Project & Shire River Basin planning framework developed to improve land and water management forecosystem and livelihood benefits in target areas & GEF Trust Fund & The World Bank & Full-size Project & Project closed, 31 May 2019 & Climate Change, Land degradation, Biodiversity & Malawi \\
Malawi Climate Transparency Framework & To strengthen the capacity of institutions in Malawi and set up an information system tofulfill the enhanced Transparency requirements of the Paris Agreement. & GEF Trust Fund & United Nations Environment Programme & Medium-size Project & Proposed start and completion dates 1/1/2021-12/31/2023 & Climate Change & Malawi \\
Sustainable Forest Management Impact Program on Dryland SustainableLandscapes & To avoid, reduce, and reverse further degradation, desertication, and deforestation ofland and ecosystems in drylands through the sustainable management of production landscapes & GEF Trust Fund & Food and Agriculture Organization & Full-size Project & Concept Approved, 01 Jun 2019 & Climate Change, Biodiversity, Land Degradation & Global \\
Malawi-climate resilient and sustainable capture fisheries, aquaculturedevelopment and watershed management project & Aimed at improving the resilience of Malawi's inland fisheries and the associate landmanagement in the face of climate change, with a focus on local community engagement,aims to complement and leverage an approved AfDB project focusing more on the enterprisedevelopment aspects of the same challenge. & Least Developed Countries Fund & African Development Bank & Full-size Project & Concept Approved , 01 Dec 2019 & Climate Change & Malawi \\
GEF-7 Africa Minigrids Program & The Africa Minigrid Program seeks to support selected African countries to increaseenergy access by reducing the cost and increasing commercial viability of renewableenergy mini-grids. & GEF Trust Fund & United Nations Development Programme & Full-size Project & Concept Approved, 01 Dec 2019 & Climate Change & Regional \\
Adapting to Climate Change Through Integrated Risk Management Strategiesand Enhanced Market Opportunities for Resilient Food Security andLivelihoods & The project seeks to enhance climate adaptation and food security of households throughaccess to integrated climate risk management strategies and structured marketopportunities, with a focus on the most vulnerable & Adaptation Fund & World Food Programme (WFP) & \, & Start date -06/11/2020 & Food security & Malawi / Africa \\
South-South Cooperation Grant (SSC) & On 16 February 2016, the Government of Malawi received peer support for accreditationof a National Implementing Entity to the Adaptation Fund from the National EnvironmentManagement Authority (NEMA) of Kenya. & Adaptation Fund & National Environmental Management Authority (NEMA), Kenya & Readiness Grants & Approval Date-16/02/2016 & Climate Change & Malawi \\
Enhancing Adaptive Capacity and Livelihood Diversification for the RuralPoor of Northern Malawi & The Project seeks to enhance resilience and adaptive capacity of rural communities thatare peripheral to Nkhata Bay, Mzimba, Karonga, Rumphi and Chitipa urban areas in thenorthern Malawi. & Adaptation Fund & African Development Bank (AfDB) & Regular sized project & Proposal Submission date - 15 Feb 2019 & Rural development & Malawi \\
Building urban climate resilience in South-eastern Africa & To develop capacities and establish conditions to adapt to the adverse effects of climatechange in vulnerable cities of Madagascar, Malawi, Mozambique and the Union of Comoros; & Adaptation Fund & United Nations Human Settlements Programme (UN-Habitat) & \, & Projected duration July 2018- January 2023 & Disaster risk reduction and early warning systems & Madagascar, Malawi, Mozambique and Union of Comoros \\
National Climate Change Programme & This project enables the Government of Malawi to climate-proof the policies, strategiesand plans of the sectors of the economy most directly affected by climate change. & Green Climate Fund and patners & Department of Environment Affairs, Malawi & \, & End date December 2020 & Climate Change & Malawi \\
Saving Lives and Protecting Agriculture Based Livelihoods in Malawi:Scaling Up the Use of Modernized Climate Information and Early WarningSystems'' (M-Climes) & The project will support the Government of Malawi to take important steps to save livesand enhance livelihoods at risk from climate-related disasters. & Green Climate Fund & Local and National Governments & Medium-size Project & Projected start date: 01/05/2017 & Climate Change & Malawi \\
Improving Preparedness to Agro-Climatic Extremes in Malawi (IPACE-Malawi) & To identify critical agro-climatic drought and flood indicators in three districts ofcentral and southern Malawi; test the skill of short term to seasonal forecast tools insimulating these indicators; and co-design agricultural climate services based on theseindicators/forecast tools. & Natural Environment Research Council, NERC & University of Leeds, School of Earth and Environment & \, & Oct 18 - Apr 21 & Agri-environmental science, Climate \& Climate Change, Development studies & Malawi \\
Building Food Security and Social Resilience to HIV/AIDS in Malawi & Working with farmers and stakeholders, researchers will document how climate change isperceived or experienced by age, gender, HIV/AIDS status, food security status andhousehold structure. They will examine how government organizations are conceptualizingand responding to climate change. & International Development Research Centre & Several Researchers from different institutions & \, & 22-03-2013 to 22-03-2017 & Food, Environment, Health & Malawi \\
\,Climate Adaptation for Rural Livelihoods and Agriculture (CARLA)\, & The overall goal to improve communities' resilience to climate variability and climatechange by developing and implementing adaptation strategies and measures that willimprove agricultural production and rural livelihoods were largely achieved. & Global Environment Facility (GEF) & African Development Bank & \, & April 12, 2012-June 30, 2016 & Climate Change & Malawi \\
\,Global Climate Change Alliance (GCCA) -- Malawi, Planning for climatechange & The project seeks to build capacity in climate change planning, with emphasis onMalawi's irrigation sector. & European Union & Human Dynamics & \, & Jun 2015-Dec 2017 & Environment & Malawi \\
Adaptation planning support for Malawi through UNEP & to reduce vulnerability of people in Malawi and to promote community and ecosystemresilience to the impacts of climate change and gender-equitable adaptive capacity forplanning and implementing adaptation interventions. & Green Climate Fund & United Nations Environment Programme & Concept note & Proposal received by the GCF Secretareat, 26 Feb 2019 & Climate Change & Malawi \\
Building climate resilience of food insecure smallholder farmers inSouthern Malawi & The project addresses the adaptation needs of vulnerable smallholder farmers and theircommunities in the Southern Region of Malawi, who have limited resilience and adaptivecapacities towards climate-related shocks. & Green Climate Fund & Sahara and Sahel Observatory & Concept note & Date of Submission 11/18/2019 & Climate Change, Food Security & Malawi \\
Climate Investor One & Providing financing to develop renewable energy projects in regions with power deficitsto reduce energy costs and CO2 emissions. & Green Climate Fund & Environmental Affairs Department & Large size project & 20 Oct 2018- 21 June 2039 & \, & Malawi and 17 other countries \\
\bottomrule
\end{longtable}

\hypertarget{promoting-action-by-all-actors-and-stakeholders-policy-and-capacity-development-outreach}{%
\subsection{Promoting action by all actors and stakeholders: policy and capacity-development, outreach}\label{promoting-action-by-all-actors-and-stakeholders-policy-and-capacity-development-outreach}}

The National Adaptation Programme and Action (NAPA) (2016) outlines key stakeholders and actors described below:

\begin{itemize}
\item
  \textbf{The Environmental Affairs Department (EAD)} is the designated government agency responsible for managing Malawi's environmental policies and programs. The EAD is accountable for coordinating NAPA projects, with line ministries carrying out specific projects. The EAD is also the lead organization in climate change planning for the Government of Malawi.
\item
  \textbf{Cabinet Committee on Health and Environment} is the highest level executive decision-making entity for environmental affairs in Malawi. The National Council
  for the Environment is a government watchdog that ensures coordination with various stakeholders, promotes compliance with environmental regulation, and
  monitors development projects to incorporate environmental concerns.
\item
  \textbf{Department of Climate Change and Meteorological Services} is tasked with providing climate and weather information and services. Its role in adaptation includes producing climate change scenarios and improving forecasting and early warning systems.
\item
  \textbf{Department of Poverty and Disaster Management Affairs} is the government watchdog for coordinating disaster and management activities. The department
  improves preparedness and response to changing disaster risks. The department also collaborates with relevant stakeholders in promoting the use of climate-proof
  structures, particularly in flood-prone areas
\item
  \textbf{Ministry of Agriculture and Food Security} promotes agricultural and rural development in Malawi. The ministry helps to educate and promote climate change
  adaptation practices as well as ensuring food security.
\item
  \textbf{Ministry of Irrigation and Water Development} ensures the provision and equitable access to water throughout Malawi. It enhances climate change adaptation
  through the provision of water supply as climate-induced threats increase.
\item
  \textbf{Ministry of Environment and Climate Change Management} plays a regulatory role by ensuring climate change projects and programmes are implemented following
  the international climate change protocols and conventions, national policies, regulations and guidelines. It also provides the infrastructure to ensure the
  information on climate change mechanisms has the widest reach.
\item
  \textbf{Environmental Affairs Department of the Ministry of Natural Resources and Environment} serves as the focal point for the United Nations Framework Convention on Climate Change (UNFCCC). It coordinates climate change adaptation planning for the government of Malawi.
\item
  \textbf{Malawi Red Cross Society} is a non-governmental organization that helps to reduce human suffering during disasters. The organization strives to reduce the
  vulnerability of those at risk and increase preparedness for more frequent disasters.
\item
  \textbf{National Smallholder Farmers' Association of Malawi} empowers and represent farmers interests. It promotes adaptation practices and supports access to risks
  pooling schemes.
\item
  \textbf{Academia and research institutions} who generate new local knowledge and develop appropriate adaptation solutions
\item
  \textbf{Private sector stakeholders} who provide and implement specific commercial adaptation solutions through public-private partnerships
\item
  \textbf{Media} to propagate climate change adaptation messages.
\end{itemize}

\hypertarget{addressing-vulnerabilities-and-risks-in-key-systems-and-sectors}{%
\subsection{Addressing vulnerabilities and risks in key systems and sectors}\label{addressing-vulnerabilities-and-risks-in-key-systems-and-sectors}}

The vulnerabilities and risks in key systems and sectors have been presented in sections 6 and 7 of this report. The approach to addressing these were discussed
during the 2016 workshop and are outlined in table XX below. Some examples of potential projects and costs are listed in Table XXX below; these should be to the
extent possible aligned with the GCF country programme and other relevant climate finance mechanisms (international and domestic) and implementing partners as
well as take into account technology and capacity building.

Table XX: Addressing vulnerability and risks in key sectors/systems (Map Stocktaking Report 2016)

\begin{longtable}[]{@{}
  >{\raggedright\arraybackslash}p{(\columnwidth - 2\tabcolsep) * \real{0.15}}
  >{\raggedright\arraybackslash}p{(\columnwidth - 2\tabcolsep) * \real{0.85}}@{}}
\toprule
ELEMENT A. LAY THE GROUNDWORK AND ADDRESS GAPS1. Initiating and launching the NAP process2. Stocktaking: Initiated in 2016 for identifying availableinformation on climate change impacts, vulnerability and adaptation,and assessing gaps and needs of the enabling environment for the NAP process.3. Addressing gaps and weaknesses in undertaking the NAP process4. Comprehensively and iteratively assessing development needs and climatevulnerabilities & - The NAP process was launched officially on September 2, 2014, where workshops wereconducted across the country for capacity building and raising awareness.- Malawi is also an active participant in the Open NAP initiative being supported by the LDC Expert Group (LEG)- The stocktaking recommended the following thematic areas be considered for the medium and long-term adaptationplanning horizon of the NAP process: 1. Improving access to energy sources, 2. Increasing resilience of foodproduction systems, 3. Improving weather and climate forecasting, 4. Improving agriculture to ensure farmersare moving from subsistence to commercialization, 5. Promoting catchment management practices, 6. Integratedwater resource management to encourage large scale commercial irrigation, 7. Population change and humansettlements, 8. Civic education and adult literacy, 9. Infrastructure development, 10. Inclusiveness of gender,disability and other socially excluded vulnerable groups in implementing climate change adaptation interventions,11. Monitoring of climate: adequate database and easy access for all people, 12. Development of collaborativewildlife management, 13. Education, science and green technology- While the stocktaking did not find conspicuous gaps in the landscape of climate change knowledge relevant toMalawi, it was clear that the most accurate and often-referenced modelling data is now somewhat obsolete. Thusnewer research will better warn the NAP process is going forward.- Similarly, while there is an adequate amount of literature for each of the sectors to have an initialunderstanding of future threats from climate change, there is a need for more Malawi-specific studies, alongwith more sector-specific studies.- The main weaknesses the stocktaking found in Malawi's climate change adaptation architecture were low levelsof awareness about climate change at all levels of society, a limited number of experts in the various sectorsof climate change adaptation; absence of climate change centres of learning and research; lack of locally drivensustainable climate change funding; and weak institutional capacity for managing climate change.- Climate vulnerabilities will be assessed at the sector, sub-national and national levels and involves analyzingthe current climate to identify vulnerability, risks, and trends in variables and indices at the national,regional or ecosystem level that could be used to support planning and decision making. Comprehensively analyzecommunity resilience and the staff to be trained on methodologies for climate risk assessment and adaptationplanning, including the integration of climate change in budgeting processes \\
\midrule
\endhead
\bottomrule
\end{longtable}

\hypertarget{better-informed-decision-making-climate-information-services-early-warning-science-and-technology-decision-support-modeling-research}{%
\subsection{Better informed decision-making: climate information services, early warning, science and technology, decision-support modeling, research}\label{better-informed-decision-making-climate-information-services-early-warning-science-and-technology-decision-support-modeling-research}}

Essential cross-cutting projects/programmes would include:

\begin{itemize}
\item
  Improving community resilience through the development of sustainable rural livelihoods.
\item
  Restoring forests in the Upper, Middle, and Lower Shire valleys.
\item
  Improving agricultural production under erratic rains and changing climatic conditions.
\item
  Improving Malawi's preparedness to cope with droughts and floods.
\item
  Improving climate monitoring to enhance Malawi's early warning capability and decision-making.
\item
  Improving sustainable utilization of Lake Malawi and its lakeshore.
\end{itemize}

\hypertarget{mobilization-of-other-sources-of-finance}{%
\chapter{Mobilization of other Sources of Finance}\label{mobilization-of-other-sources-of-finance}}

xxx

\hypertarget{monitoring-and-evaluation-of-adaptation-actions-and-process}{%
\chapter{Monitoring and evaluation of adaptation actions and process}\label{monitoring-and-evaluation-of-adaptation-actions-and-process}}

The guiding framework for monitoring and evaluation is:

\begin{itemize}
\item
  \begin{enumerate}
  \def\labelenumi{\alph{enumi}.}
  \tightlist
  \item
    Monitoring and evaluation system (mapped to goals and objectives, and targets where applicable; 5 types of metrics: inputs, process, output, outcome, impact). Indicators for each programme/project to be identified to track outcomes.
  \end{enumerate}
\item
  \begin{enumerate}
  \def\labelenumi{\alph{enumi}.}
  \setcounter{enumi}{1}
  \tightlist
  \item
    Reporting on progress on NAPs under the UNFCCC (through National Communications/Adaptation Communications)
  \end{enumerate}
\item
  \begin{enumerate}
  \def\labelenumi{\alph{enumi}.}
  \setcounter{enumi}{2}
  \tightlist
  \item
    Link to the NDC
  \end{enumerate}
\item
  \begin{enumerate}
  \def\labelenumi{\alph{enumi}.}
  \setcounter{enumi}{3}
  \tightlist
  \item
    Reporting and outreach at the national level
  \end{enumerate}
\item
  \begin{enumerate}
  \def\labelenumi{\alph{enumi}.}
  \setcounter{enumi}{4}
  \tightlist
  \item
    Use of the PEG M\&E tool9 to monitor NAP process.
  \end{enumerate}
\end{itemize}

\hypertarget{reporting}{%
\chapter{Reporting}\label{reporting}}

\hypertarget{adaptation-communicationsndcs}{%
\section{Adaptation communications/NDCs}\label{adaptation-communicationsndcs}}

All Parties are required to submit their NDCs every five years (e.g.~2020, 2025, 2030), regardless of their respective implementation time frames. The NAP process
is being developed for the first time. It is preferred that these two types of reports are be staggered, rather than submitted during the same year, so that they in
essence feed into each other in a progressive manner and lessons learned can be integrated into the evolving NAP process.

\hypertarget{links-to-sdg-voluntary-reporting-and-sendai-framework-monitor}{%
\section{Links to SDG voluntary reporting and Sendai Framework Monitor}\label{links-to-sdg-voluntary-reporting-and-sendai-framework-monitor}}

It is reported in the UN SDG Knowledge Platform10 that Malawi has operationalized the SDG Agenda through its national development planning framework, the Malawi
Growth and Development Strategy (MGDS III). It is recognised that the Malawi National Climate Change Management Policy: serves as an overarching reference document
for policy makers in Government, the private sector, civil society, and donors concerning climate change as a priority development issue and feeds into the
country's Sector Wide Approaches (SWAPs) to inform strategic government programming, including programming for the achievement of the Sustainable Development Goals
(SDGs). The Ministry of Natural Resources, Energy and Mining has the role of facilitating, coordinating and advising in ensuring the implementation of the NCCMP as
well as setting and enforcement of relevant and acceptable standards. The Disaster Preparedness and Relief Act (DPRA) 1991 probably needs to be updated and brought
into line with the latest principals and objectives on disaster risk reduction as laid out in the Sendai Framework.

\hypertarget{gender}{%
\subsection{Gender}\label{gender}}

In 2019, Malawi's female population accounts 50.67\% of the 18.63 million total approximated population (REF). \, Malawi ranks 173 out of 188 on the UN's Gender
Inequality Index (GII) (USAID 2016) with a Gender Development Index (GDI) of 0.374 indicative of sharp gender disparities between men and women (NAP Stocktaking
report 2016). In Malawi, gender inequalities are apparent in all spheres. For instance, women have limited access and control to means of production such as
land, credit and technology, and limited rights and control on their reproductive health. The challenge for reducing gender inequality is to mainstream gender
issues in all aspects of development (Vision 2020). Women represent the main source of agricultural labor in Africa and the fact that agriculture in tropical and
subtropical areas is one of the sectors most vulnerable to climate change, some women remain vulnerable and poor (Irish Aid, 2018).

Cultural and legal norms poorly protect women making them be, typically less well educated, less numerate and literate than their male contemporaries. In Malawi,
only 16.7 of parliamentary seats are held by women. There are less women who have reached secondary level of education (14.9\%) in comparison to male counterparts
(24.2\%) 71. Due to gender and social exclusion, women often face barriers in accessing the opportunities arising from economic growth, or in taking advantage of
new resources, leadership opportunities, and assets created through climate investments. Where women are heads of households due to men's migration to towns,
they make all decisions relating to land development. GoM has made gender mainstreaming a priority in its development agenda in order to narrow the inequality
gap by ensuring rights of rural women are protected in regard to food security, non-discriminatory access to resources, and equitable participation in decision
making processes (Masi 2017).

Climate change is going to inhibit women's development and gender empowerment efforts by creating more challenging economic circumstances for the population as a
whole, but also by making access to natural resources (water, fuel-wood etc) more difficult. Women also have fewer material and financial resources at hand -- and
often less autonomy -- to help themselves cope with shocks like natural disasters. Negative implications of climate change on all the sectors discussed above are
likely to disproportionately affect women in both direct and indirect ways (NAP Stocktaking Report 2016).

During NAPA consultation process, stakeholders recommended promotion of gender, disability and other socially excluded vulnerable groups in implementation of
climate change adaptation interventions as one of the thematic areas. The Malawi Growth and Development Strategy III includes the following strategies to promote
on gender mainstreaming:

\begin{itemize}
\item
  Increasing equitably access, control and utilization of social and economic services by youth and women
\item
  Ensuring the consistent implementation of gender responsive budgeting across sectors
\item
  Ensuring the effective participation of children, youth and women in decision making processes
\item
  Increasing youth and women participation in the economy including development initiatives at all levels.
\end{itemize}

These strategic priorities on gender should be reflected in adaptation programme design.

\hypertarget{further-development-of-the-programme-to-support-future-naps}{%
\chapter{Further development of the programme to support future NAPs}\label{further-development-of-the-programme-to-support-future-naps}}

\begin{enumerate}
\def\labelenumi{\arabic{enumi}.}
\tightlist
\item
\item
\item
\item
\item
\item
\item
\item
\item
\item
\item
\end{enumerate}

\hypertarget{data-and-system-observations-to-support-future-assessments}{%
\section{Data and system observations to support future assessments}\label{data-and-system-observations-to-support-future-assessments}}

The required data and system observations to support future assessments have been outlined in sections 8.3 and 8.4. Each of these sections need to be updated during
each assessment to reflect the gaps existing at the time of the assessment.

\#\#\# Roadmap for review and update of the NAP in five years
- To be developed with stakeholders

\hypertarget{annex-i-nap-outputs}{%
\chapter{Annex I: NAP Outputs}\label{annex-i-nap-outputs}}

xxx

\hypertarget{annex-2-country-profile}{%
\chapter{Annex 2: Country Profile}\label{annex-2-country-profile}}

xxxx

\hypertarget{annex-3-data-and-information-system-to-support-the-nap}{%
\chapter{Annex 3: Data and information system to support the NAP}\label{annex-3-data-and-information-system-to-support-the-nap}}

xxx

\hypertarget{alignment-with-the-gcf-country-programme}{%
\chapter{Alignment with the GCF Country Programme}\label{alignment-with-the-gcf-country-programme}}

xxx

\hypertarget{references}{%
\chapter{References}\label{references}}

\hypertarget{refs}{}
\begin{CSLReferences}{0}{0}
\end{CSLReferences}

  \bibliography{book.bib,packages.bib}

\end{document}
